
\chapter{Definície}\label{chap:definicie}

V tejto kapitole sme zaviedli niektoré pojmy, s ktorými sa budeme stretávať 
počas nasledujúcich kapitol. Sú to prevažne pojmy z teórie grafov. Keďže 
väčšina článkov, ktorými sme sa zaoberali v ďalších častiach je anglického 
pôvodu, rozhodli sme sa pre značenie uprednostniť knihu Graph Theory 
\citep{diestel} pred knihou Grafové algoritmy \citep{plesnik}. 

Základným pojmom je pre nás množina, čo je súbor navzájom rôznych objektov. 
Množinu prirodzených čísel vrátane nuly označujeme $\mnn$. Množinu celých 
čísel označujeme $\mnz$. Množinu reálnych čísel $\mnr$. Pre reálne číslo 
$x$ označujeme hornú celú časť $\lceil x\rceil$ a označuje najmenšie celé 
číslo väčšie alebo rovné ako $x$. Podobne dolnú celú časť označujeme 
$\lfloor x\rfloor$ a označuje najväčšie celé číslo menšie alebo rovné ako $x$. 
Základ logaritmov napísaných ako "$\log$" je 2 a základ logaritmov napísaných 
ako "$\ln$" je $\eul$. Množina $\mathcal = \{A_1, \ldots, A_k\}$ navzájom 
disjunktných podmnožín množina $A$ je \emph{rozdelenie} ak 
$A = \bigcup_{i=1}^{k} A_i$ a všetky $i$ platí, že $A_i = \emptyset$.

\section{Grafy}

Graf $G = (V, E)$ je usporiadaná dvojica množiny vrcholov a množiny hrán, ktorá 
má následujúce vlastnosti: 
\begin{itemize}
	\item $V \cap E = \emptyset$ (hrany a vrcholy sú rozlíšiteľné)
	\item $E \subseteq\left\{ \left\{ u, v\right\} : u \neq v; u, v \in V\right\} $ 
		(hrana spája dva vrcholy)
\end{itemize}

Graf sa zvyčajne znázorňuje nakreslením bodov pre každý vrchol a čiar medzi 
dvoma bodmi tam, kde existuje hrana. Rozmiestenie bodov a čiar nemá význam. 

O grafe s množinou vrcholov $V$ hovoríme, že je grafom \emph{na} $V$. 
\emph{Množina 
vrcholov grafu} $G$ je označovaná $V(G)$ a to aj v prípade, kedy graf $G$ má za 
množinu vrcholov inú množinu ako $V$. Napríklad, pre graf $H = (W, F)$ 
označujeme množinou vrcholov $V(H)$ a platí $V(H) = W$. Podobne označujeme 
\emph{množinu hrán grafu} $E(G)$ (v hore uvedenom príklade platí $E(H) = F$). 
Pre jednoduchosť hovoríme, že vrchol (hrana) patrí grafu a nie množine vrcholov 
(hrán) grafu a preto sa občas vyskytuje označenie $v \in G$ a nie $v \in V(G)$. 

\emph{Rád grafu} je počet vrcholov v grafe a označujeme ho ako $|G|$.
\emph{Prázdny graf} $(\emptyset, \emptyset )$ označujeme $\emptyset$. Grafy 
rádu $0$ alebo $1$ sa označujeme ako \emph{triviálne}.

Vrchol $v$ je incidentný s hranou $e$ ak platí $v \in e$. Hranu $\{x, y\}$ 
jednoduchšie označujeme ako $xy$. \emph{Hranami} $X-Y$ označujeme množinu 
hrán $E(X,Y) = \left\{ \left\{ x, y\right\} : x \in X, y \in Y\right\} $. 
Namiesto $E(\{x\},Y)$ píšeme $E(x,Y)$ a podobne aj namiesto $E(X,\{y\})$ 
píšeme $E(X,y)$. s
Zápisom $E(v)$ označujeme $E(v, V(G))$ a hovoríme o \emph{hranách vrchola} $v$. 

Dva vrcholy $x, y$ grafu $G$ sú \emph{susedia}, ak existuje hrana $xy$ v grafe 
$G$. Dve hrany $e \neq f$ sú susedné, ak majú spoločný jeden vrchol. Ak sú 
všetky vrcholy v grafe navzájom susedné, graf je \emph{kompletný} (alebo 
\emph{úplný}). Kompletný graf s $n$ vrcholmi označujeme $K^n$.

Majme dva grafy $G = (V,E)$ a $G' = (V', E')$. Potom $G \cup G' := (V \cup V', 
E \cup E')$. Podobne $G \cap G' := (V \cap V', E \cap E')$. Ak platí $G \cap G' 
= \emptyset$ tak hovoríme, že grafy sú \emph{disjunktné}. Ak platí $V \subseteq 
V'$ a $E \subseteq E'$, tak potom je graf $G'$ \emph{podgrafom} grafu $G$. 
Zapisujeme $G' \subseteq G$.

Ak $G' \subseteq G$ a $G'$ obsahuje všetky hrany $xy \in E$ pre $x,y \in V'$, 
tak hovoríme, že graf $G'$ je \emph{indukovaný podgraf} grafu $G$. Taktiež 
hovoríme, že $V'$ \emph{indukuje} $G'$ na $G$ a zapisujeme $G' =: G[V']$. 
Zápis $G[\{v\}]$ skracujeme na $G[v]$.

Ak je $U$ nejaká množina vrcholov, tak zápisom $G - U$ (operátory $-$ a 
$\setminus$ budeme občas zamieňať) označujeme $G[V(G) \setminus U]$. Inými 
slovami graf $G - U$ dosiahneme tak, že z grafu $G$ vymažeme všetky vrcholy z 
množiny $U$ a všetky incidentné hrany k nim. Pre $G - \{u\}$ používame aj zápis 
$G - u$.
Pre graf $G=(V,E)$ a množinu hrán $F = \{xy: x, y \in V\}$ zapisujeme 
$G + F = (V, E \cup F)$ a $G - F = (V, E \setminus F)$.

\emph{Komponent grafu} je taký maximálny podgraf, kde medzi každou dvojicou 
vrcholov existuje cesta.

\section{Vlastnosti grafu}

Uvažujme o grafe $G = (V, E)$. Množinu susedov vrchola $v$ označujeme
$N_G(v)$. Pokiaľ to bude z kontextu jasné, tak iba skrátene $N(v)$. Zápis 
rozšírme na množiny. Pre množinu $U \subseteq V$ zapisujeme $N(U)$ množinu 
susedov všetkých vrcholov $u \in U$. Množinu $N(U)$ nazývame \emph{susedmi} $U$. 
\emph{Susedov vrátane vrchola} označujeme susedov vrchola s vrcholom samotným. 
Pre vrchol $v$ susedov vrátane vrchola zapisujeme ako $N[v] := N(v) \cup \{v\}$. 
Podobne môžeme rozšíriť zápis aj na množiny. \emph{Pokrytím} množiny $U 
\subseteq V$ nazývame množinu $N[U] := N(U) \cup U$.

Číslo $d_G(v) = d(v) := |E_G(v)|$ sa nazýva \emph{stupeň} vrchola. Je to počet 
susedov vrchola (neplatí pre digrafy, multigrafy a iné zložité grafy, ktorými 
sa tu však nezaoberáme). Vrchol stupňa $0$ je \emph{izolovaný}. Číslo $\delta 
(G) := min \{d(v), v \in G\}$ je \emph{minimálny stupeň} grafu $G$. Podobne 
číslo $\Delta (G) := max \{d(v), v \in G\}$ je \emph{maximálny stupeň} grafu 
$G$.

\section{Cesta a cyklus}

\emph{Cesta} je graf $P = (V, E)$ v tvare: 
$$ V = \{x_0, x_1, x_2, x_3, \ldots, x_k\} \qquad 
   E = \{x_0x_1, x_1x_2, x_2x_3, \ldots, x_{k-1}x_k\},$$
kde všetky vrcholy $x_i$ sú navzájom rôzne. Vrcholy $x_0$ a $x_k$ sa nazývajú 
\emph{konce} cesty a zvyšné vrcholy sú \emph{vnútorné} vrcholy. Cestu 
zjednodušene označujeme sledom vrcholov: $P = x_0x_1x_2\cdots x_k$. Aj keď 
nevieme rozlíšiť medzi cestami $P_1 = x_0x_1x_2\cdots x_k$ a 
$P_2 = x_kx_{k-1}x_{k-2}\cdots x_0$, často si zvolíme jednu možnosť a hovoríme 
o \emph{ceste z $x_0$ do $x_k$} (v tomto prípade sme si vybrali cestu $P_1$). 
\emph{Dĺžka cesty} je číslo $k$ (cesta môže mať dĺžku 0).

\emph{Cyklus} je cesta $P = (V, E)$ v tvare 
$$ V = \{ x_0, x_1, x_2, x_3, \ldots, x_{k-1}\} \qquad 
E = \{x_0x_1, x_1x_2, x_2x_3, \ldots, x_{k-2}x_{k-1}, x_{k-1}x_0\}$$

O ceste má zmysel hovoriť iba ak $k \geq 3$. \emph{Dĺžka cyklu} je číslo $k$. 
Je to počet vrcholov (a zároveň aj hrán) v grafe.

V nasledujúcej časti si ukážeme dátové štruktúry, s ktorými sme v práci 
pracovali.

\section{Strom}

Strom je súvislý graf, ktorý má $n$ vrcholov a $n-1$ hrán.

\todo{treba uviesť základnú vlastnosť a nejaké kecy o tom, čo a ako}

\section{Toky}

Veľa vecí z reálneho sveta sa dá modelovať pomocou grafov alebo štruktúr 
podobných grafom. Ide napríklad o elektrickú rozvodnú sieť, cestnú sieť, 
vlakovú/dráhovú sieť, komunikačnú sieť. Pri týchto znázorneniach vystupujú 
vždy dvojice komunikácií a "križovatiek" (elektrické vedenie s trafostanicami, 
cesty s mestami, dráhy so zástavkami, linky s prepojovacími stanicami). Každá 
komunikácia má svoju kapacitu. V týchto štruktúrach má zmysel sa pýtať otázky, 
ako napríklad koľko veľa prúdu, zásob, dát dokáže prúdiť medzi dvoma vrcholmi. 
V teórii grafov hovoríme o tokoch. 

Konkrétne komunikáciu si môžeme predstaviť ako hranu $e = xy$, ktorá vyjadruje 
aj smer prúdenia. K usporiadanej dvojici $(x, y)$ môžeme priradiť hodnotu $k$ 
vyjadrujúcu kapacitu komunikácie. Znamená to, že $k$ jednotiek môže prúdiť z 
vrcholu $x$ do vrcholu $y$. Alebo usporiadanej dvojici $(x, y)$ môžeme priradiť
zápornú hodnotu $-k$ a to znamená, že $k$ jednotiek prúdi opačným smerom. To 
znamená, že pre zobrazenie $f:V^2\rightarrow\mnz$ (množina V označuje vrcholy), 
bude platiť, že $f(x,y) = -f(y,x)$, keď $x$ a $y$ sú susedné vrcholy.

Keď už máme vrcholy a komunikácie, musíme mať aj \emph{zdroj} vecí, ktoré po 
komunikáciach budú prúdiť a taktiež miesta, z ktorých budú tieto veci odchádzať 
z modelu. Tie sa označujú ako \emph{stoky}. Okrem týchto špeciálnych vrcholov 
platí, že $$\sum_{y\in N(x)}^{}{f(x,y) = 0}$$

Pokiaľ platia pre graf $G := (V, E)$ a zobrazenie $f:V^2\rightarrow\mnz$ 
vlastnosti $f(x,y) = -f(y,x)$ pre susedné vrcholy $x$ a $y$ a pre vrcholy mimo 
zdrojov a stôk, že $\sum_{y\in N(x)}^{}{f(x,y) = 0}$, tak budeme hovoriť o 
\emph{toku} na grafe $G$.

Graf sme si zadefinovali ako štruktúru, ktorá má hrany \emph{neorientované}, 
to znamená, že nevieme rozlíšiť, kde hrana "začína" a kde "končí". Pri tokoch 
sme ale začali rozlišovať túto vlastnosť a tým sa hrany stali 
\emph{orientovanými}. Grafy sa nazývajú neorientované, pokiaľ sú aj ich hrany 
neorientované a naopak, ak sú orientované hrany, tak hovoríme aj o 
orientovanom grafe. V ďalšom texte budeme orientovanosť uvádzať iba vtedy, keď 
nebude jasne vyplávať z kontextu.

\section{Siete}

Ďalšie veci, ktoré môžeme pri modelovaní sietí z reálneho sveta skúmať sú 
veci ohľadne štruktúry. Má zmysel sa pýtať na to, aký je priemer grafu, či vieme 
určiť hierarchiu vrcholov a jej, aké komunikácie treba prerušiť na to, aby 
sa sieť rozpadla, kam treba nasadiť obmedzený počet špiónov, aby sme získali čo 
najviac informácií a podobne.

Pri modeloch reálnych sietí má zmysel zaoberať sa nielen stupňami jednotlivých 
vrcholov ale aj distribúciou stupňov a priemerným stupňom vrchola. 
\emph{Priemerný stupeň grafu} $G = (V, E)$ označujeme $\overline{d_G} = \bar{d}$ a 
vypočítame ako: $$\overline{d_G} = \bar{d} = \frac{\sum_{v \in V}^{}{d_v}}{|V|}$$

K zadefinovaniu distribúcie budeme potrebovať ešte jednu funkciu. Táto funkcia 
vracia hodnotu 1 v bode 0 a vo všetkých ostatných bodoch vracia hodnotu 0. 
Takže dobre slúži ako filter. Označme ju $\tau$ a zadefinujme: 
$$\tau (n) = \begin{cases}
\hfill 1 \hfill & \text{ak $n=1$} \\
\hfill 0 \hfill & \text{inak} \\
\end{cases}$$

\emph{Distribúcia stupňov vrcholov} grafu $G$ je funkcia 
$p(d)$ reprezentujúca pravdepodobnosť toho, že vrchol má stupeň $d$. Platí, že: 
$$p(d) = \frac{\sum_{v \in V}^{}{\tau(d - d_v)}}{|V|}$$\todo{vysvetli, čo to znamená..}

Suma vo funkcii prechádza všetkými vrcholmi a vďaka filtračnej funkcii $\tau$ 
spočíta, koľko je vrcholov stupňa $d$.

Zaujímavou vlastnosťou grafu je \emph{klasterizačný koeficient vrchola}. Je 
definovaný ako pomer počtu hrán medzi susediacimi vrcholmi daného vrchola a 
všetkými možnými (aj potenciálnymi) hranami medzi susediacimi vrcholmi. 
Formálne, pre graf $G := (V, E)$ je \emph{klasterizačný koeficient} vrchola 
$v$ hodnota: $$c_v = \frac{|E(N(v))|}{\binom{|N(v)|}{2}}$$

\emph{Priemerný klasterizačný koeficient} $\bar{c}$ grafu $G$ je definovaný ako:
$$\bar{c} = \frac{\sum_{v\in V}^{c_v}}{|V|}$$

Podobne ako distribúciu stupňov vrcholov zadefinujeme aj distribúciu 
klasterizačných koeficientov. \emph{Distribúcia klasterizačných koeficientov 
grafu} $G$ je funkcia $c(d)$, ktorá hovorí o tom, aký je priemer
klasterizačných koeficientov pre všetky vrcholy stupňa $d$. Vypočítame ho ako: 
$$c(d) = \frac{\sum_{v \in V}^{}{\tau(d - d_v)c_v}}
{\sum_{v \in V}^{}{\tau(d - d_v)}}$$


V následujúcich kapitolách sme sa zamerali a prebrali si jednotlivé existujúce 
algoritmy, ktoré boli implementované.

\begin{table}[h]
	\begin{tabular}{lllllllll}
		\hline
		& naive & greedy & greedyQ & ch7alg33 & ch7alg34OT & ch7alg35OT & fnaive   & fproper \\ \hline
		ba10    & 0.085 & 0.001  & 0.002   & 0        & 0.003      & 0.004      & 0.008    & 0.009   \\
		ba18    & 1.284 & 0.001  & 0.002   & 0        & 0.013      & 0.008      & 0.035    & 0.035   \\
		ba20    & 4.44  & 0.001  & 0.002   & 0        & 0.011      & 0.008      & 0.049    & 0.047   \\
		ba100   & -     & 0.015  & 0.004   & 0.004    & 0.07       & 0.045      & 0.473    & 0.506   \\
		ba200   &       & 0.026  & 0.006   & 0.009    & 0.135      & 0.073      & 30.515   & 30.412  \\
		ba1000  &       & 0.087  & 0.045   & 0.037    & 0.883      & 0.512      &          &         \\
		ba2000  &       & 0.142  & 0.095   & 0.053    & 2.524      & 0.988      &          &         \\
		ba10k   &       & 1.844  & 0.362   & 0.54     & 45.781     & 6.662      &          &         \\
		ba20k   &       & 8.813  & 1.294   & 3.8      &            & 22.256     &          &         \\
		ba100k  &       & 66762  & 68.669  & 135.197  &            &            &          &         \\
		rnd10   &       & 0.001  & 0.002   &          &            &            & 0.003    & 0.003   \\
		rnd15   &       & 0.001  & 0.002   &          &            &            & 0.048    & 0.043   \\
		rnd20   &       & 0.001  & 0.002   &          &            &            & 0.107    & 0.112   \\
		rnd100  &       & 0.011  & 0.004   &          &            &            & 1293.811 & 1302.94 \\
		rnd200  &       & 0.032  & 0.008   &          &            &            &          &         \\
		rnd1000 &       & 0.072  & 0.043   &          &            &            &          &         \\
		rnd2000 &       & 0.114  & 0.096   &          &            &            &          &         \\
		rnd10k  &       & 2.484  & 0.44    &          &            &            &          &         \\
		rnd20k  &       & 11.281 & 1.671   & 4.947    &            &            &          &         \\
		zly10   &       & 0.005  & 0.003   & 0.003    & 0.04       & 0.018      & 0.119    & 0.117   \\
		zly20   &       & 0.022  & 0.01    & 0.022    & 0.098      & 0.036      & 0.809    & 0.814   \\
		zly100  &       & 0.09   & 0.151   & 0.21     & 6.213      & 2.107      &          &         \\ \hline
	\end{tabular}
\end{table}

\begin{landscape}
\begin{table}[h]
\centering
	\begin{tabular}{llllllllll}
		& naive & greedy & greedyQ & ch7alg33 & ch7alg34OT & ch7alg35OT & fnaive   & fproper & macov \\
		ba10    & 0.085 & 0.001  & 0.002   & 0        & 0.003      & 0.004      & 0.008    & 0.009   & 1     \\
		ba18    & 1.284 & 0.001  & 0.002   & 0        & 0.013      & 0.008      & 0.035    & 0.035   & 1     \\
		ba20    & 4.44  & 0.001  & 0.002   & 0        & 0.011      & 0.008      & 0.049    & 0.047   & 1     \\
		ba100   & -     & 0.015  & 0.004   & 0.004    & 0.07       & 0.045      & 0.473    & 0.506   & 1     \\
		ba200   &       & 0.026  & 0.006   & 0.009    & 0.135      & 0.073      & 30.515   & 30.412  & 1     \\
		ba1000  &       & 0.087  & 0.045   & 0.037    & 0.883      & 0.512      &          &         & 2     \\
		ba2000  &       & 0.142  & 0.095   & 0.053    & 2.524      & 0.988      &          &         &       \\
		ba10k   &       & 1.844  & 0.362   & 0.54     & 45.781     & 6.662      &          &         &       \\
		ba20k   &       & 8.813  & 1.294   & 3.8      &            & 22.256     &          &         &       \\
		ba100k  &       & 66762  & 68.669  & 135.197  &            &            &          &         & 1     \\
		rnd10   &       & 0.001  & 0.002   &          &            &            & 0.003    & 0.003   & 1     \\
		rnd15   &       & 0.001  & 0.002   &          &            &            & 0.048    & 0.043   & 2     \\
		rnd20   &       & 0.001  & 0.002   &          &            &            & 0.107    & 0.112   & 3     \\
		rnd100  &       & 0.011  & 0.004   &          &            &            & 1293.811 & 1302.94 & 3     \\
		rnd200  &       & 0.032  & 0.008   &          &            &            &          &         & 5     \\
		rnd1000 &       & 0.072  & 0.043   &          &            &            &          &         &       \\
		rnd2000 &       & 0.114  & 0.096   &          &            &            &          &         & 5     \\
		rnd10k  &       & 2.484  & 0.44    &          &            &            &          &         & 5     \\
		rnd20k  &       & 11.281 & 1.671   & 4.947    &            &            &          &         & 5     \\
		zly10   &       & 0.005  & 0.003   & 0.003    & 0.04       & 0.018      & 0.119    & 0.117   & 5     \\
		zly20   &       & 0.022  & 0.01    & 0.022    & 0.098      & 0.036      & 0.809    & 0.814   & 5     \\
		zly100  &       & 0.09   & 0.151   & 0.21     & 6.213      & 2.107      &          &         & 5    
	\end{tabular}
\end{table}
\end{landscape}
