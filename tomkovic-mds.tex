\documentclass[a4paper,oneside,12pt]{book} 
 
\usepackage[xetex]{graphicx}
\usepackage{fontspec}
\usepackage{xunicode}
\usepackage{amsfonts}
\usepackage{amsmath}
%\defaultfontfeatures{Mapping=tex-text,Ligatures={Common}}
\defaultfontfeatures{Mapping=tex-text,Scale=MatchLowercase,Ligatures={Common}}
%\setmonofont{LMMonoLt10}
\setmainfont{Junicode}
%\setromanfont{Latin Modern Roman}
%\setromancapsfont{Latin Modern Roman}
% \setmainfont[
%  BoldFont={Junicode}, 
%  ItalicFont={Junicode},
%  BoldItalicFont={Junicode}
%  ]{Gentium}

%\setromanfont{Times New Roman}
%\setromanfont{DejaVu Serif}
%\setromanfont{Linux Libertine O}
% \setsansfont[Mapping=tex-text]{DejaVu Sans}
% \setmonofont[Mapping=tex-text]{DejaVu Sans Mono}

\newICUfeature{LigType}{disc}{+dlig}

\usepackage{polyglossia}
%\setdefaultlanguage{czech}
\setdefaultlanguage{slovak}
%\PolyglossiaSetup{slovak}{indentfirst=true}

%\usepackage[slovak, english]{babel}
%\usepackage[english,slovak]{babel}

%\usepackage{csquotes}
%\DeclareQuoteStyle{slovak}
%{\quotedblbase}
%{\textquotedblleft}
%[0.05em]
%{\quotesinglbase}
%{\fixligatures\textquoteleft}


\usepackage{url}
\usepackage[xetex]{hyperref}

\usepackage[
   backend=biber      % if we want unicode 
%  ,style=iso-authoryear % iso-authoryear or iso-numeric for numeric citation method
  ,style=authoryear 
%  ,style=authoryear-comp 
  %,babel=other        % to support multiple languages in bibliography
  %,autolang=other % to support multiple languages in bibliography
  ,autolang=langname % to support multiple languages in bibliography
  %,sortlocale=cs_CZ   % locale of main language, it is for sorting
  ,bibencoding=UTF8   % this is necessary only if bibliography file is in different encoding than main document
]{biblatex}
\addbibresource{literatura.bib}

\usepackage{pdfpages}
\usepackage{xcolor}
\usepackage[head=20pt,a4paper]{geometry}
\geometry{top=2.5cm,bottom=2.5cm,left=3.5cm,right=2.0cm}
\usepackage{setspace}
\onehalfspacing
\usepackage{changepage}
\usepackage{indentfirst}

\usepackage{fancyhdr}
\fancypagestyle{plain}{ %
\fancyhf{} %
% \fancyfoot[LE,RO]{\thepage} %
\fancyfoot[RO]{\thepage} %
\renewcommand{\headrulewidth}{0pt} % remove lines as well
\renewcommand{\footrulewidth}{0pt} %
}

%\usepackage{enumerate}
\usepackage{subfig}
\usepackage{array}

\usepackage{comment}

\usepackage{pdflscape}

%%definície konštánt a makier
\def \nazovUniverzity{Univerzita Komensk{é}ho v Bratislave}
\def \NazovUniverzity{\textbf{\Large \sc \nazovUniverzity}} %\sc pre smallcaps
\def \nazovFakulty{Fakulta matematiky, fyziky a informatiky}
\def \NazovFakulty{\textbf{\large \sc \nazovFakulty}}
\def \nazovDiela{Efektívne hľadanie minimálne dominujúcej množiny na reálnych sieťach}
\def \NazovDiela{\LARGE{\em \nazovDiela}}
\def \podnazovDiela{nie je}
\def \PodazovDiela{\large{\em \podnazovDiela}}
\def \typPrace{Diplomová práca}
\def \TypPrace{\large{\typPrace}}
\def \autor{Viktor Tomkovič}
\def \veduci{Mgr. Martin Čajági}
\def \rok{2015}
\def \miestoRok{Bratislava, \rok}
\def \cisloOdboru{2511}
\def \katedra{Katedra aplikovanej informatiky FMFI}
\def \odbor{Aplikovaná informatika}
\def \program{Aplikovaná informatika}
\def\citet#1{\textcite{#1}}
\def\citep#1{\parencite{#1}}
%citat: argumenty sú citát a meno autora
\newcommand{\citat}[2]{\begin{flushright}\small\textsl{#1}\\#2\end{flushright}}
%strednp je vystredenie textu na nepárnej strane bez ohľadu na okraje(!!!), parametrom je text
\newcommand{\stredp}[1]{\begin{adjustwidth}{+0.50cm}{+0.00cm} \begin{center} #1 \end{center} \end{adjustwidth}}
%stredp je obdoba predošlého príkazu len pre párne strany
\newcommand{\strednp}[1]{\begin{adjustwidth}{+0.0cm}{+0.50cm} \begin{center} #1 \end{center} \end{adjustwidth}}
\def\todo#1{\par\smallskip{\scriptsize {\bf TODO: }#1}\par\smallskip}

\setlength\fboxsep{0pt}
\setlength\fboxrule{0.5pt}
\definecolor{grey}{gray}{0.6}
\newcommand{\mybox}[2]{{\color{#1}\fbox{\normalcolor#2}}}
\newcommand{\greybox}[1]{\mybox{grey}{#1}}

% \setlength{\parindent}{0.5in}
% \setlength{\parskip}{0.45cm}
\setlength{\parindent}{0.25in}
%\setlength{\parskip}{0.3em}

\newcommand{\mnn}{\mathbb{N}}
\newcommand{\mnz}{\mathbb{Z}}
\newcommand{\mnr}{\mathbb{R}}
\newcommand{\eul}{e}

\DeclareMathOperator*{\argmax}{arg\,max}
\DeclareMathOperator*{\argmin}{arg\,min}

\def\Java{\texttt{JAVA}}
\def\cpp{\texttt{C++}}


\def\uv#1{„#1“}
\catcode`\"=\active \def "{\begingroup„\def "{\endgroup“}}
\begin{document}
\frontmatter
\pagestyle{plain}
\thispagestyle{empty}
\noindent
\strednp{
\NazovUniverzity\\
\NazovFakulty}
\vfill
\strednp{
\NazovDiela
%% \centerline{\PodazovDiela}
\mbox{}\\
\bigskip
\TypPrace
}
\vfill
\strednp{{\rok} \hfill{\autor}}
\newpage

\thispagestyle{empty}
\noindent
\strednp{\NazovUniverzity\\ \NazovFakulty}
\vfill
\strednp{\NazovDiela
%% \centerline{\PodazovDiela}
\mbox{}\\
\bigskip
\TypPrace
}
\vfill
\begin{tabular}{ l l }
\textbf{Študijný program:} & \program\\
\textbf{Študijný odbor:} & \cisloOdboru\ \odbor\\
\textbf{Školiace pracovisko:} & \katedra\\
\textbf{Vedúci práce:} &  \veduci
\end{tabular}
\bigskip\\
\bigskip\\
\bigskip\\
\bigskip\\
\strednp{\miestoRok \hfill{\autor}}
\newpage

\includepdf[pages=1]{zadanie.pdf}

\newpage

\noindent
~\vfill

\section*{Poďakovanie}
Ďakujem.
\begin{comment}
Za odbornú pomoc, poskytnuté materiály a výborné vedenie pri tejto práci patrí 
veľká vďaka môjmu školiteľovi Martinovi Čajágimu. Taktiež by som sa chcel 
poďakovať mojim rodičom za pochopenie a poskytnuté možnosti vzdelávať sa.
\end{comment}
\\
\bigskip\\
\newpage

\chapter*{Abstrakt}
Táto práca skúma rôzne algoritmy na riešenie problému hľadania minimálnych
dominantných množín (MDS). Konkrétne skúma ich využitie v sieťach malého sveta.\\
Kľúčové slová: minimálna dominujúca množina, MDS, algoritmy a dátové štruktúry, siete 
malého sveta.

\newpage

\chapter*{Abstract}
This work is.\\
Keywords: minimal dominating set, MDS, algorithms and date structures, small-world 
network.
\newpage

\mbox{}
\newpage

\tableofcontents\newpage
%\listoffigures
%\listoftables

\mainmatter
% \pagestyle{headings}
\pagestyle{fancy}
\fancyhf{} 
% \fancyfoot[LE,RO]{\thepage} 
%  \renewcommand{\chaptermark}[1]{%
% \markboth{\MakeUppercase{%
% \chaptername}\ \thechapter.%
% \ #1}{}}
\renewcommand{\chaptermark}[1]{%
\markboth{\MakeUppercase{%
\chaptername\ \thechapter.%
\ #1}}{}}
% \fancyhead[LE,RO]{\leftmark}
\fancyhead[RO]{\leftmark}
\fancyfoot[RO]{\thepage} 
\renewcommand{\headrulewidth}{0pt} % remove lines as well
\renewcommand{\footrulewidth}{0pt} %
% \renewcommand{\chaptermark}[1]{\markboth{#1}{}} 
% \renewcommand{\sectionmark}[1]{\markright{#1}{}}
\cleardoublepage
% \phantomsection
\addcontentsline{toc}{chapter}{Úvod}
\chapter*{Úvod}\label{chap:intro}

Modely sietí majú veľký úspech. Najmä vďaka tomu, že grafy vytvorené podľa 
týchto modelov majú podobné vlastnosti ako skutočné siete. Internet, elektrické 
rozvody, sociálne kontakty, internetové sociálne siete, či neurónové prepojenia 
v mozgu majú navzájom podobné vlastnosti. A tieto vlastnosti sú zachytené aj v 
modeloch sietí. Prvými priekopníkmi v skúmaní reálnych sietí bolí maďarskí 
vedci Paul Erdős, Alfréd Rényi and Béla Bollobás. Reálne siete sa snažili 
popísať pomocou náhodných modelov \citep{erdos:rnd}. Čoskoro zistili, že 
siete v reálnom svete tieto modely nepopisujú veľmi dobre. Siete v reálnom 
svete vznikajú inak ako náhodne a tak sa vedci snažili o vernejší popis 
modelmi, ktoré by zachytávali vlastnosti reálnych sietí. Siete malého sveta 
boli spopularizované po tom, ako Stenley Milgram spravil výskum, v ktorom 
potvrdzoval tézu, že v Spojených Štátoch amerických sa všetci 
ľudia poznajú vo väčšine cez najviac šiestich známych \citep{kochen}. Na to 
popísali \citet{barabasi:albert} bezškálové siete, ktoré boli charakteristické 
krátkymi cestami medzi náhodnými dvojicami vrcholov, podobnosťou na rôznych 
škálach a odolnosťou proti náhodným útokom \citep{barabasi:albert:2}.

Reálne siete môžeme skúmať z viacerých pohľadov. Jedným z nich je aj hľadanie 
množiny vrcholov, ktoré sú pre danú sieť významné. Pre sieť kontaktov to budú 
ľudia, ktorý majú tých kontaktov najviac, pre sieť stretnutí sa rôznych ľudí 
to môžu byť ľudia, ktorý sa stretli s veľa inými ľuďmi a podobne. V týchto 
úlohách často hľadáme nejakým spôsobom vrcholy s najväčším stupňom alebo 
množinu, ktorá má za susedov všetkých ostatných ľudí. Takáto množina sa nazýva 
minimálna dominujúca množina. Ide o známu úlohu z teórie grafov, ktorého 
rozhodovacia verzia je NP-úplný problém \citep{npcomp}. Keďže reálne siete sa 
skladajú z tisícov vrcholov, výpočtová sila súčasnosti nestačí na presné 
riešenie týchto problémov. Preto je momentálnym riešením použitie 
aproximačných alebo pažravých algoritmov. Keďže považujeme túto tému za 
zaujímavú, rozhodli sme sa preskúmať bližšie rôzne heuristiky pažravého 
algoritmu. Taktiež výsledky z algoritmov dosiahnuté v tejto práci môžu pomôcť 
iným prácam k lepšiemu vstupu/spracovaniu, napríklad k práci, ktorú napísala 
Dana \citet{sunikova}.

Práca je rozdelená do piatich kapitol. Najprv zadefinujeme rôzne pojmy, ktoré 
v práci používame. To spravíme v kapitole~\ref{chap:definicie}. Potom uvedieme 
prehľad existujúcich algoritmov na hľadanie minimálnych dominujúcich množín 
(kapitola~\ref{chap:algoritmy}). Následne v kapitolách \ref{chap:popis} a 
\ref{chap:implementacia} popíšeme návrh a implementáciu softvéru. Výsledky 
implementačnej práce zhrnieme v kapitole~\ref{chap:vysledky}.






\chapter{Definície}\label{chap:def}

V tejto kapitole sme zaviedli niektoré pojmy, s ktorými sa budeme stretávať 
počas nasledujúcich kapitol. Sú to prevažne pojmy z teórie grafov. Keďže 
väčšina článkov, ktorými sme sa zaoberali v ďalších častiach je anglického 
pôvodu, rozhodli sme sa pre značenie uprednostniť knihu Graph Theory 
\citep{diestel} pred knihou Grafové algoritmy \citep{plesnik}. 

Základným pojmom je pre nás množina, čo je súbor navzájom rôznych objektov. 
Množinu prirodzených čísel vrátane nuly označujeme $\mnn$. Množinu celých 
čísel označujeme $\mnz$. Množinu reálnych čísel $\mnr$. Pre reálne číslo 
$x$ označujeme hornú celú časť $\lceil x\rceil$ a označuje najmenšie celé 
číslo väčšie alebo rovné ako $x$. Podobne dolnú celú časť označujeme 
$\lfloor x\rfloor$ a označuje najväčšie celé číslo menšie alebo rovné ako $x$. 
Základ logaritmov napísaných ako "$\log$" je 2 a základ logaritmov napísaných 
ako "$\ln$" je $\eul$. Množina $\mathcal = \{A_1, \ldots, A_k\}$ navzájom 
disjunktných podmnožín množina $A$ je \emph{rozdelenie} ak 
$A = \bigcup_{i=1}^{k} A_i$ a všetky $i$ platí, že $A_i = \emptyset$.

\section{Grafy}

Graf $G = (V, E)$ je usporiadaná dvojica množiny vrcholov a množiny hrán, ktorá 
má následujúce vlastnosti: 
\begin{itemize}
	\item $V \cap E = \emptyset$ (hrany a vrcholy sú rozlíšiteľné)
	\item $E \subseteq\left\{ \left\{ u, v\right\} : u \neq v; u, v \in V\right\} $ 
		(hrana spája dva vrcholy)
\end{itemize}

Graf sa zvyčajne znázorňuje nakreslením bodov pre každý vrchol a čiar medzi 
dvoma bodmi tam, kde existuje hrana. Rozmiestenie bodov a čiar nemá význam. 

O grafe s množinou vrcholov $V$ hovoríme, že je grafom \emph{na} $V$. 
\emph{Množina 
vrcholov grafu} $G$ je označovaná $V(G)$ a to aj v prípade, kedy graf $G$ má za 
množinu vrcholov inú množinu ako $V$. Napríklad, pre graf $H = (W, F)$ 
označujeme množinou vrcholov $V(H)$ a platí $V(H) = W$. Podobne označujeme 
\emph{množinu hrán grafu} $E(G)$ (v hore uvedenom príklade platí $E(H) = F$). 
Pre jednoduchosť hovoríme, že vrchol (hrana) patrí grafu a nie množine vrcholov 
(hrán) grafu a preto sa občas vyskytuje označenie $v \in G$ a nie $v \in V(G)$. 

\emph{Rád grafu} je počet vrcholov v grafe a označujeme ho ako $|G|$.
\emph{Prázdny graf} $(\emptyset, \emptyset )$ označujeme $\emptyset$. Grafy 
rádu $0$ alebo $1$ sa označujeme ako \emph{triviálne}.

Vrchol $v$ je incidentný s hranou $e$ ak platí $v \in e$. Hranu $\{x, y\}$ 
jednoduchšie označujeme ako $xy$. \emph{Hranami} $X-Y$ označujeme množinu 
hrán $E(X,Y) = \left\{ \left\{ x, y\right\} : x \in X, y \in Y\right\} $. 
Namiesto $E(\{x\},Y)$ píšeme $E(x,Y)$ a podobne aj namiesto $E(X,\{y\})$ 
píšeme $E(X,y)$. s
Zápisom $E(v)$ označujeme $E(v, V(G))$ a hovoríme o \emph{hranách vrchola} $v$. 

Dva vrcholy $x, y$ grafu $G$ sú \emph{susedia}, ak existuje hrana $xy$ v grafe 
$G$. Dve hrany $e \neq f$ sú susedné, ak majú spoločný jeden vrchol. Ak sú 
všetky vrcholy v grafe navzájom susedné, graf je \emph{kompletný} (alebo 
\emph{úplný}). Kompletný graf s $n$ vrcholmi označujeme $K^n$.

Majme dva grafy $G = (V,E)$ a $G' = (V', E')$. Potom $G \cup G' := (V \cup V', 
E \cup E')$. Podobne $G \cap G' := (V \cap V', E \cap E')$. Ak platí $G \cap G' 
= \emptyset$ tak hovoríme, že grafy sú \emph{disjunktné}. Ak platí $V \subseteq 
V'$ a $E \subseteq E'$, tak potom je graf $G'$ \emph{podgrafom} grafu $G$. 
Zapisujeme $G' \subseteq G$.

Ak $G' \subseteq G$ a $G'$ obsahuje všetky hrany $xy \in E$ pre $x,y \in V'$, 
tak hovoríme, že graf $G'$ je \emph{indukovaný podgraf} grafu $G$. Taktiež 
hovoríme, že $V'$ \emph{indukuje} $G'$ na $G$ a zapisujeme $G' =: G[V']$. 
Zápis $G[\{v\}]$ skracujeme na $G[v]$.

Ak je $U$ nejaká množina vrcholov, tak zápisom $G - U$ (operátory $-$ a 
$\setminus$ budeme občas zamieňať) označujeme $G[V(G) \setminus U]$. Inými 
slovami graf $G - U$ dosiahneme tak, že z grafu $G$ vymažeme všetky vrcholy z 
množiny $U$ a všetky incidentné hrany k nim. Pre $G - \{u\}$ používame aj zápis 
$G - u$.
Pre graf $G=(V,E)$ a množinu hrán $F = \{xy: x, y \in V\}$ zapisujeme 
$G + F = (V, E \cup F)$ a $G - F = (V, E \setminus F)$.

\section{Vlastnosti grafu}

Uvažujme o grafe $G = (V, E)$. Množinu susedov vrchola $v$ označujeme
$N_G(v)$. Pokiaľ to bude z kontextu jasné, tak iba skrátene $N(v)$. Zápis 
rozšírme na množiny. Pre množinu $U \subseteq V$ zapisujeme $N(U)$ množinu 
susedov všetkých vrcholov $u \in U$. Množinu $N(U)$ nazývame \emph{susedmi} $U$. 
\emph{Susedov vrátane vrchola} označujeme susedov vrchola s vrcholom samotným. 
Pre vrchol $v$ susedov vrátane vrchola zapisujeme ako $N[v] := N(v) \cup \{v\}$. 
Podobne môžeme rozšíriť zápis aj na množiny. \emph{Pokrytím} množiny $U 
\subseteq V$ nazývame množinu $N[U] := N(U) \cup U$.

Číslo $d_G(v) = d(v) := |E_G(v)|$ sa nazýva \emph{stupeň} vrchola. Je to počet 
susedov vrchola (neplatí pre digrafy, multigrafy a iné zložité grafy, ktorými 
sa tu však nezaoberáme). Vrchol stupňa $0$ je \emph{izolovaný}. Číslo $\delta 
(G) := min \{d(v), v \in G\}$ je \emph{minimálny stupeň} grafu $G$. Podobne 
číslo $\Delta (G) := max \{d(v), v \in G\}$ je \emph{maximálny stupeň} grafu 
$G$.

\section{Cesta a cyklus}

\emph{Cesta} je graf $P = (V, E)$ v tvare: 
$$ V = \{x_0, x_1, x_2, x_3, \ldots, x_k\} \qquad 
   E = \{x_0x_1, x_1x_2, x_2x_3, \ldots, x_{k-1}x_k\},$$
kde všetky vrcholy $x_i$ sú navzájom rôzne. Vrcholy $x_0$ a $x_k$ sa nazývajú 
\emph{konce} cesty a zvyšné vrcholy sú \emph{vnútorné} vrcholy. Cestu 
zjednodušene označujeme sledom vrcholov: $P = x_0x_1x_2\cdots x_k$. Aj keď 
nevieme rozlíšiť medzi cestami $P_1 = x_0x_1x_2\cdots x_k$ a 
$P_2 = x_kx_{k-1}x_{k-2}\cdots x_0$, často si zvolíme jednu možnosť a hovoríme 
o \emph{ceste z $x_0$ do $x_k$} (v tomto prípade sme si vybrali cestu $P_1$). 
\emph{Dĺžka cesty} je číslo $k$ (cesta môže mať dĺžku 0).

\emph{Cyklus} je cesta $P = (V, E)$ v tvare 
$$ V = \{ x_0, x_1, x_2, x_3, \ldots, x_{k-1}\} \qquad 
E = \{x_0x_1, x_1x_2, x_2x_3, \ldots, x_{k-2}x_{k-1}, x_{k-1}x_0\}$$

O ceste má zmysel hovoriť iba ak $k \geq 3$. \emph{Dĺžka cyklu} je číslo $k$. 
Je to počet vrcholov (a zároveň aj hrán) v grafe.

V nasledujúcej časti si ukážeme dátové štruktúry, s ktorými sme v práci 
pracovali.

\section{Strom}

Strom je súvislý graf, ktorý má $n$ vrcholov a $n-1$ hrán.

\todo{treba uviesť základnú vlastnosť a nejaké kecy o tom, čo a ako}

\section{Toky}

Veľa vecí z reálneho sveta sa dá modelovať pomocou grafov alebo štruktúr 
podobných grafom. Ide napríklad o elektrickú rozvodnú sieť, cestnú sieť, 
vlakovú/dráhovú sieť, komunikačnú sieť. Pri týchto znázorneniach vystupujú 
vždy dvojice komunikácií a "križovatiek" (elektrické vedenie s trafostanicami, 
cesty s mestami, dráhy so zástavkami, linky s prepojovacími stanicami). Každá 
komunikácia má svoju kapacitu. V týchto štruktúrach má zmysel sa pýtať otázky, 
ako koľko veľa prúdu, zásob, dát dokáže prúdiť medzi dvoma vrcholmi. V teórii 
grafov hovoríme o tokoch. 

Konkrétne komunikáciu si môžeme predstaviť ako hranu $e = xy$, ktorá vyjadruje 
aj smer prúdenia. K usporiadanej dvojici $(x, y)$ môžeme priradiť hodnotu $k$ 
vyjadrujúcu kapacitu komunikácie. Znamená to, že $k$ jednotiek môže prúdiť z 
vrcholu $x$ do vrcholu $y$. Alebo usporiadanej dvojici $(x, y)$ môžeme priradiť
zápornú hodnotu $-k$ a to znamená, že $k$ jednotiek prúdi opačným smerom. To 
znamená, že pre zobrazenie $f:V^2\rightarrow\mnz$ (množina V označuje vrcholy), 
bude platiť, že $f(x,y) = -f(y,x)$, keď $x$ a $y$ sú susedné vrcholy.

Keď už máme vrcholy a komunikácie, musíme mať aj \emph{zdroj} vecí, ktoré po 
komunikáciach budú prúdiť a taktiež miesta, z ktorých budú tieto veci odchádzať 
z modelu. Tie sa označujú ako \emph{stoky}. Okrem týchto špeciálnych vrcholov 
platí, že $$\sum_{y\in N(x)}^{}{f(x,y) = 0}$$

Pokiaľ platí pre zobrazenie $f:V^2\rightarrow\mnz$, že $f(x,y) = -f(y,x)$ pre 
susedné vrcholy $x$ a $y$ a pre vrcholy mimo zdrojov a stôk, že 
$\sum_{y\in N(x)}^{}{f(x,y) = 0}$, tak budeme hovoriť o \emph{toku} na grafe 
$G := (V, E)$.

\section{Siete}

Ďalšie veci, ktoré môžeme pri modelovaní sietí z reálneho sveta skúmať sú 
veci ohľadne štruktúry. Má zmysel sa pýtať na to, aký je priemer grafy, čí vieme 
spraviť a aká je hierarchia vrcholov, aké komunikácie treba prerušiť na to, aby 
sa sieť rozpadla, kam treba nasadiť špiónov, aby sme získali čo najviac 
informácií a podobne.



V následujúcich kapitolách sme sa zamerali a prebrali si jednotlivé existujúce 
algoritmy, ktoré boli implementované.


\chapter{Prehľad algoritmov}\label{chap:algoritmy}

V tejto kapitole si spravíme prehľad algoritmov, ktoré existujú na nájdenie 
minimálnej dominujúcej množiny. Najprv zadefinujeme minimálnu dominujúcu 
množinu, neskôr určíme požiadavky na algoritmy a potom popíšeme jednotlivé 
konkrétne algoritmy.

\section{Dominujúce množiny}

\emph{Dominujúca množina} $S$ na grafe $G = (V, E)$ je podmnožina množiny 
vrcholov $V$ grafu $G$ taká, že každý vrchol grafu sa v množine nachádza 
alebo je susedný s dominujúcou množinou. Pre množinu platí: $N[S] = V$. 
Z definície je zrejmé, že o dominujúcich množinách má zmysel hovoriť iba pri 
grafoch s konečným počtom vrcholov.

\emph{Minimálna dominujúca množina} $S_M$ je dominujúca množina s najmenšou 
kardinalitou.

\emph{Dominančé číslo} $\gamma (G)$ grafu $G$ je kardinalita minimálnej 
dominujúcej množiny. Ak je množina $S_M$ minimálnou dominujúcou množinou, tak 
platí, že $\gamma (G) = |S_M|$.

Na tomto mieste zadefinujeme aj vrcholové pokrytie. Uvádzame ho tu preto, lebo 
problém nájdenia vrcholového pokrytia súvisí s problémom nájdenia minimálnej 
dominujúcej množiny. \emph{Vrcholové pokrytie} $S^\prime$ na grafe $G = (V, E)$ 
je podmnožina množiny vrcholov $V$ grafu $G$ taká, že každá hrana $xy \in E$ 
je incidentná s vrcholom vrcholového pokrytia. Pre množinu platí: 
$\forall xy \in E: x \in S^\prime \vee y \in S^\prime $

Podobne ako pri minimálnej dominujúcej množine, existuje aj minimálne vrcholové 
pokrytie.

Jedným zo spôsobov, ako vyriešiť problém nájdenia minimálnej dominujúcej 
množiny je previesť ho na problém množinového pokrytia. \emph{Množinové 
pokrytie} je súbor množín, ktorých zjednotenie obsahuje všetky prvky univerza. 
Formálne je daná usporiadaná dvojica $(\mathcal{S}, \mathcal{U})$, kde:

\begin{itemize}
	\item množina množín $\mathcal{S}$ obsahuje množiny $S_1, S_2, S_3, ..., 
		S_n$ také, že $\bigcup_{i = 1}^{n} S_i = \mathcal{U}$;
	\item množina $\mathcal{U}$ je množina všetkých prvkov a nazýva sa 
		\emph{univerzum}.
\end{itemize}

Množinové pokrytie je podmnožina $\mathcal{C} \subseteq \mathcal{S}$ množín, 
pre ktorú platí, že $\bigcup_{C \in \mathcal{C}} C = \mathcal{U}$.

\section{Požiadavky na algoritmy}

Táto práca ma za úlohu nájsť vhodný algoritmus na hľadanie minimálnej 
dominujúcej množiny. Ale pre reálne dáta. To znamená, že grafy, na ktorých 
budeme minimálnu dominujúcu množinu hľadať majú veľa vrcholov. Avšak problém 
nájdenia minimálnej dominujúcej množiny na grafe sa dá redukovať na problém 
vrcholového pokrytia, o ktorom vieme, že je NP-ťažký. To znamená, že na 
vyrátanie minimálnej dominujúcej množiny treba veľmi veľa výpočtového času na 
súčasných počítačoch. Keďže pre reálne požiadavky je častokrát lepší nejaký, 
aj keď nie optimálny výsledok, tak sme sa rozhodli, že algoritmus nemusí dávať 
optimálny výsledok, ale môže dať približný výsledok v rozumnom čase. Rozumný 
čas však neurčujeme absolútne, keďže počítače sa vyvíjajú a výpočtová sila sa 
zväčšuje, ale relatívne vzhľadom na ostatné algoritmy.

\subsection{Výhoda sieti malého sveta}

V tejto časti ukážeme dôkaz, že v sieťach malého sveta je počet hrán rádovo 
rovnaký ako počet vrcholov. A teda siete malého sveta sú riedke grafy. 

Budeme vychádzať z toho, že pre distribúciu stupňov vrcholov platí:
$$p(d) = kd^{-a} + q, 2 \leq a \leq 3$$

Pre dostatočne veľké siete môžeme nespojitú funkciu $p : \mnn \leftarrow \mnn$ 
aproximovať spojitou verziou $p : \mnr \leftarrow \mnr$. Ďalej o $q$ vieme, že 
$q \leq 1/V$, kde $V$ je počet vrcholov grafu. Pre počet hrán v grafe 
$G = (V, E)$ platí, že je to polovica súčtu stupňov grafu. Teda môžeme napísať, 
že: $$|E| = \frac{1}{2}\sum_{d = 1}^{\Delta(G)} p(d) \sim 
\frac{1}{2}\int_1^{\Delta (G)} \! p(x) \, \mathrm{d}x$$

Úpravou integrálu dostaneme:

\begin{align*}
\frac{1}{2}\int_1^{\Delta (G)} \! p(x) \, \mathrm{d}x \
&= \frac{1}{2}\int_1^{\Delta (G)} \! kd^{-a} + q \, \mathrm{d}x \\
&= \frac{1}{2}\left[\frac{1}{a - 1} \left(- k x^{- a + 1} + q x \
\left(a - 1\right)\right)\right]_1^{|V|}
\end{align*}

Keďže $2 \leq a \leq 3$ a integrujeme medzi $1$ a $\Delta (G)$, tak môžeme 
ohraničiť výpočet:

\begin{align*}
\frac{1}{2}\left[\frac{1}{a - 1} \left(- k x^{- a + 1} + q x \
\left(a - 1\right)\right)\right]_1^{\Delta (G)} \leq \
\frac{1}{2}\left[- \left( k x^{- 1} - q x \right)\right]_1^{\Delta (G)}
\end{align*}

Pre konštantu $q$ platí, že $q \leq \frac{1}{\Delta (G)}$. Inak by súčet v 
distribúcií bol väčší ako počet vrcholov.

\begin{align*}
\frac{1}{2}\left[- \left( k x^{- 1} - q x \right)\right]_1^{\Delta (G)} \leq \
\frac{1}{2}\left[- \left( k x^{- 1} - \
\frac{x}{\Delta (G)} \right)\right]_1^{\Delta (G)} 
\end{align*}

Ďalšou úpravou dostaneme:

\begin{align*}
\frac{1}{2}\left[- \left( k x^{- 1} - \frac{x}{\Delta (G)} \
\right)\right]_1^{\Delta (G)} \
\!\!\!\!&= \frac{1}{2}\left\{\left[- \left( k \Delta (G)^{- 1} - \
\frac{\Delta (G)}{\Delta (G)}\right)\right] - \
\left[- \left( k 1^{- 1} - \frac{1}{\Delta (G)}\right)\right] \right\}\\
&= \frac{1}{2}\left\{\left[- \left( k \Delta (G)^{- 1} - 1 \right)\right] - \
\left[- \left( k - \frac{1}{\Delta (G)} \right)\right] \right\} \\
&= \frac{1}{2}\left\{\left(k + 1\right) - \frac{k+1}{\Delta (G)} \right\}\\
&= \frac{1}{2}\left(k + 1\right)\left( 1 - \frac{1}{\Delta (G)} \right)\\
\end{align*}

Konštanta $k$ nemôže byť väčšia ako $|V|$. Inak by vrcholov stupňa 1 bolo viac 
ako vrcholov v grafe. Triviálne platí, že $\Delta (G) \leq |V|$. Takže 
odhad môžeme ďalej upraviť:
\begin{align*}
\frac{1}{2}\left(k + 1\right)\left( 1 - \frac{1}{\Delta (G)} \right)\
&=\frac{1}{2}\left(|V| + 1\right)\left( 1 - \frac{1}{|V|} \right)
\end{align*}

Takže platí, že:
\begin{align*}
|E|&\sim \frac{1}{2}\left(|V| + 1\right)\left( 1 - \frac{1}{|V|} \right)
\end{align*}
a teda aj, že počet hrán je $O(V)$.

Táto vlastnosť sietí malého sveta nám pomôže najmä pri voľnejšom výbere 
heuristík a pravidiel v algoritmoch uvedených ďalej. Keďže algoritmy vo 
všeobecnosti neuvažujú s týmto faktom, dáva nám možnosť vytvoriť zložitejší a 
stále rovnako efektívny algoritmus.

\subsection{Test, či je množina dominujúcou}

Keďže graf máme reprezentovaný susednosťou vrcholov, tak priamočiary algoritmus 
na zistenie, či je množina dominujúcou vyzerá následovne:

\begin{enumerate}
	\item Pre každý vrchol testovanej množiny pridaj do výslednej množiny 
všetkých susedov vrchola;
	\item porovnaj výslednú množinu s množinou vrcholov grafu.
\end{enumerate}

Nech má graf $n$ vrcholov, $m$ hrán a testovaná množina $s \leq n$ vrcholov. 
Potom v prvom kroku vykonáme $O(sm)$ operácií a v druhom $O(n^2)$ operácií. 
Test teda trvá $O(sm + n^2)$ operácií. Keďže počet vrcholov v testovanej 
množine môže byť rovnaký ako počet vrcholov grafu, tak odhad môžeme upraviť na 
$O(nm + n^2)$. Keďže pre siete malého sveta platí $O(m) = O(n)$, tak časový 
odhad pre siete malého sveta je $O(n^2)$.

\section{Skúšanie všetkých možností}

Prvým algoritmom, ktorý je v prehľade, je najzákladnejší algoritmu vyskúšania 
všetkých možností. Tento algoritmus budeme volať aj \emph{naivný}. Algoritmus 
vždy poskytne správny výsledok, ale výpočet bude trvať dlho. Je 
to však dobrý začiatok k ďalším algoritmom. Pracuje podľa krokov:

\begin{enumerate}
	\item \label{itm:bf:one} Vyber podmnožinu grafu;
	\item \label{itm:bf:two} otestuj, či je podmnožina dominujúcou množinou;
	\item \label{itm:bf:three} ak je podmnožina dominujúcou množinou a zároveň má najmenšiu 
kardinalitu, zapamätaj si ju;
	\item \label{itm:bf:four} opakuj, kým nevyberieš všetky možné podmnožiny práve raz;
	\item \label{itm:bf:five} jednou z minimálnych dominujúcich množín je zapamätaná množina a 
dominančné číslo grafu je jej kardinalita.
\end{enumerate}

Algoritmus je pomalý hlavne kvôli kroku \ref{itm:bf:four} -- všetkých možných 
podmnožín je $2^n$, takže výsledný algoritmus skúšania všetkých možností bude 
$\Omega (2^n)$. Súčasné počítače zvládnu úlohu v rozumnom čase vyrátať pre 
$n\le 40$.

Tam, kde je najväčšia slabina, je zväčša aj najväčší priestor na zlepšenie. 
Existujú mnohé zlepšenia, ktoré zrýchlia algoritmus nielen v priemernom 
(resp.~reálnom) prípade, ale aj zlepšia teoretický odhad.

\section{Heuristiky pre algoritmus skúšania všetkých možností}

V tejto sekcii si povieme niečo o možných heuristikách pre naivný algoritmus. 
\emph{Heuristika} v algoritme je nejaký prvok, zväčša zo skúsenosti z reálneho 
sveta, o ktorom predpokladáme, že nám pomôže zrýchliť výpočet. Aj keď obvykle 
nezlepšuje asymptotickú zložitosť, heuristiky sa snažia byť navrhnuté tak, aby 
vo väčšine prípad zrýchlili beh algoritmu.

Častým príkladom a aplikáciou je jedna z heuristík na hľadanie najkratšej 
cesty. Možná heuristika je, že prehľadávanie bude uprednostňovať cesty 
smerujúce k hľadanému bodu. Tu si všimnime, že pri heuristike potrebujeme 
poznať informáciu, ako je daná voľba dobrá. Pokiaľ nie je medzi hľadaným bodom 
a bodom, z ktorého hľadáme prekážka, algoritmus prehľadá oveľa menej hrán, 
ako pri bežnom prehľadávaní. Samozrejme, pokiaľ sme v bludisku a najkratšia 
cesta vedie "opačným" smerom, tak prehľadáme všetky hrany, kým sa dostaneme k 
cieľu.

Podobne je to aj pri hľadaní minimálnej dominujúcej množiny. Dobrým odhadom sa 
javí možnosť vybrať do potenciálnej množiny $S$ ten vrchol $v$, ktorý vie 
pokryť čo najviac vrcholov, teda sa javí byť čo najbližšie k cieľu. Čiže hľadáme 
$\argmax_v \left|N\left[S \cup {v}\right]\right|$ pre $v \in V$. Spojenie 
tejto heuristiky s vedomosťou, že siete malého sveta majú veľa klastrov ešte 
upevňuje predpoklad, že táto heuristika bude dávať na sieťach malého sveta 
rýchlejšie výsledky a "zlých" prípadov bude málo.

V pôvodnom algoritme, ktorý skúša všetky možnosti to znamená, že si pamätáme 
dočasný najlepší výsledok a neskúšame tie možnosti, ktoré obsahujú viac vrcholov 
ako dočasný najlepší výsledok. Samotné vynechanie tých možností, ktoré majú viac 
prvkov ako momentálny najlepší výsledok je veľké zrýchlenie, keďže pre súvislý 
graf s $N$ vrcholmi platí, že veľkosť minimálnej dominujúcej množiny je nanajvýš 
${N}\!/{2}$.

\section{Prevedenie na problém množinového pokrytia}

Ďalšou možnosťou, ako presne nájsť minimálnu dominujúcu množinu je previesť 
problém na problém množinového pokrytia, vyriešiť ten a výsledok opäť previesť.
Tento spôsob navrhol \citet{grandoni04} vo svojej dizertačnej práci. Dôvodom 
prevodu je fakt, že problém množinového pokrytia bol v minulosti oveľa 
skúmanejším problémom.
Keďže pri exponenciálnych algoritmoch celkom záleží aj na konštante pri 
exponente, odhad tohto algoritmu spresnil \citet{fomin05}. 

\citet{grandoni04} previedol problém minimálnej dominujúcej množiny na 
hľadanie minimálneho množinového pokrytia na grafe $G := (V, E)$ tak, že 
množiny predstavovali vrchol a jeho susedov. Univerzom je množina vrcholov 
grafu. Takže vytvoril usporiadanú dvojicu $({N[v] : v \in V }, V)$, čo je 
vstupný údaj pre hľadanie množinového pokrytia.

\subsection{Pomocné tvrdenia}

V algoritme využijeme následujúce tvrdenia, ktoré platia v každej dvojici 
$(\mathcal{S}, \mathcal{U})$ problému množinového pokrytia:
\begin{enumerate}
	\item pre každé dve navzájom odlišné množiny $S$ a $R$ také, že 
		$S, R \in \mathcal{S}, S \subseteq R$, platí, že existuje vrcholové 
		pokrytie, ktoré neobsahuje $S$;
	\item ak existuje prvok $u \in U$, ktorý patrí iba do jednej množiny 
		$S \in \mathcal{S}$, tak táto množina $S$ patrí do každého vrcholového 
		pokrytia.
\end{enumerate}

Zaujímavým pozorovaním je, že každá podmnožina s kardinalitou jeden, spĺňa 
práve jedno z tvrdení.

V prípade, že všetky podmnožiny $S \in \mathcal{S}$ sú dvojprvkové, problém 
sa dá redukovať na hľadanie maximálneho párenia. \emph{Párenie} v grafe $G$ 
je množina hrán $M$ taká, že hrany nemajú spoločný ani jeden vrchol. 
\emph{Maximálne párenie} je párenie s najväčšou mohutnosťou.

Z inštancie problému množinového pokrytia $(\mathcal{S}, \mathcal{U})$, kde 
$|S| = 2, S \in \mathcal{S}$, vieme spraviť graf $G^\prime = (V, E)$ tak, 
že množinou vrcholov bude množina univerza, čiže $V = \mathcal{U}$ a množinu 
hrán $E$ budú tvoriť podmnožiny $S \in \mathcal{U}$. Minimálnu dominujúcu 
množinu na grafe reprezentovanú dvojicou $(\mathcal{S}, \mathcal{U})$ vieme 
určiť pomocou minimálneho hranového pokrytia na grafe $G^\prime$. 
Minimálne hranové pokrytie zase vieme získať pomocou maximálneho párenia. 
Keďže hranám v grafe $G^\prime$ zodpovedá práve jedna podmnožina 
$S \in \mathcal{S}$ v dvojici $(\mathcal{S}, \mathcal{U})$, tak vieme určiť 
maximálne množinové pokrytie.

\subsection{Algoritmus}

Samotný algoritmus hľadania minimálneho množinového pokrytia je založený na 
princípe rozdeľuj a panuj. Pracujeme s inštanciou $(\mathcal{S}, \mathcal{U})$. 
Jeho triviálny prípad je, keď $|\mathcal{S}| = 0$. Pred rozdelením sa snaží 
odstrániť podmnožiny kardinality $1$.  Ak majú všetky podmnožiny mohutnosť 
práve dva, problém sa prevedie na problém maximálneho párenia. Ak má nejaká 
množina kardinalitu väčšiu ako 2 nastáva delenie. Spájanie výsledkov spočíva 
iba v porovnaní, ktorá z vetiev dala lepší výsledok. Výstupom algoritmu je 
množina podmnožín, ktorá tvorím minimálne množinové pokrytie. Algoritmus 
dostáva za vstup iba množinu podmnožín $\mathcal{S}$ a vyzerá následovne 
(v algoritme sú kvôli prehľadnosti podmnožiny nazývané množinami):

\begin{enumerate}
	\item ak je množina množín $\mathcal{S}$ prázdna, vráť prázdnu množinu;
	\item ak je nejaká množina $R$ podmnožinou inej množiny $S$, vráť výsledok 
		algoritmu pre $\mathcal{S} \setminus R$;
	\item ak existuje jedinečný prvok medzi množinami a ten je obsiahnutý (iba) 
		v množine $R$, vráť výsledok algoritmu pre $\mathcal{S} \setminus R$ 
		zjednotený s množinou $R$;
	\item ak majú všetky množiny mohutnosť dva, tak vráť výsledok z maximálneho 
		párenia;
	\item inak spusti dvakrát algoritmus s vynechaním ľubovoľnej množiny $R$; 
		raz ju vynechaj s množiny množín $\mathcal{S}$, vtedy sa do výsledku 
		nezaráta; druhý raz odstráň zo všetkých množín prvky množiny $R$ a 
		zarátaj množinu do výsledku; porovnaj, pre ktoré spustenie dal 
		algoritmus lepší výsledok a ten vráť.
\end{enumerate}

Ako vidno, krok 1 je triviálny prípad. Kroky 2, 3 a 4 sú popísané vyššie. 
V kroku 5 je slovo ľubovoľný. Toto správanie sa dá zameniť pomocou nejakej 
heuristiky. Prirodzene sa núka skúsiť vyberať množiny s najväčšou mohutnosťou 
najskôr, keďže tie redukujú ostatné množiny najviac.

\section{Pažravý algoritmus}\label{sec:greedy}

V predchádzajúcich častiach sme si popísali algoritmy, ktoré rátajú presné 
výsledky. Na veľkých sieťach, napríklad na zobrazeniach skutočných sietí, 
sú však nepoužiteľné. Preto si musíme vystačiť iba s približným výsledkom. 

Prvým algoritmom, ktorý uvedieme a bude výsledok určovať iba približne, bude 
jednoduchý pažravý algoritmus, od ktorého si postupne odvodíme iné a použijeme 
viacero postupov, na riešenie problému nájdenia minimálnej dominujúcej množiny. 
Pre pripomenutie -- \emph{pokrytie vrchola} je množina všetkých jeho susedov 
vrátane vrchola a \emph{pokrytie množiny} je prienik pokrytie vrcholov množiny. 

Algoritmus vyzerá následovne:
\begin{enumerate}
	\item na začiatku je výsledná množina prázdna;
	\item do výslednej množiny pridaj vrchol, ktorý pokryje čo najviac ešte 
		nepokrytých vrcholov;
	\item opakuj predošlý bod, až kým nebude výsledná množina pokrývať všetky 
		vrcholy;
	\item vráť výslednú množinu.
\end{enumerate}

Za zmienku stojí, že v bode 3 algoritmus musí určiť, či výsledná množina už 
pokrýva všetky vrcholy. Toto z tohto algoritmu a jeho variantov robí algoritmy, 
ktoré sú zložité najmenej kvadraticky od počtu vrcholov.

Ako aj pri algoritmoch s exaktnými výsledkami, aj tu môžeme použiť nejaké 
heuristiky na druhý krok. O tých si povieme neskôr.

\section{Distribuovaný algoritmus}\label{sec:dist}

V tejto časti si ukážeme prerobenie pažravého algoritmu na distribuovaný, ktorý 
navrhol \citet{chapS}. 
Využijeme pri tom jedno pozorovanie. V bode 2 sa vyberie vrchol, ktorý pokrýva 
čo najviac ešte nepokrytých vrcholov. Počet ešte nepokrytých vrcholov môže 
ovplyvniť iba výber vrcholov zo vzdialenosti najviac 2. Preto, ak vrchol môže 
pokryť najviac vrcholov s pomedzi vrcholov vzdialených najviac 2, tak sa tento 
vrchol môže vybrať do výslednej množiny pred ostatnými. 

Toto pozorovanie vedie ku konštrukcii veľmi jednoduchého algoritmu (v každom 
vrchole):

\begin{enumerate}
	\item pre svoj vrchol vypočítaj počet ešte nepokrytých vrcholov;
	\item tento počet pošli algoritmom vo vrcholoch najviac vzdialených dve 
		hrany;
	\item ak vrchol pokrýva najväčší počet vrcholov vo vzdialenosti najviac 
		dva, tak vrchol pridaj do dominujúcej množiny (ak je takých vrcholov 
		viac, rozhodni náhodne -- napríklad podľa ID);
	\item opakuj od bodu 1, až kým vrchol nebude mať všetkých susedov 
		pokrytých.
\end{enumerate}

Tento algoritmus teoreticky funguje veľmi dobre. Počet vykonaných krokov bude 
lineárne úmerný počtu vrcholov grafu, keďže v každom kroku sa aspoň jeden 
vrchol vyberie. V skutočnosti má okrem veľa implementačných problémov uvedených 
v kapitole \ref{chap:vysledky} aj zlý počet krokov výpočtu pre niektoré typy 
grafov.

\begin{figure}
	\centering
	\includegraphics[width=1.0\textwidth]{obrazky/zle1}
	\caption{\emph{Graf, na ktorom beží distribuovaný algoritmus pomaly.} 
		Algoritmus v takejto situácií vyberá vrcholy postupne zľava doprava 
		a v každom kroku môže vybrať iba jeden vrchol.}
	\label{img:zle1}
\end{figure}

\begin{figure}
	\centering
	\includegraphics[width=0.5\textwidth]{obrazky/zle2}
	\caption{\emph{Graf, na ktorom beží distribuovaný algoritmus pomaly.} 
		Algoritmus v takejto situácií vyberá vrcholy náhodne, ale v každom 
		kroku iba jeden, pretože všetci potenciálny kandidáti sa blokujú.}
	\label{img:zle2}
\end{figure}

Na obrázku \ref{img:zle1} je vidieť jeden z možných grafov, na ktorom beží 
algoritmus pomaly. Algoritmus síce vyberie minimálnu dominujúcu množinu presne. 
Tá pozostáva z vrcholov na jej "osi". Ale v jednom opakovaní cyklu algoritmus 
vyberie najviac jeden vrchol, postupne zľava doprava. Je vidno, že algoritmus 
je $\Theta (N)$, kde $N$ je počet vrcholov grafu. Tento problém môžeme vyriešiť 
tak, že povolíme vybrať aj vrcholy s menším pokrytím vrchola. Tým síce 
zhoršíme aproximačný faktor, ale nie o veľa. Konkrétne, ak zhoršíme aproximačný 
faktor dvojnásobne, môžeme zaokrúhliť pokrytie vrchola na najbližšiu mocninu 
dvojky. Pri vyberaní najlepšieho kandidáta môžeme potom vybrať aj vrchol s 
menším pokrytím.

Druhou zlou možnosťou pre algoritmus je, keď je v grafe veľký klaster s 
vrcholmi rovnakého stupňa. Možnosť je znázornená na obrázku \ref{img:zle2}. 
Algoritmus nemôže naraz vybrať všetky vrcholy stupňa 3. Bráni mu v tom 
podmienka, že sa dá vybrať iba jeden z vrcholov spomedzi okolia vzdialenosti 
dva. Tu pomôže, ak povolíme vybrať viac vrcholov naraz, pokiaľ si príliš 
neprekážajú.

Algoritmus s týmito dvoma uvedenými vylepšeniami je zložitejší. Horným 
ohraničením pokrytia budeme nazývať najbližšiu hornú mocninu dvojky. Podobne, 
dolným ohraničením pokrytia budeme označovať najbližšiu dolnú mocninu dvojky.


\begin{enumerate}
	\item pre svoj vrchol vypočítaj počet ešte nepokrytých vrcholov;
	\item vypočítaj dolné ohraničenie pokrytia;
	\item zober dolné ohraničenia susedov vzdialených najviac 2;
	\item ak je dolné ohraničenie také, ako najväčšie dolné ohraničenie 
		spomedzi susedov, tak sa vrchol má šancu stať vybratým;
	\item táto šanca je nepriamo úmerná počtu vrcholov s najväčším dolným 
		pokrytím;
	\item vyber vrchol s pravdepodobnosťou vypočítanou vyššie;
	\item vypočítaj hodnotu $c$ -- počet kandidátov na vybratie spomedzi 
		pokrývajúcich vrcholov;
	\item ak je vrchol vybratý a súčet hodnôt $c$ pre pokrývajúce vrcholy 
		je menší alebo rovný ako trojnásobok počtu pokrývajúcich vrcholov, 
		tak vrchol pridaj do medzivýsledku;
	\item opakuj od bodu 1, až kým vrchol nebude mať všetkých susedov 
	pokrytých;
\end{enumerate}

Takto upravený algoritmus zvláda vypočítať dominujúce množiny o čosi 
rýchlejšie. Keďže množstvo klastrov a počet vrcholov s rovnakým stupňom je v 
sieťach malého sveta veľký, tak je predpoklad, že tento upravený algoritmus 
bude bežať oveľa rýchlejšie na sieťach malého sveta ako neupravený.

Vďalšej časti sa už nezaoberáme distribuovanými algoritmami ale jednoduchým 
pažravým algoritmom s rôznymi heuristikami.


\section{Heuristiky pre pažravý algoritmus}

V predchádzajúcich častiach sme zhrnuli rôzne postupy, ako riešiť problém 
minimálnej dominujúcej množiny. V tejto časti uvedieme heuristiky pre pažravý 
algoritmus spomenutý vyššie. Prvou heuristikou bude výber vrchola, ktorý 
pokrýva vrchol stupňa najviac jeden. Túto heuristiku rozvinieme a pridáme k 
nej ďalšie. Potom ich skombinujeme do rôznych algoritmov.

\subsection{Odstraňovanie výhonkov}

Prvá heuristikou, ktorú uvedieme je jednoduchá. V pažravom algoritme, pred tým, 
ako začneme vyberať vrcholy (krok 2), tak vyberieme všetky vrcholy stupňa nula 
a vrcholy, ktoré susedia s vrcholmi stupňa jeden (to znamená ich jediných 
susedov). Tieto vrcholy budeme nazývať \emph{výhonky}.


\subsection{Odstraňovanie kvetín}

Druhá heuristika spočíva v rozšírení prvej -- odstraňovaní výhonkov. Vo 
všeobecnosti sa dá povedať, že sme sa snažili vybrať vrchol s čo najväčším 
počtom nepokrytých susedov takých, ktoré majú všetkých susedov spoločných s 
vybratým vrcholom. Teda tvoria akýsi lokálny klaster "na okraji" grafu. 

Vrchol $v$, ktorý pokrýva $n$ vrcholov, ktoré majú za susedov iba susedov 
vrchola $v$, nazývame \emph{kvetinou} rádu $n$.

\subsection{Rozhodovanie rovnakých vrcholov}

Keď máme viacero rovnakých kandidátov (z pohľadu) na výber, je otázne, či si 
vybrať náhodný vrchol, vrchol s najmenším/najväčším ID alebo pridať nejaké 
pravidlo, ktoré rozhodne o priorite výberu. Tu si uvedieme nejaké pravidlá.

V základnom pažravom algoritme na hľadanie sietí malého sveta je často 
situácia, kde majú rovnaký počet ešte nepokrytých vrcholov vrcholy s úplne 
iným stupňom. Možno ešte podstatnejšia hodnota ako počet ešte nepokrytých 
susedov je počet vrcholov, ktoré po vybratí daného vrchola prestanú pokrývať 
akékoľvek vrcholy. Toto nie je to isté ako hľadanie kvetiny s najväčším rádom, 
lebo kvetiny sa vyskytujú iba na okrajoch grafu. Vrcholy vybraté takýmto 
rozhodovaním sa nachádzajú všade a preferujú susedov vzdialených 3 od už 
vybratých susedov (respektíve vzdialených 2 od pokrytých vrcholov). 

Treba si ale uvedomiť, že priemer siete malého sveta je malý a tak skutočná 
preferencia na "umiestnenie v grafe" nie je. Taktiež tento spôsob vyberania 
môže trochu korigovať "slepé" vyberanie vrcholov s väčšími stupňami.

\subsection{Výsledné algoritmy}

Vyššie uvedené heuristiky ide rôzne kombinovať a rôzne použiť. Na tomto mieste 
uvedieme algoritmus, ktorý skombinuje všetky pravidlá. Pre odstraňovanie 
výhonkov a odstraňovanie kvetín platí, že sa vykonajú iba raz -- na začiatku 
algoritmu. Po odstránení sa spustí pažravý algoritmus, ktorý vhodného kandidáta 
vyberie na základe pokrytia vrcholu a odstraňovania najvplyvnejších vrcholov.

Algoritmus vyzerá následovne:

\begin{enumerate}
	\item odstráň výhonky:
	\begin{enumerate}
		\item vyber vrcholy do výsledku;
		\item pokry susedov;
	\end{enumerate}
	\item postupne odstraňuj kvety (postupne preto, lebo z niektorý kvetov 
		prestanú byť kvety) -- kvety zotrieď podľa rádu a stupňa; potom 
		postupne, až kým nie je množina kvetov prázdna:
		\begin{enumerate}
			\item vyber kvet do výsledku;
			\item pokry susedov;
			\item odstráň tie kvety, ktoré už nie sú kvetmi;
		\end{enumerate}
	\item prepočítaj hodnoty pre "vplyvnosť" vrcholov;
	\item opakuj, až kým množina pokrytých vrcholov netvorí celý graf:
		\begin{enumerate}
			\item do výsledku vyber najlepšieho kandidáta podľa "vplyvnosti" 
				a stupňa;
			\item pokry susedov vybratého vrchola;
			\item prerátaj "vplyvnosť" susedov;
		\end{enumerate}
\end{enumerate}

Ide o upravený pažravý algoritmus. Upravený o heuristiky a pozorovania uvedené 
vyššie. Tento algoritmus sa dá rôzne modifikovať upravovaním heuristík a 
pravidiel. 

Ide o posledný uvedený algoritmus. V ďalších kapitolách sa venujeme 
implementačnej časti, rôznym testom a výsledkom.
\chapter{Popis softvéru}\label{chap:popis}

V predchádzajúcich kapitolách sme spravili prehľad problematiky a algoritmov, 
ktoré slúžia na hľadanie minimálnej dominujúcej množiny. V tejto kapitole sa 
venujeme popisu a návrhu softvéru, ktorý slúži na porovnanie týchto algoritmov 
a otestovanie týchto algoritmov v praxi.


\section{Analýza existujúcich riešení}

Keďže problém nájdenia minimálnej dominujúcej množiny je podobný problému 
vrcholového pokrytia je často súčasťou aplikácií venujúcim sa zhromažďovaniu 
rôznych grafových algoritmov. Veľa nám známych z nich obsahuje iba základy 
výpočet všetkých možností, alebo jednoduchý aproximačný algoritmus, preto 
tu uvedieme iba jedného zástupcu tohto druhu. Popri vyvíjaní nášho 
softvéru sme pracovali aj s nástrojmi pre analýzu grafov. Tieto sa taktiež 
zväčša vyskytujú v balíkoch s inými nástrojmi pracujúcimi s grafmi (napr. 
vizualizácia). Uvedieme tiež iba jedného zástupcu.

% \todo{
% Cieľom tejto etapy je oboznámenie sa s prostredím, v ktorom bude aplikácia používaná. 
% Dôraz sa kladie na hlbšie pochopenie interakcie aplikácie s prostredím. 
% Ak existujú, treba vykonať aj analýzu podobných systémov.
% }

\subsection{Network Benchmark}

Reprezentantom softvérov na analýzu grafov je Network Benchmark. Ide o softvér, 
ktorý získal podporu aj Alberta Barabásiho. Bol vyvíjaný v rokoch 2005 až 2011. 
Obsahuje príklady, dokumentáciu a podklady k práci s analýzou grafu. Je 
naprogramovaný v jazyku \Java\ a k jeho silným stránkam patrí relatívna 
prehľadnosť pri veľkom množstve nastaveniach a možnostiach analýzy. 

Je dostupný na adrese: \url{http://nwb.cns.iu.edu/index.html}

\begin{figure}
	\centering
	\greybox{%
		\includegraphics[scale=0.3]{obrazky/nwb1}%
	}
	\caption{\emph{Softvér Network Benchmark.} V ľavej hornej časti 
		obrazovky sa nachádzajú výpisy. V dolnej prebiehajúce a ukončené 
		akcie a výpočty. Vpravo sú zhromaždené výsledky výpočtov.}
	\label{img:vis:nwb1}
\end{figure}

\begin{figure}
	\centering
	\greybox{%
		\includegraphics[scale=0.3]{obrazky/nwb2}%
	}
	\caption{\emph{Softvér Network Benchmark.} Tu je zobrazená vizualizácia 
		siete malého sveta. Bublinky znázorňujú jednotlivé klastre.}
	\label{img:vis:nwb2}
\end{figure}

Na obrázku \ref{img:vis:nwb1} je vidno hlavnú obrazovku aplikácie. Práca s 
aplikáciou prebieha tradične tak, že používateľ načíta nejakú grafovú štruktúru, 
následne sa mu zobrazí v stĺpci napravo, označí ju a z bohatej ponuky zvolí 
nejakú analýzu. Vľavo dole, potom vidí priebeh jeho akcií a vľavo hore bližšie 
popisy toho, čo sa práve vykonáva.


\subsection{NetworkX}

Tento softvér je reprezentantom zbierky rôznych algoritmov na grafoch. Je 
vyvíjaný od roku 2005 až doteraz. Posledná stabilná verzia v čase písania 
tejto práce je zo septembra roku 2014. Ide o rozšírenie jazyka Python o novú 
knižnicu.

Sídlo projektu je na adrese: \url{https://networkx.github.io/}

\begin{figure}
	\centering
	\greybox{%
		\includegraphics[scale=0.6]{obrazky/nx}%
	}
	\caption{\emph{Softvér NetworkX.} Tu je zobrazená vizualizácia 
		náhodného grafu s rôznymi pravdepodobnosťami hrán.}
	\label{img:vis:nx}
\end{figure}

Na obrázku \ref{img:vis:nx} je znázornená vizualizácia so softvérom NetworkX. 
Ide o tvorbu náhodného grafu pomocou určenia pravdepodobnosti vzniku hrany. 
Vidno, že aj keď je NetworkX hlavne zbierkou algoritmov, poskytuje peknú 
vizualizáciu. Jeho nedostatkom pre nás je, že na riešenie problému hľadania 
minimálnej dominujúcej množiny poskytuje iba pažravý algoritmus.

\subsection{Gephi}

Gephi je platforma na vizualizáciu grafov. Zameriava sa na vizualizovanie 
grafov a počítanie rôznych hodnôt pre analýzu komplexných sietí. Podporuje veľa 
grafových formátov. Je vo vývoji od roku 2009 \citep{gephi}. Zatiaľ posledná 
verzia vyšla v roku 2013. Platforma je dostupná na internetovej stránke 
\url{http://gephi.github.io/}. Na obrázku \ref{img:vis:gephi} je príklad 
vizualizácie grafu v programe Gephi.

Aj keď platforma je veľmi silným nástrojom na analýzu komplexných sietí, 
neposkytuje žiaden algoritmus na nájdenie minimálnej dominujúcej množiny.

\begin{landscape}
\begin{figure}
	\centering
	\greybox{%
		\includegraphics[scale=0.61]{obrazky/gephi}%
	}
	\caption{\emph{Platforma Gephi.} Vľavo je vizualizácia grafu. Vpravo sú 
		uvedené rôzne vlastnosti grafu.}
	\label{img:vis:gephi}
\end{figure}
\end{landscape}

\subsection{JUNG}

Projekt JUNG (Java Universal Network/Graph Framework) je grafová knižnica 
naprogramovaná v jazyku \Java. Ponúka vlastný jazyk, ktorý poskytuje 
modelovanie, analýzu a vizualizáciu grafov a sietí.

\section{Špecifikácia požiadaviek}

Ako vidno, súčasné softvéry majú dobrú ponuku rôznych algoritmov a veľa 
odlišných vizualizácií. Avšak daň za to, že ponúkajú riešenie na veľa problémov 
je tá, že zväčša je dostupný iba jeden algoritmus na riešenie.

Keďže sa my v práci zaoberáme porovnávaním algoritmov, veľmi nám táto 
skutočnosť nesedí. Preto by sme mali navrhnúť vlastný softvér, ktorý zrejme 
nebude poskytovať riešenia na veľa problémov ale veľa riešení na jeden problém.

%\todo{Čo očakávame od softvéru a prečo je viac fajn ako konkurencia.}

\subsection{Požiadavky}

Naša aplikácia slúži najmä na porovnávanie relatívnej rýchlosti behu algoritmov 
a bude používaná zrejme iba úzkym okruhom ľudí.

Preto sa nám zdalo byť vhodné aplikáciu nechať iba v príkazovom riadku. Znížia 
sa tým odchýlky spôsobené grafickým prostredím. Softvér by mal vedieť spracovať 
graf zadaný v špecifikovanom tvare. Taktiež by mal správne 
implementovať všetky algoritmy popísané v kapitole \ref{chap:algoritmy} a 
poskytnúť informácie o behu daného algoritmu.

Implementované algoritmy by mali zahŕňať:

\begin{itemize}
	\item exaktný algoritmus skúšaním všetkých možností,
	\item exaktný algoritmus skúšaním všetkých možností s heuristikou,
	\item exaktný algoritmus riešený pomocou prevedenia na problém množinového 
		pokrytia,
	\item distribuovaný algoritmus,
	\item upravený distribuovaný algoritmus,
	\item jednoduchý pažravý algoritmus,
	\item pažravý algoritmus s rôznymi heuristikami;
\end{itemize}

\subsection{Prevádzkové požiadavky}

Keďže ide o softvér, ktorý medzi sebou porovnáva jednotlivé algoritmy 
relatívne, tak nepotrebuje špecifický operačný systém. Výkon algoritmov je 
priamo úmerný od hardvéru. To al neznamená, že by hardvér mal byť nejako 
špecifický. Keďže sa súčasné architektúry z nášho pohľadu veľmi nelíšia, 
nekladieme po tejto stránke žiadne požiadavky.

% \todo{
% Východiskom tejto etapy je etapa analýzy. 
% Zaoberá sa úlohami, ktoré má aplikácia zabezpečiť. 
% Oboznámime sa s prostredím, v ktorom bude aplikácia používaná. 
% Zaujímame sa o to, čo chce zadávateľ riešiť, čo požaduje, aby aplikácia vykonávala a pre koho je určená. 
% Zatiaľ nás nezaujíma spôsob realizácie.  
% Cieľom špecifikácie požiadaviek je stanovenie služieb, ktoré zákazník požaduje od systému a ohraničenia na jeho vývoj a prevádzku. 
% Teda zaujímame sa o funkcionálnu stránku systému, potom o to, aké by mali byť vstupy a výstupy systému 
% a s akými údajmi systém bude pracovať, potom o prevádzkové požiadavky 
% (počet a charakteristika používateľov, časová odozva systému, potrebný hardvér a softvér, bezpečnosť a ochrana systému a iné potrebné požiadavky). 
% Prevádzkové požiadavky nazývame aj nefunkcionálne požiadavky. 
% Detailný opis špecifikácie požiadaviek na softvér  najdeme v (1) alebo v štandarde IEEE 830.
% }

\section{Návrh}

Našou hlavnou úlohou je naprogramovať program, ktorý vie spracovať nejaký 
formát vstupu, spraviť z neho graf, vykonať na ňom zvolený algoritmus. Takže 
bude obsahovať moduly na spracovanie týchto požiadaviek. V následujúcich 
častiach popisujeme požiadavky kladené na náš softvér.

% vzťahy medzi časťami softvéru, 
%štruktúru dát a rozhranie systému.

\subsection{Technické požiadavky}

Softvér nevyžaduje žiaden špecifické technické veci. Vstupné a výstupné údaje 
môžu prúdiť cez štandardný systémový vstup a výstup. V prípade lepšej práce 
môžu byť použité knižnice na prácu s nejakými dátovými štruktúrami.

\subsection{Používateľské požiadavky}

Používateľmi tejto aplikácie sú najmä ľudia, ktorí chcú porovnať jednotlivé 
algoritmy. Teda stačí, aby bolo poskytnuté dobré vstupno-výstupné prostredie.

Ak chceme porovnávať medzi sebou rôzne algoritmy na nejakom grafe, alebo ak 
chceme porovnávať jeden algoritmus na rôznych grafoch je dobré mať aj 
prostredie pre dávkové úlohy.

\subsection{Použité technológie}

Keďže algoritmy porovnávame relatívne na sebe, tak môžeme zvoliť ľubovoľný 
programovací jazyk za hlavný. Keďže chceme, aby bola aplikácia ľahko 
spustiteľná na každom operačnom systéme, vhodným kandidátom je programovací 
jazyk \Java .

\Java\ poskytuje aj implementáciu základných dátových štruktúr. Avšak túto možno 
zameniť s nejakými inými externými knižnicami. Na implementáciu rôznych 
algoritmov na tom istom základe je vhodné použiť návrhový vzor \emph{strategy}.

\subsection{Rozhranie aplikácie}

Ako sme už spomenuli, rozhranie aplikácie tvorí príkazový štandardný vstup a 
výstup. Na vstupe pre aplikáciu sú dva argumenty. Algoritmus použitý pri behu a 
cesta k súboru s grafom. Súbor s grafom je textový súbor obsahujúci riadky, 
v ktorých sú dvojice čísel, označujúcich neorientované hrany medzi vrcholmi. 
Na vytvorenie vnútornej reprezentácie grafu teda treba meno súboru, ktorý graf 
obsahuje. výstupom je objekt, s ktorým vie algoritmus pracovať. Algoritmus 
dostane na vstup tento objekt, vypočíta potrebné pomocné dátové štruktúry, 
spustí výpočet a vyráta výsledok. Ten potom vypíše na štandardný výstup. Čo je 
zároveň aj výstupom aplikácie.

Pre algoritmus výpočtu pomocou prevedenia sú potrebné pomocné dátové štruktúry, 
ale v zásade ide o jednoduchý princíp posielania si rôznych reprezentácií grafu 
objektami.

Pre dávkový proces je vstupom textový súbor, ktorý obsahuje riadky. V každom 
riadku je argument pre jeden beh aplikácie. Aplikácia sa spúšťajú za sebou, 
následujúca až vtedy, keď sa predošlá dokončí.


% \todo{
% V tejto etape sa ujasňuje koncepcia systému. 
% Navrhne sa jeho dekompozícia, určia sa vzťahy medzi časťami systému a ohraničenia funkcionality. 
% Túto časť návrhu zvyčajne modelujeme technikami softvérového inžinierstva napr. procesným modelom DFD (Data Flow Diagram, str.24 v (1)) 
% Dekompozícia sa môže urobiť aj na základe iných princípov ako je funkcionálny (str. 53 v (1)). 
% V ďalšom sa identifikuje štruktúra údajov, ktoré do systému vstupujú, ktoré systém spracováva a produkuje. 
% Vzťah medzi údajmi sa modeluje entitno-relačným diagramom, v ktorom sa pomenovávajú vzťahy medzi údajovými entitami. 
% Ak použijeme model DMD (Data Model Diagram), potom v ňom znázorňujeme kvantifikované vzťahy dohodnutými značkami na označenie kardinality. 
% Ak rozložíme vzťah typu M:N pomocou väzobnej entity, potom dostávame fyzický model údajov.  (str.31 v (1))
% V etape návrhu sa navrhne rozhranie systému ( čo do systému vstupuje a čo z neho vystupuje), 
% navrhne sa typ používateľského rozhrania (príkazovo orientované rozhranie, menu, priame riadenie, komunikácia v prirodzenom jazyku ), 
% plán realizácie systému a stanovia sa podmienky, za akých bude používateľ akceptovať produkt. 
% Odhadnú sa potrebné ľudské a materiálne zdroje a navrhne sa postup zaškolovania používateľov.
% }


\def\alg#1{\texttt{#1}}

\chapter{Implementácia softvéru}\label{chap:implementacia}

V tejto kapitole postupne uvedieme realizáciu návrhu popísaného 
v~kapitole~\ref{chap:popis}. Popíšeme ako sme implementovali jednotlivé 
algoritmy a ukážeme si štruktúru a ovládanie aplikácie. 

% \todo{
% Programovo realizujeme podrobný návrh databázovej aplikácie. 
% Vypracuje sa podrobná dokumentácia k vytvorenému produktu. 
% Vytvorí sa používateľská príručka, kde sa podrobne dokumentuje používateľské rozhranie, spresní sa popis funkcií a spôsob ich aktivácie.
% }

\section{Algoritmy}

Aj keď sme porovnávali iba 10 algoritmov, dokopy bolo implementovaných 
algoritmov viac. V následujúcich častiach postupne uvedieme jednotlivé 
algoritmy a ich označenia pre aplikáciu. Algoritmy boli implementované podľa 
popisu v~kapitole~\ref{chap:algoritmy}.

Pred tým, než rozpíšeme implementáciu algoritmov uvedieme štruktúru a 
implementáciu grafu. Graf je implementovaný ako objekt, ktorý obsahuje 
zoznam hrán. Pokiaľ je to potrebné (čo zväčša je), vie sa iniciovať a zavolať 
vypočítanie susedov do vzdialenosti 2 pre každý vrchol (aby sa nemuselo počítať 
pri každom použití ale iba raz, na začiatku). Taktiež objekt reprezentácie 
grafu vie skontrolovať, či je nejaká množina dominujúcou. To je spravené 
pomocou delegácie na algoritmus kontroly dominujúcej množiny.

Nasleduje popis jednotlivých skupín algoritmov s ich označeniami.

\subsection{Algoritmy skúšajúce všetky možnosti}

Prvým a základným algoritmom je algoritmus, ktorý skúša všetky možnosti. Jeho 
označenie v softvéri je \alg{naive}. Je implementovaný rekurzívnym 
prehľadávaním. Obdobou sú algoritmy označené \alg{mynaive} a \alg{mynaive2}, 
ktoré obsahuje základné heuristiky. Prvou je, že algoritmus najprv zoradí pole 
vrcholov podľa stupňa a až potom začne vyberať do výsledku. Druhou je, že ak 
je medzivýsledok menšej mohutnosti, tak neskúša vyberať množiny s väčšou 
mohutnosťou.

\subsection{Algoritmy prevedenia problému}

Ďalšími algoritmami, ktoré sme implementovali, boli algoritmy, ktoré prevádzali 
hľadanie minimálnej dominujúcej množiny na problém množinového pokrytia. Keďže 
algoritmus bol reprezentovaný návrhovým vzorom \emph{strategy}, tak 
sa implementácia zjednodušila a sprehľadnila. Časť algoritmu sa venovala 
spracovaniu vstupu a časť výpočtu a implementovaniu náročnejších algoritmov. 

Aj keď \citet{fomin05} uvádzajú algoritmus jeden, my sme implementovali dve rôzne 
verzie: \alg{fnaive} a \alg{fproper}. Verzia \alg{fnaive} neobsahuje finálne 
prevedenie na hľadanie minimálneho hranového pokrytia v grafe. Verzia 
\alg{fproper} toto prevedenie obsahuje a implementuje ho cez hľadanie 
maximálneho párenia v grafe. Keďže v čase použitia algoritmu je graf 
bipartitný, tak sa to dá spraviť.

\subsection{Distribuované algoritmy}

Podobne ako predošlé algoritmy, aj tieto algoritmy mali viac verzií. Konkrétne 
jednovláknovú a viacvláknovú. Distribuovanosť bola implementovaná cez veľa 
na sebe nezávislých vlákien. Nedistribuované verzie sme označili príponou 
\alg{-OT} (one thread).

Algoritmy sme implementovali tak ako v prehľade -- dva. Prvý bez riešenia 
problémových grafov (obrázky \ref{img:zle1} a \ref{img:zle2}), ale prehľadnejší 
a zrozumiteľnejší. Druhý s riešením týchto problémov. Prvý je označený ako 
\alg{ch7alg34} (resp.~\alg{ch7alg34OT} pre jednovláknovú verziu) a druhý je 
označený ako \alg{ch7alg35} (podobne \alg{ch7alg35OT} pre jednovláknovú verziu).

\subsection{Pažravé algoritmy}

Keďže jedným z hlavných cieľov práce bolo spraviť dobrú heuristiku na pažravý 
algoritmus, tak boli tieto algoritmy implmentované vo viacerých verziách. Okrem 
rôznych heuristík ide aj o spôsob vyberania vrcholov. Počas behu algoritmu sú 
niektoré vrcholy vybraté do medzivýsledku, niektoré pokryté a niektoré 
nevybraté a nepokryté. Tie vrcholy, ktoré sú nevybraté a nepokryté sú 
označované aj ako \emph{biele}, tie ktoré nie sú vybraté, ale sú pokryté sú 
označované ako \emph{šedé} a tie, ktoré sú vybraté sú označované ako 
\emph{čierne}. 

Podľa algoritmu, ktorý je uvedený v časti \ref{sec:greedy} sme implementovali 
algoritmus, ktorý sme označili \alg{greedy}. Podobne sme implementovali aj 
verziu \alg{ch7alg33}. Avšak tá nevyberá vrcholy, ktoré pokrývajú čo najviac 
iných spomedzi šedých a bielych, ale iba z bielych vrcholov. Veľmi podobne je 
implementovaná aj verzia \alg{greedyq}. Jediný rozdiel je, že vrcholy na 
začiatku behu usporiada podľa stupňa od najväčšieho po najmenší.

\subsubsection{Heuristiky na rozhodovanie}

Po troch podobných verziách sme implementovali rôzne heuristiky. Prvou bolo 
odstraňovanie výhonkov. Toto odstraňovanie prebieha na začiatku 
algoritmu, postupne v iteráciách, až kým nie je čo odstrániť. Motiváciou za 
týmto algoritmom je efektívne odstraňovanie ciest. Verzia algoritmu s touto 
heuristikou je označovaná ako \alg{greedysr}. 

V ďalšej verzii sme použili opäť základný pažravý algoritmus, ale ako 
rozhodovací faktor pre výber vrchola sme vybrali počet vrcholov, ktoré po 
vybratí už nebudú mať bieleho suseda (toto nie je to isté ako rád kveta) 
a až potom počet vrcholov, ktorým vrchol dominuje. Motiváciou bolo 
presvedčenie, že často je lepšie vybrať vrchol, ktorý po svojom vybratí čo 
najviac zmenší počet potenciálne vybratých vrcholov v ďalších iteráciách, 
namiesto vyberania vrcholov s najväčším pokrytím.
Heuristika by mala byť nejakým 
kompromisom medzi vyberaním vrcholov "zospodu" a "zvrchu". Táto verzia 
algoritmu je označená ako \alg{greedysw}.

\subsubsection{Zložitejšia heuristika}

Treťou verziou je \alg{floweru}, ktorá je podobná ako \alg{greedysr}, s tým 
rozdielom, že na začiatku spraví iba jeden beh odstraňovania výhonkov. Toto je 
základom pre posledný, zložitejší algoritmus, ktorý je označený \alg{flower}. 

Verzia \alg{flower} na začiatku behu zoradí vrcholy podľa stupňa. Následne sa 
snaží označiť výhonky a kvety. Vyčistí výhonky a upraví graf tak, aby sa mohli 
vybrať kvety. Kvety sa vyberajú postupne. Najviac záleží na ráde kvetu. Ak majú 
nejaké kvety rovnaký rád, tak sa rozhoduje podľa toho, koľko vrcholov po 
vybratí už nebude mať bieleho suseda (podobne ako pri verzii \alg{greedysw}). 
Ak je aj táto hodnota rovnaká, tak zaváži počet bielych susedov (najväčšie 
pokrytie). Ak nepomôže ani toto pravidlo, tak sa vyberie náhodný vrchol.

Po označení výhonkov a kvetov sa začnú označovať zvyšné vrcholy podľa 
jednoduchého pažravého algoritmu. Avšak ako rozhodujúci faktor pri vyberaní 
slúži najprv počet bielych susedov (pokrytie) a potom počet vrcholov, ktoré 
po vybratí už nebudú mať bieleho suseda.

\subsection{Dátové štruktúry}

Aj keď na porovnanie rôznych algoritmov sú základné dátové štruktúry 
poskytované ako súčasť knižníc jazyka \Java, rozhodli sme sa použiť na 
implementáciu základných dátových štruktúr knižnicu HPPC od poskytovateľa 
Carrot Search Labs vo verzii 0.6.0. Je dostupná na webovej adrese 
\url{http://labs.carrotsearch.com/hppc.html} a licencovaná Apache License 2.0. 
hlavnou výhodou knižnice oproti štandardnej knižnici je menšia spotreba pamäte 
a priamejší prístup k vnútornej reprezentácií (aspoň v danej verzii). To 
spôsobilo celkové zrýchlenie algoritmov a posunulo hranice testovateľných 
grafov.



\chapter{Dosiahnuté výsledky}\label{chap:vysledky}

V predošlých kapitolách sme spravili prehľad existujúcich algoritmov, popísali 
sme ich implementáciu. V tejto kapitole uvádzame porovnanie jednotlivých 
implementácií. Najprv uvedieme jednotlivé obmedzenia implementácie, potom 
popíšeme spôsob testovania a napokon označenie pre jednotlivé testovacie dáta.

\section{Obmedzenia implementácie}

Vývoj softvéru sa začal na staršom počítači, ktorý má procesor Intel Core 2 
Duo P7350 (2,0GHz dvojjadrový) a 4GB RAM pamäte. Pamäť nebola obmedzením. 
Bohužiaľ počítač 
nebol vhodný na viacvláknové algoritmy, čo spôsobovalo problémy pre algoritmy 
\alg{ch7alg34} a \alg{ch7alg35}. Preto sme sa zamerali na ich jednovláknové 
verzie (\alg{ch7alg34OT}, respektíve \alg{ch7alg35OT}). Heuristiky na pažravý 
algoritmus sa doimplementovali neskôr, preto sa neuvádzajú v starších testoch.

Počas vývoja začal byť dostupný novší počítač, na ktorom sa testovali už všetky 
heuristiky pre pažravý algoritmus. Počítač má procesor Intel Core i5-4690K CPU 
(3,50GHz štvorjadrový) a 16GB pamäte. Tento počítač zvláda viacvláknové 
aplikácie lepšie (o 6 rokov modernejšia architektúra), no 
vzhľadom na zachovanie kompaktnosti sme opäť viacvláknové verzie algoritmov z 
testov vynechali.

Neskoršia práca s externými knižnicami HPPC časy zrýchlila, čo bohužiaľ nie je 
zachované v žiadnej z uvedených tabuliek.

\section{\Java\ verzus \cpp}

Zo začiatku implementácie nebolo jasné, ktorý programovací jazyk bude použitý. 
Ako prvý sa zvolil jazyk \Java. Keďže prevláda povedomie, že interpretovaný 
programovací jazyk (\Java) nemôže byť rýchlejší ako kopmilovaný programovací 
jazyk (\cpp), tak sme sa v priebehu implementácie rozhodli, že skúsime 
naprogramovať niektoré algoritmy aj v jazyku \cpp. Ako vidno z porovnania časov 
behov jednotlivých algoritmov v tabuľkách \ref{table:cpp} a 
\ref{table:java-stare}, jazyk \cpp\ neposkytuje žiadne 
výrazné zrýchlenie. Spolu s faktom, že na porovnanie jednotlivých algoritmov až 
tak rýchlosť netreba, sme sa rozhodli, že budeme pokračovať vo vývoji v jazyku 
\Java.

\section{Spôsob testovania}

Keďže programovací jazyk \Java\ robí optimalizácie kódu počas behu a triedy 
inicializuje "lenivo", tak je potrebné algoritmus spustiť viackrát, aby ukázal 
relevantné výsledky. Algoritmus teda necháme spustiť trikrát a zoberieme tretí 
výsledok. Rozhodnutie je vecou pozorovania. Je kompromisom medzi relevantnými 
výsledkami a dĺžkou skúšania rôznych behov. Z časov nad 20 sekúnd je v 
tabuľkách uvedený iba čas prvého behu algoritmu. 

Rôzne algoritmy sme testovali na rôznych dátach. Najmä podľa toho, aké veľké 
dáta vedeli spracovať v rozumnom čase. Testovacie dáta uvádzame v ďalšej časti.

\section{Testovacie dáta}

Testovacie dáta boli grafy rôznych veľkostí a štruktúr. Naše testovacie dáta sa 
dajú rozdeliť do štyroch hlavných skupín. Podľa skupiny majú aj predponu. Sú to 
tieto:

\begin{itemize}
	\item grafy tvorené prioritným pripájaním tvoriace Barabási-Albertov model 
		-- majú predponu \texttt{ba-};
	\item grafy tvorené hranami s náhodnou pravdepodobnosťou vzniku tvoriace 
	náhodný model -- majú predponu \texttt{rnd-};
	\item grafy vytvorené špeciálne podľa obrázka \ref{img:zle1} -- majú 
		predponu \texttt{zle-};
	\item grafy vytvorené z reálnych dát -- majú predponu \texttt{ca-}.
\end{itemize}

Grafy tvorené prioritným pripájaním boli vytvorené pomocou programu Network 
Workbench. Ku grafu sa pridali v každom kroku dve hrany.
Grafy tvorené náhodnou pravdepodobnosťou vzniku hrán boli taktiež pomocou 
programu Network Workbench. Grafy boli vytvorené tak, aby mali približne 
rovnakú hustotu ako grafy Barabási-Albertoveho modelu rovnakej veľkosti.
Reálne dáta boli prevzaté zo stránky \url{http://snap.stanford.edu/data/}.

Podľa počtu vrcholov majú názvy jednotlivých grafov príponu. Pre grafy 
vytvorené prioritným pripájaním a náhodné grafy, číslo za predponou vyjadruje 
počet vrcholov grafu. Pre grafy vytvorené podľa obrázka \ref{img:zle1} číslo 
vyjadruje počet vrcholov na "hlavnej" ceste. Pre grafy reálnych dát boli čísla 
zvolené umelo a nemajú žiaden význam. Všetky použité reálne dáta boli siete 
citácií vedeckých publikácií.

\section{Porovnanie algoritmov}

V tejto časti porovnáme medzi sebou jednotlivé skupiny algoritmov. Začneme 
algoritmami dávajúcimi presné výsledky, budeme pokračovať všeobecným porovnaním 
a na koniec zhodnotíme jednotlivé heuristiky pažravého algoritmu.

\subsection{Presné algoritmy}

Do tejto kategórie spadajú algoritmy \alg{naive}, \alg{fnaive} a \alg{fproper}. 
Boli implementované najmä kvôli prehľadu a získaniu veľkostí minimálnych 
dominujúcich množín. Ako vidno zo všetkých tabuliek (\ref{table:java-stare}, 
\ref{table:cpp} a aj \ref{table:java}), pri exaktných algoritmoch nemôžeme 
rátať s tým, že sa dočkáme výsledkov na reálnych dátach.

Môžeme však vidieť, že použitím vhodných úprav vieme posunúť veľkosť grafu, 
kedy vieme vypočítať minimálnu dominujúcu množinu v rozumnom čase, zo zhruba 
20 vrcholov na zhruba 100 vrcholov. Taktiež je dobré všimnúť si, že aj keď 
algoritmus \alg{fproper} používa oveľa zložitejšie výpočty, tak čas behu 
algoritmu to oveľa nezrýchli. Prevedenie na hľadanie maximálneho párenia 
(respektíve minimálneho hranového pokrytia) je teda viac zlepšením teoretickej 
zložitosti ako pomoc reálnemu behu algoritmu.

Celkom zaujímavým je pozorovanie, že na neštruktúrovaných dátach (\texttt{rnd-} 
grafy) mali algoritmy \alg{fnaive} a \alg{fproper} oveľa horší čas ako na 
dátach so štruktúrou (\texttt{ba-} a \texttt{zle-} grafy).

\subsection{Všeobecné porovnanie}

Skôr ako sa pozrieme na celkové hodnotenie, tak sa môžeme pozrieť na algoritmy 
\alg{ch7alg34OT} a \alg{ch7alg35OT}. V tabuľke \ref{table:java-stare} je vidno, 
žo algoritmus \alg{ch7alg35OT} naozaj rieši uvedené problémy v časti 
\ref{sec:dist}. Škoda, že sa nepodarilo vytvoriť prostredie pre distribuovaný 
algoritmus.

Celkovo si môžeme všimnúť (tabuľka \ref{table:java}), že algoritmy bežia 
rýchlejšie na štruktúrovaných dátach (\texttt{ba-} a \alg{zle-}) okrem 
algoritmu \alg{ch7alg33OT}, ktorý beží lepšie na neštruktúrovaných dátach 
(\texttt{rnd-}). 

Keďže štruktúrované aj neštruktúrované dáta majú pomerne rovnakú hustotu hrán, 
tak sme neporovnávali algoritmy na rôzne hustých grafoch.

\subsection{Porovnanie hauristík pažravých algoritmov}

Zatiaľ sme sa pozerali čisto iba na časy. No pri porovnávaní reálnych časov 
pažravých algoritmov nám o rýchlosti veľa prezradí aj veľkosť výslednej 
množiny. Veľkosti výslednej množiny sú uvedené v tabuľke \ref{table:size}. 
Za príklad si môžeme zobrať trebárs základný algoritmus \alg{greedy}. Tam, 
kde mal menšiu výslednú množinu, tam mal aj nižší čas. Napríklad, pre grafy 
\texttt{ba20k} a \texttt{rnd20k} bol rozdiel vo veľkosti nájdených množín 1347. 
To znamená, že sa základný cyklus musel zopakovať oveľa viac krát, čo vyústilo 
do horšieho času algoritmu. Keďže všetky verzie pažravého algoritmu až na 
verziu \alg{flower} mali na začiatku behu algoritmu (pred opakujúcim sa cyklom) 
takmer nulové predpočítavanie rôznych hodnôt, respektíve vyberanie vrcholov do 
výsledku, Tak je tento trend badať na všetkých algoritmoch. 

Algoritmus \alg{flower} je v tomto smere výnimkou. Keďže si predpočítava 
susednosť do vzdialenosti tri a grafy tvorené Barabási-Albertovým modelom sú 
viac spojité ako náhodné grafy, tak sa počet opakovaní následného cyklu prejaví 
až pri grafoch s veľkým počtom vrcholov (v našom prípade \texttt{-20k}).

Keďže v Barabási-Albertovom modely sa každý nový vrchol spojí s práve (v našom 
prípade) dvoma inými, tak nevznikajú výhonky. Taktiež nie je šanca ani na 
vytvorenie kvetov. Takže veľa heuristík založených na odstraňovaní výhonkov a 
kvetov nič neodstráni a teda výsledkom je taká istá množina ako pri základnom 
pažravom algoritme.

\subsection{Porovnanie pažravých algoritmov na reálnych dátach}

V reálnych dátach je situácia iná ako v modelových. V reálnych dátach sa tvoria 
malé lokálne klastre, ktoré sú občas hustejšie prepojené a vytvoria kvet. 
Taktiež je šanca, že nejaký vrchol bude izolovaný alebo takmer izolovaný. Teda 
má zmysel vytvárať heuristiky na základe vyberania výhonkov a kvetov. Keďže 
reálne siete mali nad 5000 vrcholov, iné ako pažravé algoritmy nebolo možné 
testovať. Výsledky testovania sú uvedené v tabuľke \ref{table:real}. 

Algoritmy \alg{ch7alg33} a \alg{greedyq} boli navrhnuté s tým, že budú bežať 
rýchlejšie, ale aj oveľa nepresnejšie, čo sa potvrdilo. Bohužiaľ miera 
nepresnosti na reálnych dátach je príliš veľká na akékoľvek použitie.

Najjednoduchší pažravý algoritmus (\alg{greedy}) dáva 
prekvapivo veľmi dobré výsledky. Je prekonaný iba algoritmom \alg{greedysw}, 
ktorý je rozšírením jednoduchého pažravého algoritmu iba o vhodný spôsob 
výberu vrchola. Najzložitejší, algoritmus \alg{flower}, síce našiel najmenšiu 
dominújúcu množinu spomedzi testovaných algoritmov na dvoch grafoch 
(\texttt{ca1} a \texttt{ca22}), ale mal horšie časy. 

Grafy \texttt{ca22} a \texttt{ca3} sú hustejšie ako ostatné, čo sa prejavilo aj 
na pomalom behu algoritmu \alg{greedysw}.

\section{Tabuľky}

V tejto časti sa nachádzajú tabuľky zobrazujúce výsledky z testovania 
algoritmov.

\begin{table}[h]
	\centering
	\begin{tabular}{lllll}
		\hline
		& naive & greedy & ch7alg33 & ch7alg34OT \\ \hline
		ba10    & 0     & 0      & 0        & 0          \\
		ba18    & 1     & 0      & 0        & 0          \\
		ba20    & 4,37  & 0      & 0        & 0          \\
		ba100   &       & 0      & 0        & 0          \\
		ba200   &       & 0      & 0,01     & 0,06       \\
		ba1000  &       & 0      & 0        & 1,28       \\
		ba2000  &       & 0,01   & 0,02     & 5,15       \\
		ba10k   &       & 0,08   & 0,51     & 120,32     \\
		ba20k   &       & 0,29   & 2,68     &            \\
		ba100k  &       & 6,52   & 147,56   &            \\
		rnd100  &       & 0      & 0        & 0,03       \\
		rnd200  &       & 0      & 0        & 0,04       \\
		rnd1000 &       & 0      & 0        & 0,21       \\
		rnd10k  &       & 0,07   & 0,53     & 5,6        \\
		zly10   &       & 0      & 0        & 0          \\
		zly20   &       & 0      & 0        & 0,01       \\
		zly100  &       & 0,02   & 0,22     & 4,72       \\ \hline
	\end{tabular}
	\caption{Výsledky behov algoritmov v jazyku \cpp\ na staršom počítači. Výsledky sú uvedené v sekundách.}
	\label{table:cpp}
\end{table}

\begin{table}[h]
	\centering
	\begin{tabular}{lllllllll}
		\hline
		         & ca1 - |S| & ca1 - čas & ca2 - |S| & ca2 - čas & ca22 - |S| & ca22 - čas & ca3 - |S| & ca3 - čas \\ \hline
		greedy   & 1176      & 0,124     & 2009      & 0,578     & 2275       & 0,330      & 3694      & 1,311     \\
		ch7alg33 & 1570      & 0,070     & 3243      & 0,121     & 3230       & 0,099      & 5651      & 0,392     \\
		greedyQ  & 1586      & 0,034     & 3239      & 0,087     & 3243       & 0,080      & 5625      & 0,319     \\
		greedysr & 1803      & 0,072     & 3100      & 0,340     & 3667       & 0,118      & 5553      & 0,708     \\
		greedysw & 1159      & 0,348     & 1989      & 1,613     & 2258       & 1,022      & 3649      & 3,548     \\
		floweru  & 1329      & 0,111     & 2131      & 0,511     & 2501       & 0,313      & 3920      & 1,196     \\
		flower   & 1157      & 0,293     & 2018      & 2,406     & 2256       & 0,591      & 3774      & 2,830     \\ \hline
	\end{tabular}
	\caption{\emph{Testovanie rôznych heuristík na dátach z reálneho sveta.} %
		V stĺpci |S| je veľkosť nájdenej množiny. Čas je uvedený v sekundách.}
	\label{table:real}
\end{table}

\begin{table}[h]
	\centering
	\begin{tabular}{lllllllll}
		\hline
		& naive & greedy & greedyQ & ch7alg33 & ch7alg34OT & ch7alg35OT & fnaive   & fproper \\ \hline
		ba10    & 0,085 & 0,001  & 0,002   & 0        & 0,003      & 0,004      & 0,008    & 0,009   \\
		ba18    & 1,284 & 0,001  & 0,002   & 0        & 0,013      & 0,008      & 0,035    & 0,035   \\
		ba20    & 4,440 & 0,001  & 0,002   & 0        & 0,011      & 0,008      & 0,049    & 0,047   \\
		ba100   &       & 0,015  & 0,004   & 0,004    & 0,070      & 0,045      & 0,473    & 0,506   \\
		ba200   &       & 0,026  & 0,006   & 0,009    & 0,135      & 0,073      & 30,515   & 30,412  \\
		ba1000  &       & 0,087  & 0,045   & 0,037    & 0,883      & 0,512      &          &         \\
		ba2000  &       & 0,142  & 0,095   & 0,053    & 2,524      & 0,988      &          &         \\
		ba10k   &       & 1,844  & 0,362   & 0,540    & 45,781     & 6,662      &          &         \\
		ba20k   &       & 8,813  & 1,294   & 3,800    &            & 22,256     &          &         \\
		ba100k  &       & 66,762 & 68,669  & 135,197  &            &            &          &         \\
		rnd10   &       & 0,001  & 0,002   &          &            &            & 0,003    & 0,003   \\
		rnd15   &       & 0,001  & 0,002   &          &            &            & 0,048    & 0,043   \\
		rnd20   &       & 0,001  & 0,002   &          &            &            & 0,107    & 0,112   \\
		rnd100  &       & 0,011  & 0,004   &          &            &            & 1293,811 & 1302,940\\
		rnd200  &       & 0,032  & 0,008   &          &            &            &          &         \\
		rnd1000 &       & 0,072  & 0,043   &          &            &            &          &         \\
		rnd2000 &       & 0,114  & 0,096   &          &            &            &          &         \\
		rnd10k  &       & 2,484  & 0,440   &          &            &            &          &         \\
		rnd20k  &       & 11,281 & 1,671   & 4,947    &            &            &          &         \\
		zly10   &       & 0,005  & 0,003   & 0,003    & 0,040      & 0,018      & 0,119    & 0,117   \\
		zly20   &       & 0,022  & 0,010   & 0,022    & 0,098      & 0,036      & 0,809    & 0,814   \\
		zly100  &       & 0,090  & 0,151   & 0,210    & 6,213      & 2,107      &          &         \\ \hline
	\end{tabular}
	\caption{Výsledky behov algoritmov v jazyku \Java\ na staršom počítači. Výsledky sú uvedené v sekundách.}
	\label{table:java-stare}
\end{table}

\begin{landscape}

\begin{table}[h]
	\centering
	\begin{tabular}{lllllllllllll}
		\hline
		& naive & greedy & greedyQ & ch7alg33 & greedysr & greedysw & floweru & flower & ch7alg34OT & ch7alg35OT & fnaive & fproper \\ \hline
		ba10    & 2     & 2      & 2       & 2        & 2        & 2        & 2       & 2      & 6          & 6          & 2      & 2       \\
		ba18    & 4     & 4      & 6       & 6        & 5        & 4        & 4       & 4      & 9          & 11         & 4      & 4       \\
		ba20    & 4     & 4      & 7       & 5        & 6        & 4        & 4       & 5      & 10         & 11         & 4      & 4       \\
		ba100   &       & 17     & 33      & 31       & 17       & 17       & 17      & 18     & 54         & 61         & 15     & 15      \\
		ba200   &       & 32     & 59      & 56       & 31       & 35       & 31      & 31     & 102        & 113        & 29     & 29      \\
		ba1000  &       & 142    & 284     & 290      & 147      & 151      & 147     & 146    & 515        & 558        &        &         \\
		ba2000  &       & 276    & 570     & 577      & 273      & 284      & 273     & 274    & 1005       & 1091       &        &         \\
		ba10k   &       & 1393   & 2901    & 2974     & 1391     & 1465     & 1391    & 1390   & 5094       & 5593       &        &         \\
		ba20k   &       & 2828   & 5749    & 5896     & 2828     & 2942     & 2828    & 2817   &            & 11093      &        &         \\
		ba100k  &       & 13947  & 29378   & 29375    & 13947    & 14656    & 13947   & 13928  &            &            &        &         \\
		rnd10   & 2     & 2      & 2       & 2        & 2        & 2        & 2       & 2      & 7          &            & 2      & 2       \\
		rnd15   & 3     & 4      & 3       & 4        & 3        & 3        & 3       & 3      & 10         &            & 3      & 3       \\
		rnd20   & 4     & 4      & 5       & 5        & 4        & 5        & 4       & 4      & 15         &            & 4      & 4       \\
		rnd100  &       & 21     & 28      & 24       & 22       & 22       & 19      & 19     & 62         &            & 18     & 18      \\
		rnd200  &       & 41     & 52      & 53       & 40       & 39       & 40      & 39     & 126        &            &        &         \\
		rnd1000 &       & 201    & 284     & 281      & 202      & 210      & 200     & 199    & 633        &            &        &         \\
		rnd2000 &       & 411    & 541     & 552      & 414      & 411      & 404     & 399    & 1272       &            &        &         \\
		rnd10k  &       & 2083   & 2775    & 2760     & 2122     & 2098     & 2035    & 2024   & 6325       &            &        &         \\
		rnd20k  &       & 4175   & 5560    & 5542     & 4279     & 4193     & 4091    & 4053   &            &            &        &         \\
		zly10   &       & 10     & 35      & 35       & 10       & 10       & 10      & 10     & 10         & 10         & 10     & 10      \\
		zly20   &       & 20     & 120     & 120      & 20       & 20       & 20      & 20     & 20         & 20         & 20     & 20      \\
		zly100  &       & 100    & 2600    & 2600     & 100      & 100      & 100     & 100    & 100        & 100        &        &         \\ \hline
	\end{tabular}
	\caption{Veľkosti nájdených dominujúcich množín pre daný graf a algoritmus.}
	\label{table:size}
\end{table}

\end{landscape}

\begin{landscape}
\begin{table}[h]
	\centering
	\begin{tabular}{llllllllllll}
		\hline
		& naive & greedy & greedyQ & ch7alg33 & greedysr & greedysw & floweru & flower & ch7alg34OT & fnaive  & fproper \\ \hline
		ba10    & 0,010 & 0      & 0       & 0        & 0        & 0        & 0       & 0,001  & 0,001      & 0,001   & 0,001   \\
		ba18    & 0,331 & 0      & 0       & 0        & 0        & 0        & 0       & 0,001  & 0,001      & 0,001   & 0,001   \\
		ba20    & 1,321 & 0      & 0       & 0        & 0        & 0        & 0       & 0,001  & 0,001      & 0,001   & 0,001   \\
		ba100   &       & 0,003  & 0       & 0,002    & 0,001    & 0,003    & 0,001   & 0,004  & 0,016      & 0,084   & 0,145   \\
		ba200   &       & 0,004  & 0       & 0,003    & 0,001    & 0,006    & 0,001   & 0,007  & 0,043      & 15,595  & 15,577  \\
		ba1000  &       & 0,021  & 0,003   & 0,013    & 0,011    & 0,040    & 0,011   & 0,046  & 0,358      &         &         \\
		ba2000  &       & 0,047  & 0,008   & 0,025    & 0,024    & 0,740    & 0,025   & 0,090  & 1,322      &         &         \\
		ba10k   &       & 0,308  & 0,079   & 0,115    & 0,251    & 0,933    & 0,249   & 0,952  & 21,37      &         &         \\
		ba20k   &       & 0,987  & 0,290   & 0,410    & 0,837    & 3,097    & 0,873   & 3,368  &            &         &         \\
		ba100k  &       & 19,602 & 6,831   & 19,201   & 21,315   & 65,587   & 20,743  & 60,146 &            &         &         \\
		rnd10   & 0,011 & 0      & 0       & 0        & 0        & 0        & 0       & 0      & 0          & 0       & 0       \\
		rnd15   & 0,058 & 0      & 0       & 0        & 0        & 0        & 0       & 0      & 0          & 0       & 0       \\
		rnd20   & 1,445 & 0      & 0       & 0        & 0        & 0        & 0       & 0      & 0          & 0,001   & 0,001   \\
		rnd100  &       & 0,003  & 0       & 0        & 0        & 0        & 0       & 0,001  & 0,002      & 634,429 & 631,217 \\
		rnd200  &       & 0,004  & 0       & 0,001    & 0        & 0,001    & 0       & 0,002  & 0,008      &         &         \\
		rnd1000 &       & 0,019  & 0,001   & 0,007    & 0,004    & 0,015    & 0,004   & 0,016  & 0,105      &         &         \\
		rnd2000 &       & 0,045  & 0,004   & 0,020    & 0,013    & 0,048    & 0,013   & 0,048  & 0,303      &         &         \\
		rnd10k  &       & 0,384  & 0,101   & 0,120    & 0,294    & 0,984    & 0,313   & 0,929  & 2,827      &         &         \\
		rnd20k  &       & 1,337  & 0,390   & 0,478    & 1,224    & 3,833    & 1,260   & 3,703  &            &         &         \\
		zly10   &       & 0      & 0       & 0        & 0        & 0        & 0       & 0      & 0          & 0       & 0       \\
		zly20   &       & 0      & 0       & 0        & 0        & 0        & 0       & 0,001  & 0,001      & 0,009   & 0,006   \\
		zly100  &       & 0,027  & 0,016   & 0,037    & 0,003    & 0,100    & 0,011   & 0,088  & 0,662      &         &         \\ \hline
	\end{tabular}
	\caption{Výsledky behov algoritmov v jazyku \Java\ na novšom počítači. Výsledky sú uvedené v sekundách.}
	\label{table:java}
\end{table}
\end{landscape}

\backmatter
\cleardoublepage
\phantomsection
\addcontentsline{toc}{chapter}{Záver}
\chapter*{Záver}\label{chap:zaver}

V práci sme sa snažili naplniť ciele zadania. A to spraviť prehľad algoritmov 
hľdajúcich minimálne dominujúce množiny a skúsiť vylepšiť niektoré algoritmy, 
ktoré hľadajú minimálne dominujúce množiny na sieťach malého sveta. 

V kapitole \ref{chap:definicie} sme popísali základné pojmy a uviedli čitateľa 
do problematiky grafov, komplexných sietí a hľadania minimálnej dominujúcej 
množiny. Kapitolu \ref{chap:algoritmy} sme začali spísaním prehľadu súčasných 
základných existujúcich algoritmov na hľadanie minimálnej dominujúcej množiny. 
Ďalej sme v tejto kapitole načrtli výhody v prípade, že vstupným grafom je 
komplexná sieť. Nakoniec sme navrhli niekoľko heuristík, o ktorých sme 
predpokladali, že nejako zlepšia čas behu alebo výsledky algoritmov. Keďže 
sme sa v práci chceli zaoberať dátami z reálneho sveta, zvolili sme si 
heuristiky na zlepšenie tých algoritmov, ktoré majú šancu spracovať také veľké 
dáta.

V kapitolách \ref{chap:popis} a \ref{chap:implementacia} sme popísali návrh a 
implementáciu softvéru. Preskúmali sme existujúce riešenia a na základe analýzy 
navrhli, ako by mala implementačná časť práce vyzerať.
Softvér nakonie nenadväzoval na žiadne predošlé práce, aj keď 
používal pomocnú knižnicu (HPPC) na uchovanie základných dátových štruktúr. 
V konečnom dôsledku malo použitie externej knižnice pozitívny dopad. 
Implementačnú časť práce sme sa rozhodli spraviť samostatne a nenadväzovať 
kvôli úzkej špecifikácií algoritmov a pohodlnosti práce s vlastnými 
zložitejšími dátovými a programovými štruktúrami oproti existujúcim riešeniam.

V kapitole \ref{chap:vysledky} sme opísali výsledky z implementácie softvéru a 
porovnali sme algoritmy medzi sebou a na rôznych dátach, včetne reálnych sietí 
citácií. Z navrhovaných heuristík mala každá svoje slabé a silné miesta. 
Čo dáva priestor na ďalšie zlepšovania. 

Veľmi potešujúce však je, že sme našli heuristiku, ktorá na všetkých 
testovaných reálnych dátach prekonala veľkosťou dominujúcej množiny 
jednoduchý pažravý algoritmus (verzia \alg{greedysw}). Iba o trochu viac 
sofistikovaný spôsob vyberania (uprednostňuj vrcholy, ktoré svojím vybratím 
čo najviac zmenšia množinu vrcholov, ktoré majú nejakého bieleho suseda) 
dosiahol zlepšenie. Zlepšenie sa nepodarilo vždy pri algoritme so zložitejšou 
štruktúrou -- verzii \alg{flower}.
Tento výsledok odráža skutočnosť, že často algoritmy s 
jednoduchou štruktúrou bývajú tým najvhodnejším riešením.

V tejto práci sme nevyužili vedomosť, že siete malého sveta sú síce odolné voči 
náhodným útokom, no zraniteľné cielenými útokmi v tom zmysle, že ich vieme 
odobratím malého počtu hrán rozdeliť na viacero komponentov. Je možné, že keby 
sme vedeli, tieto hrany efektívne určiť, mohli by sme skúmať, ako dôležité sú 
vrcholy, ktoré sú incidentné s hranou a na základe toho ich prioritne ne/vybrať 
do výslednej množiny.

Vďaka tomu, že vieme, že siete malého sveta sú riedke grafy, môžeme tieto grafy 
rozdeliť na množinu stromov, na ktorých sa minimálna dominujúca množina hľadá 
ľahšie. Je však otázne, ako vyriešiť vrcholy, v ktorých tieto stromy v pôvodnom 
grafe susedia. Aj keď sa tento prístup zdá byť prijateľný, tak vieme, že siete 
malého sveta majú veľa lokálnych klastrov a pridelení týchto klastrov by 
vznikalo veľa stromov, čo by mohlo viesť ku vybratiu veľa "zbytočných" 
vrcholov. Vďaka tejto komplikácií a tomu, že sa tento spôsob riešenia výrazne 
odlišuje od zvyšných, sme takýto postup hľadania minimálnej dominujúcej množiny 
nakoniec vynechali.

V ďalšej činnosti je teda vhodné zamerať sa podrobnejšie na to, aké heuristiky 
dávajú lepšie výsledky pre rôzne typy dát ako napríklad hustejšie/redšie grafy, 
grafy s väčšou/menšou klasterizáciou alebo trebárs rôzne spôsoby vyberania 
výhonkov. To by umožňovalo vzniku možno lepšej heuristiky (či už z pohľadu času 
alebo veľkosti nájdenej množiny) alebo kombinovaného algoritmu s využitím 
viacerých heuristík (podobne ako naša verzia \alg{flower}). Taktiež je vhodné 
zamerať sa na dobré implementovanie distribuovaných algoritmov. Buď vylepšiť 
algoritmy \alg{ch7alg34} a \alg{ch7alg35} tým, že lepšie využijeme nejaké 
distribuované prostredie alebo skúsiť rozobrať distribuovanú verziu vynechaného 
algoritmu rozdelenia grafu na stromy. Taktiež by v ďalšej práci bolo vhodné 
výsledky vizualizovať. Vhodná vizualizácia by mohla pomôcť pri odstraňovaní 
slabých miest heuristík a priniesť veľa podnetov na tvorbu ďalších heuristík.


\cleardoublepage
\phantomsection
\addcontentsline{toc}{chapter}{Literatúra}

\printbibliography

\backmatter

\newpage
\pagestyle{empty}
\hbox{}
 
\end{document}

