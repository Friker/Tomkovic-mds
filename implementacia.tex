\def\alg#1{\texttt{#1}}

\chapter{Implementácia softvéru}\label{chap:implementacia}

V tejto kapitole postupne uvedieme realizáciu návrhu popísaného 
v~kapitole~\ref{chap:popis}. Popíšeme ako sme implementovali jednotlivé 
algoritmy a ukážeme si štruktúru a ovládanie aplikácie. 

% \todo{
% Programovo realizujeme podrobný návrh databázovej aplikácie. 
% Vypracuje sa podrobná dokumentácia k vytvorenému produktu. 
% Vytvorí sa používateľská príručka, kde sa podrobne dokumentuje používateľské rozhranie, spresní sa popis funkcií a spôsob ich aktivácie.
% }

\section{Algoritmy}

Aj keď sme porovnávali iba 10 algoritmov, dokopy bolo implementovaných 
algoritmov viac. V následujúcich častiach postupne uvedieme jednotlivé 
algoritmy a ich označenia pre aplikáciu. Algoritmy boli implementované podľa 
popisu v~kapitole~\ref{chap:algoritmy}.

Pred tým, než rozpíšeme implementáciu algoritmov uvedieme štruktúru a 
implementáciu grafu. Graf je implementovaný ako objekt, ktorý obsahuje 
zoznam hrán. Pokiaľ je to potrebné (čo zväčša je), vie sa iniciovať a zavolať 
vypočítanie susedov do vzdialenosti 2 pre každý vrchol (aby sa nemuselo počítať 
pri každom použití ale iba raz, na začiatku). Taktiež objekt reprezentácie 
grafu vie skontrolovať, či je nejaká množina dominujúcou. To je spravené 
pomocou delegácie na algoritmus kontroly dominujúcej množiny.

Nasleduje popis jednotlivých skupín algoritmov s ich označeniami.

\subsection{Algoritmy skúšajúce všetky možnosti}

Prvým a základným algoritmom je algoritmus, ktorý skúša všetky možnosti. Jeho 
označenie v softvéri je \alg{naive}. Je implementovaný rekurzívnym 
prehľadávaním. Obdobou sú algoritmy označené \alg{mynaive} a \alg{mynaive2}, 
ktoré obsahuje základné heuristiky. Prvou je, že algoritmus najprv zoradí pole 
vrcholov podľa stupňa a až potom začne vyberať do výsledku. Druhou je, že ak 
je medzivýsledok menšej mohutnosti, tak neskúša vyberať množiny s väčšou 
mohutnosťou.

\subsection{Algoritmy prevedenia problému}

Ďalšími algoritmami, ktoré sme implementovali, boli algoritmy, ktoré prevádzali 
hľadanie minimálnej dominujúcej množiny na problém množinového pokrytia. Keďže 
algoritmus bol reprezentovaný návrhovým vzorom \emph{strategy}, tak 
sa implementácia zjednodušila a sprehľadnila. Časť algoritmu sa venovala 
spracovaniu vstupu a časť výpočtu a implementovaniu náročnejších algoritmov. 

Aj keď \citet{fomin05} uvádzajú algoritmus jeden, my sme implementovali dve rôzne 
verzie: \alg{fnaive} a \alg{fproper}. Verzia \alg{fnaive} neobsahuje finálne 
prevedenie na hľadanie minimálneho hranového pokrytia v grafe. Verzia 
\alg{fproper} toto prevedenie obsahuje a implementuje ho cez hľadanie 
maximálneho párenia v grafe. Keďže v čase použitia algoritmu je graf 
bipartitný, tak sa to dá spraviť.

\subsection{Distribuované algoritmy}

Podobne ako predošlé algoritmy, aj tieto algoritmy mali viac verzií. Konkrétne 
jednovláknovú a viacvláknovú. Distribuovanosť bola implementovaná cez veľa 
na sebe nezávislých vlákien. Nedistribuované verzie sme označili príponou 
\alg{-OT} (one thread).

Algoritmy sme implementovali tak ako v prehľade -- dva. Prvý bez riešenia 
problémových grafov (obrázky \ref{img:zle1} a \ref{img:zle2}), ale prehľadnejší 
a zrozumiteľnejší. Druhý s riešením týchto problémov. Prvý je označený ako 
\alg{ch7alg34} (resp.~\alg{ch7alg34OT} pre jednovláknovú verziu) a druhý je 
označený ako \alg{ch7alg35} (podobne \alg{ch7alg35OT} pre jednovláknovú verziu).

\subsection{Pažravé algoritmy}

Keďže jedným z hlavných cieľov práce bolo spraviť dobrú heuristiku na pažravý 
algoritmus, tak boli tieto algoritmy implmentované vo viacerých verziách. Okrem 
rôznych heuristík ide aj o spôsob vyberania vrcholov. Počas behu algoritmu sú 
niektoré vrcholy vybraté do medzivýsledku, niektoré pokryté a niektoré 
nevybraté a nepokryté. Tie vrcholy, ktoré sú nevybraté a nepokryté sú 
označované aj ako \emph{biele}, tie ktoré nie sú vybraté, ale sú pokryté sú 
označované ako \emph{šedé} a tie, ktoré sú vybraté sú označované ako 
\emph{čierne}.