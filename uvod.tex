\cleardoublepage
% \phantomsection
\addcontentsline{toc}{chapter}{Úvod}
\chapter*{Úvod}\label{chap:intro}

Modely sietí majú veľký úspech. Najmä vďaka tomu, že grafy vytvorené podľa 
týchto modelov majú podobné vlastnosti ako skutočné siete. Internet, elektrické 
rozvody, sociálne kontakty, internetové sociálne siete, či neurónové prepojenia 
v mozgu majú navzájom podobné vlastnosti. A tieto vlastnosti sú zachytené aj v 
modeloch sietí. Prvými priekopníkmi v skúmaní reálnych sietí bolí maďarskí 
vedci Paul Erdős, Alfréd Rényi and Béla Bollobás. Reálne siete sa snažili 
popísať pomocou náhodných modelov \citep{erdos:rnd}. Čoskoro zistili, že 
siete v reálnom svete tieto modely nepopisujú veľmi dobre. Siete v reálnom 
svete vznikajú inak ako náhodne a tak sa vedci snažili o vernejší popis 
modelmi, ktoré by zachytávali vlastnosti reálnych sietí. Siete malého sveta 
boli spopularizované po tom, ako Stenley Milgram spravil výskum, v ktorom 
potvrdzoval tézu, že v Spojených Štátoch amerických sa všetci 
ľudia poznajú vo väčšine cez najviac šiestich známych \citep{kochen}. Na to 
popísali \citet{barabasi:albert} bezškálové siete, ktoré boli charakteristické 
krátkymi cestami medzi náhodnými dvojicami vrcholov, podobnosťou na rôznych 
škálach a odolnosťou proti náhodným útokom \citep{barabasi:albert:2}.

Reálne siete môžeme skúmať z viacerých pohľadov. Jedným z nich je aj hľadanie 
množiny vrcholov, ktoré sú pre danú sieť významné. Pre sieť kontaktov to budú 
ľudia, ktorý majú tých kontaktov najviac, pre sieť stretnutí sa rôznych ľudí 
to môžu byť ľudia, ktorý sa stretli s veľa inými ľuďmi a podobne. V týchto 
úlohách často hľadáme nejakým spôsobom vrcholy s najväčším stupňom alebo 
množinu, ktorá má za susedov všetkých ostatných ľudí. Takáto množina sa nazýva 
minimálna dominujúca množina. Ide o známu úlohu z teórie grafov, ktorého 
rozhodovacia verzia je NP-úplný problém \citep{npcomp}. Keďže reálne siete sa 
skladajú z tisícov vrcholov, výpočtová sila súčasnosti nestačí na presné 
riešenie týchto problémov. Preto je momentálnym riešením použitie 
aproximačných alebo pažravých algoritmov. Keďže považujeme túto tému za 
zaujímavú, rozhodli sme sa preskúmať bližšie rôzne heuristiky pažravého 
algoritmu. Taktiež výsledky z algoritmov dosiahnuté v tejto práci môžu pomôcť 
iným prácam k lepšiemu vstupu/spracovaniu, napríklad k práci, ktorú napísala 
Dana \citet{sunikova}.

Práca je rozdelená do piatich kapitol. Najprv zadefinujeme rôzne pojmy, ktoré 
v práci používame. To spravíme v kapitole~\ref{chap:definicie}. Potom uvedieme 
prehľad existujúcich algoritmov na hľadanie minimálnych dominujúcich množín 
(kapitola~\ref{chap:algoritmy}). Následne v kapitolách \ref{chap:popis} a 
\ref{chap:implementacia} popíšeme návrh a implementáciu softvéru. Výsledky 
implementačnej práce zhrnieme v kapitole~\ref{chap:vysledky}.




