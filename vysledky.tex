\chapter{Dosiahnuté výsledky}\label{chap:vysledky}

V predošlých kapitolách sme spravili prehľad existujúcich algoritmov, popísali 
sme ich implementáciu. V tejto kapitole uvádzame porovnanie jednotlivých 
implementácií. Najprv uvedieme jednotlivé obmedzenia implementácie, potom 
popíšeme spôsob testovania a napokon označenie pre jednotlivé testovacie dáta.

\section{Obmedzenia implementácie}

Vývoj softvéru sa začal na staršom počítači, ktorý má procesor Intel Core 2 
Duo P7350 (2,0GHz dvojjadrový) a 4GB RAM pamäte. Pamäť nebola obmedzením. 
Bohužiaľ počítač 
nebol vhodný na viacvláknové algoritmy, čo spôsobovalo problémy pre algoritmy 
\alg{ch7alg34} a \alg{ch7alg35}. Preto sme sa zamerali na ich jednovláknové 
verzie (\alg{ch7alg34OT}, respektíve \alg{ch7alg35OT}). Heuristiky na pažravý 
algoritmus sa doimplementovali neskôr, preto sa neuvádzajú v starších testoch.

Počas vývoja začal byť dostupný novší počítač, na ktorom sa testovali už všetky 
heuristiky pre pažravý algoritmus. Počítač má procesor Intel Core i5-4690K CPU 
(3,50GHz štvorjadrový) a 16GB pamäte. Tento počítač zvláda viacvláknové 
aplikácie lepšie (o 6 rokov modernejšia architektúra), no 
vzhľadom na zachovanie kompaktnosti sme opäť viacvláknové verzie algoritmov z 
testov vynechali.

Neskoršia práca s externými knižnicami HPPC časy zrýchlila, čo bohužiaľ nie je 
zachované v žiadnej z uvedených tabuliek.

\section{\Java\ verzus \cpp}

Zo začiatku implementácie nebolo jasné, ktorý programovací jazyk bude použitý. 
Ako prvý sa zvolil jazyk \Java. Keďže prevláda povedomie, že interpretovaný 
programovací jazyk (\Java) nemôže byť rýchlejší ako kopmilovaný programovací 
jazyk (\cpp), tak sme sa v priebehu implementácie rozhodli, že skúsime 
naprogramovať niektoré algoritmy aj v jazyku \cpp. Ako vidno z porovnania časov 
behov jednotlivých algoritmov v tabuľkách \ref{table:cpp} a 
\ref{table:java-stare}, jazyk \cpp\ neposkytuje žiadne 
výrazné zrýchlenie. Spolu s faktom, že na porovnanie jednotlivých algoritmov až 
tak rýchlosť netreba, sme sa rozhodli, že budeme pokračovať vo vývoji v jazyku 
\Java.

\section{Spôsob testovania}

Keďže programovací jazyk \Java\ robí optimalizácie kódu počas behu a triedy 
inicializuje "lenivo", tak je potrebné algoritmus spustiť viackrát, aby ukázal 
relevantné výsledky. Algoritmus teda necháme spustiť trikrát a zoberieme tretí 
výsledok. Rozhodnutie je vecou pozorovania. Je kompromisom medzi relevantnými 
výsledkami a dĺžkou skúšania rôznych behov. Z časov nad 20 sekúnd je v 
tabuľkách uvedený iba čas prvého behu algoritmu. 

Rôzne algoritmy sme testovali na rôznych dátach. Najmä podľa toho, aké veľké 
dáta vedeli spracovať v rozumnom čase. Testovacie dáta uvádzame v ďalšej časti.

\section{Testovacie dáta}

Testovacie dáta boli grafy rôznych veľkostí a štruktúr. Naše testovacie dáta sa 
dajú rozdeliť do štyroch hlavných skupín. Podľa skupiny majú aj predponu. Sú to 
tieto:

\begin{itemize}
	\item grafy tvorené prioritným pripájaním tvoriace Barabási-Albertov model 
		-- majú predponu \texttt{ba-};
	\item grafy tvorené hranami s náhodnou pravdepodobnosťou vzniku tvoriace 
	náhodný model -- majú predponu \texttt{rnd-};
	\item grafy vytvorené špeciálne podľa obrázka \ref{img:zle1} -- majú 
		predponu \texttt{zle-};
	\item grafy vytvorené z reálnych dát -- majú predponu \texttt{ca-}.
\end{itemize}

Grafy tvorené prioritným pripájaním boli vytvorené pomocou programu Network 
Workbench. Ku grafu sa pridali v každom kroku dve hrany.
Grafy tvorené náhodnou pravdepodobnosťou vzniku hrán boli taktiež pomocou 
programu Network Workbench. Grafy boli vytvorené tak, aby mali približne 
rovnakú hustotu ako grafy Barabási-Albertoveho modelu rovnakej veľkosti.
Reálne dáta boli prevzaté zo stránky \url{http://snap.stanford.edu/data/}.

Podľa počtu vrcholov majú názvy jednotlivých grafov príponu. Pre grafy 
vytvorené prioritným pripájaním a náhodné grafy, číslo za predponou vyjadruje 
počet vrcholov grafu. Pre grafy vytvorené podľa obrázka \ref{img:zle1} číslo 
vyjadruje počet vrcholov na "hlavnej" ceste. Pre grafy reálnych dát boli čísla 
zvolené umelo a nemajú žiaden význam. Všetky použité reálne dáta boli siete 
citácií vedeckých publikácií.

\section{Porovnanie algoritmov}

V tejto časti porovnáme medzi sebou jednotlivé skupiny algoritmov. Začneme 
algoritmami dávajúcimi presné výsledky, budeme pokračovať všeobecným porovnaním 
a na koniec zhodnotíme jednotlivé heuristiky pažravého algoritmu.

\subsection{Presné algoritmy}

Do tejto kategórie spadajú algoritmy \alg{naive}, \alg{fnaive} a \alg{fproper}. 
Boli implementované najmä kvôli prehľadu a získaniu veľkostí minimálnych 
dominujúcich množín. Ako vidno zo všetkých tabuliek (\ref{table:java-stare}, 
\ref{table:cpp} a aj \ref{table:java}), pri exaktných algoritmoch nemôžeme 
rátať s tým, že sa dočkáme výsledkov na reálnych dátach.

Môžeme však vidieť, že použitím vhodných úprav vieme posunúť veľkosť grafu, 
kedy vieme vypočítať minimálnu dominujúcu množinu v rozumnom čase, zo zhruba 
20 vrcholov na zhruba 100 vrcholov. Taktiež je dobré všimnúť si, že aj keď 
algoritmus \alg{fproper} používa oveľa zložitejšie výpočty, tak čas behu 
algoritmu to oveľa nezrýchli. Prevedenie na hľadanie maximálneho párenia 
(respektíve minimálneho hranového pokrytia) je teda viac zlepšením teoretickej 
zložitosti ako pomoc reálnemu behu algoritmu.

Celkom zaujímavým je pozorovanie, že na neštruktúrovaných dátach (\texttt{rnd-} 
grafy) mali algoritmy \alg{fnaive} a \alg{fproper} oveľa horší čas ako na 
dátach so štruktúrou (\texttt{ba-} a \texttt{zle-} grafy).

\subsection{Všeobecné porovnanie}

Skôr ako sa pozrieme na celkové hodnotenie, tak sa môžeme pozrieť na algoritmy 
\alg{ch7alg34OT} a \alg{ch7alg35OT}. V tabuľke \ref{table:java-stare} je vidno, 
žo algoritmus \alg{ch7alg35OT} naozaj rieši uvedené problémy v časti 
\ref{sec:dist}. Škoda, že sa nepodarilo vytvoriť prostredie pre distribuovaný 
algoritmus.

Celkovo si môžeme všimnúť (tabuľka \ref{table:java}), že algoritmy bežia 
rýchlejšie na štruktúrovaných dátach (\texttt{ba-} a \alg{zle-}) okrem 
algoritmu \alg{ch7alg33OT}, ktorý beží lepšie na neštruktúrovaných dátach 
(\texttt{rnd-}). 

Keďže štruktúrované aj neštruktúrované dáta majú pomerne rovnakú hustotu hrán, 
tak sme neporovnávali algoritmy na rôzne hustých grafoch.

\subsection{Porovnanie hauristík pažravých algoritmov}

Zatiaľ sme sa pozerali čisto iba na časy. No pri porovnávaní reálnych časov 
pažravých algoritmov nám o rýchlosti veľa prezradí aj veľkosť výslednej 
množiny. Veľkosti výslednej množiny sú uvedené v tabuľke \ref{table:size}. 
Za príklad si môžeme zobrať trebárs základný algoritmus \alg{greedy}. Tam, 
kde mal menšiu výslednú množinu, tam mal aj nižší čas. Napríklad, pre grafy 
\texttt{ba20k} a \texttt{rnd20k} bol rozdiel vo veľkosti nájdených množín 1347. 
To znamená, že sa základný cyklus musel zopakovať oveľa viac krát, čo vyústilo 
do horšieho času algoritmu. Keďže všetky verzie pažravého algoritmu až na 
verziu \alg{flower} mali na začiatku behu algoritmu (pred opakujúcim sa cyklom) 
takmer nulové predpočítavanie rôznych hodnôt, respektíve vyberanie vrcholov do 
výsledku, Tak je tento trend badať na všetkých algoritmoch. 

Algoritmus \alg{flower} je v tomto smere výnimkou. Keďže si predpočítava 
susednosť do vzdialenosti tri a grafy tvorené Barabási-Albertovým modelom sú 
viac spojité ako náhodné grafy, tak sa počet opakovaní následného cyklu prejaví 
až pri grafoch s veľkým počtom vrcholov (v našom prípade \texttt{-20k}).

Keďže v Barabási-Albertovom modely sa každý nový vrchol spojí s práve (v našom 
prípade) dvoma inými, tak nevznikajú výhonky. Taktiež nie je šanca ani na 
vytvorenie kvetov. Takže veľa heuristík založených na odstraňovaní výhonkov a 
kvetov nič neodstráni a teda výsledkom je taká istá množina ako pri základnom 
pažravom algoritme.

\subsection{Porovnanie pažravých algoritmov na reálnych dátach}

V reálnych dátach je situácia iná ako v modelových. V reálnych dátach sa tvoria 
malé lokálne klastre, ktoré sú občas hustejšie prepojené a vytvoria kvet. 
Taktiež je šanca, že nejaký vrchol bude izolovaný alebo takmer izolovaný. Teda 
má zmysel vytvárať heuristiky na základe vyberania výhonkov a kvetov. Keďže 
reálne siete mali nad 5000 vrcholov, iné ako pažravé algoritmy nebolo možné 
testovať. Výsledky testovania sú uvedené v tabuľke \ref{table:real}. 

Algoritmy \alg{ch7alg33} a \alg{greedyq} boli navrhnuté s tým, že budú bežať 
rýchlejšie, ale aj oveľa nepresnejšie, čo sa potvrdilo. Bohužiaľ miera 
nepresnosti na reálnych dátach je príliš veľká na akékoľvek použitie.

Najjednoduchší pažravý algoritmus (\alg{greedy}) dáva 
prekvapivo veľmi dobré výsledky. Je prekonaný iba algoritmom \alg{greedysw}, 
ktorý je rozšírením jednoduchého pažravého algoritmu iba o vhodný spôsob 
výberu vrchola. Najzložitejší, algoritmus \alg{flower}, síce našiel najmenšiu 
dominújúcu množinu spomedzi testovaných algoritmov na dvoch grafoch 
(\texttt{ca1} a \texttt{ca22}), ale mal horšie časy. 

Grafy \texttt{ca22} a \texttt{ca3} sú hustejšie ako ostatné, čo sa prejavilo aj 
na pomalom behu algoritmu \alg{greedysw}.

\section{Tabuľky}

V tejto časti sa nachádzajú tabuľky zobrazujúce výsledky z testovania 
algoritmov.

\begin{table}[h]
	\centering
	\begin{tabular}{lllll}
		\hline
		& naive & greedy & ch7alg33 & ch7alg34OT \\ \hline
		ba10    & 0     & 0      & 0        & 0          \\
		ba18    & 1     & 0      & 0        & 0          \\
		ba20    & 4,37  & 0      & 0        & 0          \\
		ba100   &       & 0      & 0        & 0          \\
		ba200   &       & 0      & 0,01     & 0,06       \\
		ba1000  &       & 0      & 0        & 1,28       \\
		ba2000  &       & 0,01   & 0,02     & 5,15       \\
		ba10k   &       & 0,08   & 0,51     & 120,32     \\
		ba20k   &       & 0,29   & 2,68     &            \\
		ba100k  &       & 6,52   & 147,56   &            \\
		rnd100  &       & 0      & 0        & 0,03       \\
		rnd200  &       & 0      & 0        & 0,04       \\
		rnd1000 &       & 0      & 0        & 0,21       \\
		rnd10k  &       & 0,07   & 0,53     & 5,6        \\
		zly10   &       & 0      & 0        & 0          \\
		zly20   &       & 0      & 0        & 0,01       \\
		zly100  &       & 0,02   & 0,22     & 4,72       \\ \hline
	\end{tabular}
	\caption{Výsledky behov algoritmov v jazyku \cpp\ na staršom počítači. Výsledky sú uvedené v sekundách.}
	\label{table:cpp}
\end{table}

\begin{table}[h]
	\centering
	\begin{tabular}{lllllllll}
		\hline
		         & ca1 - |S| & ca1 - čas & ca2 - |S| & ca2 - čas & ca22 - |S| & ca22 - čas & ca3 - |S| & ca3 - čas \\ \hline
		greedy   & 1176      & 0,124     & 2009      & 0,578     & 2275       & 0,330      & 3694      & 1,311     \\
		ch7alg33 & 1570      & 0,070     & 3243      & 0,121     & 3230       & 0,099      & 5651      & 0,392     \\
		greedyQ  & 1586      & 0,034     & 3239      & 0,087     & 3243       & 0,080      & 5625      & 0,319     \\
		greedysr & 1803      & 0,072     & 3100      & 0,340     & 3667       & 0,118      & 5553      & 0,708     \\
		greedysw & 1159      & 0,348     & 1989      & 1,613     & 2258       & 1,022      & 3649      & 3,548     \\
		floweru  & 1329      & 0,111     & 2131      & 0,511     & 2501       & 0,313      & 3920      & 1,196     \\
		flower   & 1157      & 0,293     & 2018      & 2,406     & 2256       & 0,591      & 3774      & 2,830     \\ \hline
	\end{tabular}
	\caption{\emph{Testovanie rôznych heuristík na dátach z reálneho sveta.} %
		V stĺpci |S| je veľkosť nájdenej množiny. Čas je uvedený v sekundách.}
	\label{table:real}
\end{table}

\begin{table}[h]
	\centering
	\begin{tabular}{lllllllll}
		\hline
		& naive & greedy & greedyQ & ch7alg33 & ch7alg34OT & ch7alg35OT & fnaive   & fproper \\ \hline
		ba10    & 0,085 & 0,001  & 0,002   & 0        & 0,003      & 0,004      & 0,008    & 0,009   \\
		ba18    & 1,284 & 0,001  & 0,002   & 0        & 0,013      & 0,008      & 0,035    & 0,035   \\
		ba20    & 4,440 & 0,001  & 0,002   & 0        & 0,011      & 0,008      & 0,049    & 0,047   \\
		ba100   &       & 0,015  & 0,004   & 0,004    & 0,070      & 0,045      & 0,473    & 0,506   \\
		ba200   &       & 0,026  & 0,006   & 0,009    & 0,135      & 0,073      & 30,515   & 30,412  \\
		ba1000  &       & 0,087  & 0,045   & 0,037    & 0,883      & 0,512      &          &         \\
		ba2000  &       & 0,142  & 0,095   & 0,053    & 2,524      & 0,988      &          &         \\
		ba10k   &       & 1,844  & 0,362   & 0,540    & 45,781     & 6,662      &          &         \\
		ba20k   &       & 8,813  & 1,294   & 3,800    &            & 22,256     &          &         \\
		ba100k  &       & 66,762 & 68,669  & 135,197  &            &            &          &         \\
		rnd10   &       & 0,001  & 0,002   &          &            &            & 0,003    & 0,003   \\
		rnd15   &       & 0,001  & 0,002   &          &            &            & 0,048    & 0,043   \\
		rnd20   &       & 0,001  & 0,002   &          &            &            & 0,107    & 0,112   \\
		rnd100  &       & 0,011  & 0,004   &          &            &            & 1293,811 & 1302,940\\
		rnd200  &       & 0,032  & 0,008   &          &            &            &          &         \\
		rnd1000 &       & 0,072  & 0,043   &          &            &            &          &         \\
		rnd2000 &       & 0,114  & 0,096   &          &            &            &          &         \\
		rnd10k  &       & 2,484  & 0,440   &          &            &            &          &         \\
		rnd20k  &       & 11,281 & 1,671   & 4,947    &            &            &          &         \\
		zly10   &       & 0,005  & 0,003   & 0,003    & 0,040      & 0,018      & 0,119    & 0,117   \\
		zly20   &       & 0,022  & 0,010   & 0,022    & 0,098      & 0,036      & 0,809    & 0,814   \\
		zly100  &       & 0,090  & 0,151   & 0,210    & 6,213      & 2,107      &          &         \\ \hline
	\end{tabular}
	\caption{Výsledky behov algoritmov v jazyku \Java\ na staršom počítači. Výsledky sú uvedené v sekundách.}
	\label{table:java-stare}
\end{table}

\begin{landscape}

\begin{table}[h]
	\centering
	\begin{tabular}{lllllllllllll}
		\hline
		& naive & greedy & greedyQ & ch7alg33 & greedysr & greedysw & floweru & flower & ch7alg34OT & ch7alg35OT & fnaive & fproper \\ \hline
		ba10    & 2     & 2      & 2       & 2        & 2        & 2        & 2       & 2      & 6          & 6          & 2      & 2       \\
		ba18    & 4     & 4      & 6       & 6        & 5        & 4        & 4       & 4      & 9          & 11         & 4      & 4       \\
		ba20    & 4     & 4      & 7       & 5        & 6        & 4        & 4       & 5      & 10         & 11         & 4      & 4       \\
		ba100   &       & 17     & 33      & 31       & 17       & 17       & 17      & 18     & 54         & 61         & 15     & 15      \\
		ba200   &       & 32     & 59      & 56       & 31       & 35       & 31      & 31     & 102        & 113        & 29     & 29      \\
		ba1000  &       & 142    & 284     & 290      & 147      & 151      & 147     & 146    & 515        & 558        &        &         \\
		ba2000  &       & 276    & 570     & 577      & 273      & 284      & 273     & 274    & 1005       & 1091       &        &         \\
		ba10k   &       & 1393   & 2901    & 2974     & 1391     & 1465     & 1391    & 1390   & 5094       & 5593       &        &         \\
		ba20k   &       & 2828   & 5749    & 5896     & 2828     & 2942     & 2828    & 2817   &            & 11093      &        &         \\
		ba100k  &       & 13947  & 29378   & 29375    & 13947    & 14656    & 13947   & 13928  &            &            &        &         \\
		rnd10   & 2     & 2      & 2       & 2        & 2        & 2        & 2       & 2      & 7          &            & 2      & 2       \\
		rnd15   & 3     & 4      & 3       & 4        & 3        & 3        & 3       & 3      & 10         &            & 3      & 3       \\
		rnd20   & 4     & 4      & 5       & 5        & 4        & 5        & 4       & 4      & 15         &            & 4      & 4       \\
		rnd100  &       & 21     & 28      & 24       & 22       & 22       & 19      & 19     & 62         &            & 18     & 18      \\
		rnd200  &       & 41     & 52      & 53       & 40       & 39       & 40      & 39     & 126        &            &        &         \\
		rnd1000 &       & 201    & 284     & 281      & 202      & 210      & 200     & 199    & 633        &            &        &         \\
		rnd2000 &       & 411    & 541     & 552      & 414      & 411      & 404     & 399    & 1272       &            &        &         \\
		rnd10k  &       & 2083   & 2775    & 2760     & 2122     & 2098     & 2035    & 2024   & 6325       &            &        &         \\
		rnd20k  &       & 4175   & 5560    & 5542     & 4279     & 4193     & 4091    & 4053   &            &            &        &         \\
		zly10   &       & 10     & 35      & 35       & 10       & 10       & 10      & 10     & 10         & 10         & 10     & 10      \\
		zly20   &       & 20     & 120     & 120      & 20       & 20       & 20      & 20     & 20         & 20         & 20     & 20      \\
		zly100  &       & 100    & 2600    & 2600     & 100      & 100      & 100     & 100    & 100        & 100        &        &         \\ \hline
	\end{tabular}
	\caption{Veľkosti nájdených dominujúcich množín pre daný graf a algoritmus.}
	\label{table:size}
\end{table}

\end{landscape}

\begin{landscape}
\begin{table}[h]
	\centering
	\begin{tabular}{llllllllllll}
		\hline
		& naive & greedy & greedyQ & ch7alg33 & greedysr & greedysw & floweru & flower & ch7alg34OT & fnaive  & fproper \\ \hline
		ba10    & 0,010 & 0      & 0       & 0        & 0        & 0        & 0       & 0,001  & 0,001      & 0,001   & 0,001   \\
		ba18    & 0,331 & 0      & 0       & 0        & 0        & 0        & 0       & 0,001  & 0,001      & 0,001   & 0,001   \\
		ba20    & 1,321 & 0      & 0       & 0        & 0        & 0        & 0       & 0,001  & 0,001      & 0,001   & 0,001   \\
		ba100   &       & 0,003  & 0       & 0,002    & 0,001    & 0,003    & 0,001   & 0,004  & 0,016      & 0,084   & 0,145   \\
		ba200   &       & 0,004  & 0       & 0,003    & 0,001    & 0,006    & 0,001   & 0,007  & 0,043      & 15,595  & 15,577  \\
		ba1000  &       & 0,021  & 0,003   & 0,013    & 0,011    & 0,040    & 0,011   & 0,046  & 0,358      &         &         \\
		ba2000  &       & 0,047  & 0,008   & 0,025    & 0,024    & 0,740    & 0,025   & 0,090  & 1,322      &         &         \\
		ba10k   &       & 0,308  & 0,079   & 0,115    & 0,251    & 0,933    & 0,249   & 0,952  & 21,37      &         &         \\
		ba20k   &       & 0,987  & 0,290   & 0,410    & 0,837    & 3,097    & 0,873   & 3,368  &            &         &         \\
		ba100k  &       & 19,602 & 6,831   & 19,201   & 21,315   & 65,587   & 20,743  & 60,146 &            &         &         \\
		rnd10   & 0,011 & 0      & 0       & 0        & 0        & 0        & 0       & 0      & 0          & 0       & 0       \\
		rnd15   & 0,058 & 0      & 0       & 0        & 0        & 0        & 0       & 0      & 0          & 0       & 0       \\
		rnd20   & 1,445 & 0      & 0       & 0        & 0        & 0        & 0       & 0      & 0          & 0,001   & 0,001   \\
		rnd100  &       & 0,003  & 0       & 0        & 0        & 0        & 0       & 0,001  & 0,002      & 634,429 & 631,217 \\
		rnd200  &       & 0,004  & 0       & 0,001    & 0        & 0,001    & 0       & 0,002  & 0,008      &         &         \\
		rnd1000 &       & 0,019  & 0,001   & 0,007    & 0,004    & 0,015    & 0,004   & 0,016  & 0,105      &         &         \\
		rnd2000 &       & 0,045  & 0,004   & 0,020    & 0,013    & 0,048    & 0,013   & 0,048  & 0,303      &         &         \\
		rnd10k  &       & 0,384  & 0,101   & 0,120    & 0,294    & 0,984    & 0,313   & 0,929  & 2,827      &         &         \\
		rnd20k  &       & 1,337  & 0,390   & 0,478    & 1,224    & 3,833    & 1,260   & 3,703  &            &         &         \\
		zly10   &       & 0      & 0       & 0        & 0        & 0        & 0       & 0      & 0          & 0       & 0       \\
		zly20   &       & 0      & 0       & 0        & 0        & 0        & 0       & 0,001  & 0,001      & 0,009   & 0,006   \\
		zly100  &       & 0,027  & 0,016   & 0,037    & 0,003    & 0,100    & 0,011   & 0,088  & 0,662      &         &         \\ \hline
	\end{tabular}
	\caption{Výsledky behov algoritmov v jazyku \Java\ na novšom počítači. Výsledky sú uvedené v sekundách.}
	\label{table:java}
\end{table}
\end{landscape}
