\chapter{Dosiahnuté výsledky}\label{chap:vysledky}

V predošlých kapitolách sme spravili prehľad existujúcich algoritmov, popísali 
sme ich implementáciu. V tejto kapitole uvádzame porovnanie jednotlivých 
implementácií. Najprv uvedieme jednotlivé obmedzenia implementácie, potom 
popíšeme spôsob testovania a napokon označenie pre jednotlivé testovacie dáta.

\section{Obmedzenia implementácie}

Vývoj softvéru sa začal na staršom počítači, ktorý má procesor Intel Core 2 
Duo P7350 (2,0GHz dvojjadrový) a 4GB RAM pamäte. Pamäť nebola obmedzením. 
Bohužiaľ počítač 
nebol vhodný na viacvláknové algoritmy, čo spôsobovalo problémy pre algoritmy 
\alg{ch7alg34} a \alg{ch7alg35}. Preto sme sa zamerali na ich jednovláknové 
verzie (\alg{ch7alg34OT}, respektíve \alg{ch7alg35OT}). Heuristiky na pažravý 
algoritmus sa implementovali neskôr, preto sa neuvádzajú v starších testoch.

Počas vývoja začal byť dostupný novší počítač, na ktorom sa testovali už všetky 
heuristiky pre pažravý algoritmus. Počítač má procesor Intel Core i5-4690K CPU 
(3,50GHz štvorjadrový) a 16GB pamäte. Tento počítač zvláda viacvláknové 
aplikácie lepšie (o 6 rokov modernejšia architektúra), no 
vzhľadom na zachovanie kompaktnosti sme opäť viacvláknové verzie algoritmov z 
testov vynechali.

Neskoršia práca s externými knižnicami HPPC časy zrýchlila, čo bohužiaľ nie je 
zachované v žiadnej z uvedených tabuliek.

\section{\Java\ verzus \cpp}

Zo začiatku implementácie nebolo jasné, ktorý programovací jazyk bude použitý. 
Ako prvý sa zvolil jazyk \Java. Keďže prevláda povedomie, že interpretovaný 
programovací jazyk (\Java) nemôže byť rýchlejší ako kompilovaný programovací 
jazyk (\cpp), tak sme sa v priebehu implementácie rozhodli, že skúsime 
naprogramovať niektoré algoritmy aj v jazyku \cpp. Ako vidno z porovnania časov 
behov jednotlivých algoritmov v tabuľkách \ref{table:cpp} a 
\ref{table:java-stare}, jazyk \cpp\ neposkytuje žiadne 
výrazné zrýchlenie. Spolu s faktom, že na porovnanie jednotlivých algoritmov až 
tak rýchlosť netreba, sme sa rozhodli, že budeme pokračovať vo vývoji v jazyku 
\Java.

\section{Spôsob testovania}

Keďže programovací jazyk \Java\ robí optimalizácie kódu počas behu a triedy 
inicializuje "lenivo", tak je potrebné algoritmus spustiť viackrát, aby ukázal 
relevantné výsledky. Algoritmus teda necháme spustiť trikrát a zoberieme tretí 
výsledok. Rozhodnutie je vecou pozorovania. Je kompromisom medzi relevantnými 
výsledkami a dĺžkou skúšania rôznych behov. Z časov nad 20 sekúnd je v 
tabuľkách uvedený iba čas prvého behu algoritmu. 

Rôzne algoritmy sme testovali na rôznych dátach. Najmä podľa toho, aké veľké 
dáta vedeli spracovať v rozumnom čase. Testovacie dáta uvádzame v ďalšej časti.

\section{Testovacie dáta}\label{sec:data}

Testovacie dáta boli grafy rôznych veľkostí a štruktúr. Naše testovacie dáta sa 
dajú rozdeliť do štyroch hlavných skupín. Podľa skupiny majú aj predponu. Sú to 
tieto:

\begin{itemize}
	\item grafy tvorené prioritným pripájaním tvoriace Barabási-Albertov model 
		-- majú predponu \texttt{ba-};
	\item grafy tvorené hranami s náhodnou pravdepodobnosťou vzniku tvoriace 
	náhodný model -- majú predponu \texttt{rnd-};
	\item grafy vytvorené špeciálne podľa obrázka \ref{img:zle1} -- majú 
		predponu \texttt{zle-};
	\item grafy vytvorené z reálnych dát -- majú predponu \texttt{ca-}.
\end{itemize}

Grafové modely sú taktiež popísané aj v časti \ref{sec:modely}. Grafy tvorené 
prioritným pripájaním (podľa Barabási-Albertového modelu) boli 
vytvorené pomocou programu Network Workbench. Ku grafu sa pridali v každom 
kroku tri hrany.
Grafy tvorené náhodnou pravdepodobnosťou vzniku hrán boli taktiež vytvorené 
pomocou programu Network Workbench tak, že hrany sa tvorili s určitou 
pravdepodobnosťou. Táto pravdepodobnosť bola nastavená tak, aby mali grafy 
tvorené náhodne veľmi podobný priemerný stupeň (okolo 6) a teda aj hustotu ako 
grafy vytvorené podľa Barabási-Albertového modelu rovnakej veľkosti. 

Reálne dáta boli prevzaté zo stránky \url{http://snap.stanford.edu/data/} 
\citep{snapnets}. Dáta predstavujú citácie prác ohľadne nejakého vedného oboru. 
Všetky dáta sú spracované zo stránky \url{http://arxiv.org/}. Jednotlivé dáta 
predstavujú spoluprácu v daných oboroch:
\begin{enumerate}
	\item dáta \texttt{ca1} zachytávajú citácie v oblasti všeobecnej teórie relativity;
	\item dáta \texttt{ca2} -- jadrová a subjadrová fyzika;
	\item dáta \texttt{ca22} -- jadrová a subjadrová fyzika, rozšírené;
	\item dáta \texttt{ca3} -- fyzika tuhých látok.
\end{enumerate}

Podľa počtu vrcholov majú názvy jednotlivých grafov príponu. Pre grafy 
vytvorené prioritným pripájaním a náhodné grafy, číslo za predponou vyjadruje 
počet vrcholov grafu. Pre grafy vytvorené podľa obrázka \ref{img:zle1} číslo 
vyjadruje počet vrcholov na "hlavnej" ceste. Pre grafy reálnych dát boli čísla 
zvolené umelo a nemajú žiaden význam. Všetky použité reálne dáta boli siete 
citácií vedeckých publikácií.

\section{Porovnanie algoritmov}

V tejto časti porovnáme medzi sebou jednotlivé skupiny algoritmov. Začneme 
algoritmami dávajúcimi presné výsledky, budeme pokračovať všeobecným porovnaním 
a na koniec zhodnotíme jednotlivé heuristiky pažravého algoritmu.

\subsection{Presné algoritmy}

Do tejto kategórie spadajú algoritmy \alg{naive}, \alg{fnaive} a \alg{fproper}. 
Boli implementované najmä kvôli prehľadu a získaniu veľkostí minimálnych 
dominujúcich množín. Ako vidno zo všetkých tabuliek (\ref{table:java-stare}, 
\ref{table:cpp} a aj \ref{table:java}), pri exaktných algoritmoch nemôžeme 
rátať s tým, že sa dočkáme výsledkov na reálnych dátach.

Môžeme však vidieť, že použitím vhodných úprav vieme posunúť veľkosť grafu, 
kedy vieme vypočítať minimálnu dominujúcu množinu v rozumnom čase, zo zhruba 
20 vrcholov na zhruba 100 vrcholov. Taktiež je dobré všimnúť si, že aj keď 
algoritmus \alg{fproper} používa oveľa zložitejšie výpočty, tak čas behu 
algoritmu to oveľa nezrýchli. Prevedenie na hľadanie maximálneho párenia 
(respektíve minimálneho hranového pokrytia) je teda viac zlepšením teoretickej 
zložitosti ako pomoc reálnemu behu algoritmu.

Celkom zaujímavým je pozorovanie, že na neštruktúrovaných dátach (\texttt{rnd-} 
grafy) mali algoritmy \alg{fnaive} a \alg{fproper} oveľa horší čas ako na 
dátach so štruktúrou (\texttt{ba-} a \texttt{zle-} grafy).

\subsection{Všeobecné porovnanie}

Skôr ako sa pozrieme na celkové hodnotenie, tak sa môžeme pozrieť na algoritmy 
\alg{ch7alg34OT} a \alg{ch7alg35OT}. V tabuľke \ref{table:java-stare} je vidno, 
žo algoritmus \alg{ch7alg35OT} naozaj rieši uvedené problémy v časti 
\ref{sec:dist}. Bohužiaľ, nepodarilo sa vytvoriť prostredie pre distribuovaný 
algoritmus.

Celkovo si môžeme všimnúť (tabuľka \ref{table:java}), že algoritmy bežia 
rýchlejšie na štruktúrovaných dátach (\texttt{ba-} a \alg{zle-}) okrem 
algoritmu \alg{ch7alg33OT}, ktorý beží lepšie na neštruktúrovaných dátach 
(\texttt{rnd-}). 

Keďže štruktúrované aj neštruktúrované dáta majú pomerne rovnakú hustotu hrán, 
tak sme neporovnávali algoritmy na rôzne hustých grafoch.

\subsection{Porovnanie heuristík pažravých algoritmov}

Zatiaľ sme sa pozerali čisto iba na časy. No pri porovnávaní reálnych časov 
pažravých algoritmov nám o rýchlosti veľa prezradí aj veľkosť výslednej 
množiny. Veľkosti výslednej množiny sú uvedené v tabuľke \ref{table:size}. 
Za príklad si môžeme zobrať trebárs základný algoritmus \alg{greedy}. Tam, 
kde mal menšiu výslednú množinu, tam mal aj nižší čas. Napríklad, pre grafy 
\texttt{ba20k} a \texttt{rnd20k} bol rozdiel vo veľkosti nájdených množín 1347. 
To znamená, že sa základný cyklus musel zopakovať oveľa viac krát, čo vyústilo 
do horšieho času algoritmu. Keďže všetky verzie pažravého algoritmu až na 
verziu \alg{flower} mali na začiatku behu algoritmu (pred opakujúcim sa cyklom) 
takmer nulové predpočítavanie rôznych hodnôt, respektíve vyberanie vrcholov do 
výsledku, tak je tento trend badať na všetkých algoritmoch. 

Algoritmus \alg{flower} je v tomto smere výnimkou. Keďže si predpočítava 
susednosť do vzdialenosti tri a grafy tvorené Barabási-Albertovým modelom sú 
viac spojité ako náhodné grafy, tak sa počet opakovaní následného cyklu prejaví 
až pri grafoch s veľkým počtom vrcholov (v našom prípade \texttt{-20k}).

Keďže v Barabási-Albertovom modely sa každý nový vrchol spojí s práve (v našom 
prípade) dvoma inými, tak nevznikajú výhonky. Taktiež nie je šanca ani na 
vytvorenie kvetov. Takže veľa heuristík založených na odstraňovaní výhonkov a 
kvetov nič neodstráni a teda výsledkom je taká istá množina ako pri základnom 
pažravom algoritme.

\subsection{Porovnanie pažravých algoritmov na reálnych dátach}

V reálnych dátach je situácia iná ako v modelových. V reálnych dátach sa tvoria 
malé lokálne klastre, ktoré sú občas hustejšie prepojené a vytvoria kvet. 
Taktiež je šanca, že nejaký vrchol bude izolovaný alebo takmer izolovaný. Teda 
má zmysel vytvárať heuristiky na základe vyberania výhonkov a kvetov. Keďže 
reálne siete mali nad 5000 vrcholov, iné ako pažravé algoritmy nebolo možné 
testovať. Výsledky testovania sú uvedené v tabuľke \ref{table:real}. 

Algoritmy \alg{ch7alg33} a \alg{greedyq} boli navrhnuté s tým, že budú bežať 
rýchlejšie, ale aj oveľa nepresnejšie, čo sa potvrdilo. Bohužiaľ miera 
nepresnosti na reálnych dátach je príliš veľká na akékoľvek použitie.

Najjednoduchší pažravý algoritmus (\alg{greedy}) dáva 
prekvapivo veľmi dobré výsledky. Je prekonaný iba algoritmom \alg{greedysw}, 
ktorý je rozšírením jednoduchého pažravého algoritmu iba o vhodný spôsob 
výberu vrchola. Najzložitejší, algoritmus \alg{flower}, síce našiel najmenšiu 
dominujúcu množinu spomedzi testovaných algoritmov na dvoch grafoch 
(\texttt{ca1} a \texttt{ca22}), ale mal horšie časy. 

Predpokladom bolo, že 
tento algoritmus bude nachádzať najmenšie dominujúce množiny, pretože má 
najsofistikovanejšie vnútorné rozhodovanie spomedzi všetkých implementovaných 
pažravých algoritmov. Bohužiaľ, niektoré dáta majú štruktúru, pri ktorej sa 
neoplatí najprv vyberať výhonky. Predpokladom je, že v týchto dátach sú 
výhonky napojené na väčšie klastre alebo kvety a pri vybratí výhonku sa odsunie 
vybratie oveľa výhodnejšieho vrchola na neskôr.

Na dátach s vyšším priemerným stupňom (\texttt{ca2} a \texttt{ca3}) 
algoritmus \alg{flower} preukázal horšie výsledky ako jednoduchý pažravý 
algoritmus (\alg{flower}). Vyšší priemerný stupeň mení vnútornú štruktúru 
grafu, vrcholy sú viac prepojené a je možné, že určovanie kvetov, v hustejšie 
prepojených štruktúrach nie je až tak účinné.

Väčšia hustota grafov \texttt{ca2} a \texttt{ca3} sa prejavila aj v pomalšom 
behu algoritmu \alg{greedysw}, keďže tento robí po každom kroku viac prepočtov 
dátových štruktúr. Tento algoritmus nepredpočítava kvety, čo vytvára vzájomnú 
odlišnosť s algoritmom \alg{flower}. Z výsledkov vidno, že na niektorých dátach 
sa oplatí kvety predrátať, na niektorých nie.

\section{Tabuľky}

V tejto časti sa nachádzajú tabuľky zobrazujúce výsledky z testovania 
algoritmov. Ich obsah je podrobnejšie rozobratá v predošlých častiach tejto 
kapitoly. Sú tu uvedené tabuľky rýchlosti behov na staršom počítači v jazyku 
\cpp\ a \Java\ (tabuľky \ref{table:cpp} a \ref{table:java-stare}), výsledky 
rýchlosti behov na novšom počítači v jazyku \Java\ (tabuľka \ref{table:java}). 
V tabuľke \ref{table:size} sú uvedené veľkosti výsledných nájdených množín.
Taktiež sú tu uvedené výsledky testovania algoritmov na reálnych dátach 
(tabuľka \ref{table:real}) a štatistické údaje o dátach (tabuľka 
\ref{table:stats}).

\begin{table}[h]
	\centering
	\caption{Výsledky behov algoritmov v jazyku \cpp\ na staršom počítači.}
	\begin{tabular}{l|rrrr}
		\hline
		& naive & greedy & ch7alg33 & ch7alg34OT \\ \hline
		ba10    & 0     & 0      & 0        & 0          \\
		ba18    & 1     & 0      & 0        & 0          \\
		ba20    & 4,37  & 0      & 0        & 0          \\
		ba100   &       & 0      & 0        & 0          \\
		ba200   &       & 0      & 0,01     & 0,06       \\
		ba1000  &       & 0      & 0        & 1,28       \\
		ba2000  &       & 0,01   & 0,02     & 5,15       \\
		ba10k   &       & 0,08   & 0,51     & 120,32     \\
		ba20k   &       & 0,29   & 2,68     &            \\
		ba100k  &       & 6,52   & 147,56   &            \\
		rnd100  &       & 0      & 0        & 0,03       \\
		rnd200  &       & 0      & 0        & 0,04       \\
		rnd1000 &       & 0      & 0        & 0,21       \\
		rnd10k  &       & 0,07   & 0,53     & 5,6        \\
		zly10   &       & 0      & 0        & 0          \\
		zly20   &       & 0      & 0        & 0,01       \\
		zly100  &       & 0,02   & 0,22     & 4,72       \\ \hline
	\end{tabular}
	\bigskip\par
	V prvom stĺpci je označenie dát (štatistické vlastnosti dát sú v tabuľke 
	\ref{table:stats}) a ku príslušným dátam sú uvedené časy behov pre daný 
	algoritmus (algoritmy sú popísané v kapitole \ref{chap:algoritmy}) v 
	sekundách. Keď si porovnáme výsledné časy s tabuľkou 
	\ref{table:java-stare} zistíme, že prepísanie zdrojového kódu do jazyka 
	\cpp\ neprinieslo výrazné zlepšenie.
	\label{table:cpp}
\end{table}

\begin{table}[h]
	\centering
	\caption{Testovanie rôznych heuristík na dátach z reálneho sveta.}
	\begin{tabular}{l|rrrrrrrr}
		\hline
		         & ca1 - |S| & ca1 - čas & ca2 - |S| & ca2 - čas & ca22 - |S| & ca22 - čas & ca3 - |S| & ca3 - čas \\ \hline
		greedy   & 1176      & 0,124     & 2009      & 0,578     & 2275       & 0,330      & 3694      & 1,311     \\
		ch7alg33 & 1570      & 0,070     & 3243      & 0,121     & 3230       & 0,099      & 5651      & 0,392     \\
		greedyQ  & 1586      & 0,034     & 3239      & 0,087     & 3243       & 0,080      & 5625      & 0,319     \\
		greedysr & 1803      & 0,072     & 3100      & 0,340     & 3667       & 0,118      & 5553      & 0,708     \\
		greedysw & 1159      & 0,348     & 1989      & 1,613     & 2258       & 1,022      & 3649      & 3,548     \\
		floweru  & 1329      & 0,111     & 2131      & 0,511     & 2501       & 0,313      & 3920      & 1,196     \\
		flower   & 1157      & 0,293     & 2018      & 2,406     & 2256       & 0,591      & 3774      & 2,830     \\ \hline
	\end{tabular}
	\bigskip\par
	V prvom stĺpci sú vymenované algoritmy (popísané v kapitole 
	\ref{chap:algoritmy}). Pre daný algoritmus je v ďalších stĺpcoch uvedená 
	veľkosť výslednej nájdenej dominujúcej množiny (stĺpec |S|) a vedľa nej 
	čas, za ktorý algoritmus množinu našiel. Čas je uvedený v sekundách. 
	Štatistické vlastnosti dát sú v tabuľke \ref{table:stats}.
	\label{table:real}
\end{table}

\begin{table}[h]
	\centering
	\caption{Výsledky behov algoritmov v jazyku \Java\ na staršom počítači.}
	\begin{tabular}{l|rrrrrrrr}
		\hline
		& naive & greedy & greedyQ & ch7alg33 & ch7alg34OT & ch7alg35OT & fnaive   & fproper \\ \hline
		ba10    & 0,085 & 0,001  & 0,002   & 0        & 0,003      & 0,004      & 0,008    & 0,009   \\
		ba18    & 1,284 & 0,001  & 0,002   & 0        & 0,013      & 0,008      & 0,035    & 0,035   \\
		ba20    & 4,440 & 0,001  & 0,002   & 0        & 0,011      & 0,008      & 0,049    & 0,047   \\
		ba100   &       & 0,015  & 0,004   & 0,004    & 0,070      & 0,045      & 0,473    & 0,506   \\
		ba200   &       & 0,026  & 0,006   & 0,009    & 0,135      & 0,073      & 30,515   & 30,412  \\
		ba1000  &       & 0,087  & 0,045   & 0,037    & 0,883      & 0,512      &          &         \\
		ba2000  &       & 0,142  & 0,095   & 0,053    & 2,524      & 0,988      &          &         \\
		ba10k   &       & 1,844  & 0,362   & 0,540    & 45,781     & 6,662      &          &         \\
		ba20k   &       & 8,813  & 1,294   & 3,800    &            & 22,256     &          &         \\
		ba100k  &       & 66,762 & 68,669  & 135,197  &            &            &          &         \\
		rnd10   &       & 0,001  & 0,002   &          &            &            & 0,003    & 0,003   \\
		rnd15   &       & 0,001  & 0,002   &          &            &            & 0,048    & 0,043   \\
		rnd20   &       & 0,001  & 0,002   &          &            &            & 0,107    & 0,112   \\
		rnd100  &       & 0,011  & 0,004   &          &            &            & 1293,811 & 1302,940\\
		rnd200  &       & 0,032  & 0,008   &          &            &            &          &         \\
		rnd1000 &       & 0,072  & 0,043   &          &            &            &          &         \\
		rnd2000 &       & 0,114  & 0,096   &          &            &            &          &         \\
		rnd10k  &       & 2,484  & 0,440   &          &            &            &          &         \\
		rnd20k  &       & 11,281 & 1,671   & 4,947    &            &            &          &         \\
		zly10   &       & 0,005  & 0,003   & 0,003    & 0,040      & 0,018      & 0,119    & 0,117   \\
		zly20   &       & 0,022  & 0,010   & 0,022    & 0,098      & 0,036      & 0,809    & 0,814   \\
		zly100  &       & 0,090  & 0,151   & 0,210    & 6,213      & 2,107      &          &         \\ \hline
	\end{tabular}
	\smallskip \par
	V prvom stĺpci je označenie dát (štatistické vlastnosti dát sú v tabuľke 
	\ref{table:stats}) a ku príslušným dátam sú uvedené časy behov pre daný 
	algoritmus (algoritmy sú popísané v kapitole \ref{chap:algoritmy}) v 
	sekundách.
	\label{table:java-stare}
\end{table}

\begin{landscape}

\begin{table}[h]
	\centering
	\caption{Veľkosti nájdených dominujúcich množín pre daný graf a algoritmus.}
\scalebox{0.9}{
		\begin{tabular}{l|rrrrrrrrrrrr}
		\hline
		& naive & greedy & greedyQ & ch7alg33 & greedysr & greedysw & floweru & flower & ch7alg34OT & ch7alg35OT & fnaive & fproper \\ \hline
		ba10    & 2     & 2      & 2       & 2        & 2        & 2        & 2       & 2      & 6          & 6          & 2      & 2       \\
		ba18    & 4     & 4      & 6       & 6        & 5        & 4        & 4       & 4      & 9          & 11         & 4      & 4       \\
		ba20    & 4     & 4      & 7       & 5        & 6        & 4        & 4       & 5      & 10         & 11         & 4      & 4       \\
		ba100   &       & 17     & 33      & 31       & 17       & 17       & 17      & 18     & 54         & 61         & 15     & 15      \\
		ba200   &       & 32     & 59      & 56       & 31       & 35       & 31      & 31     & 102        & 113        & 29     & 29      \\
		ba1000  &       & 142    & 284     & 290      & 147      & 151      & 147     & 146    & 515        & 558        &        &         \\
		ba2000  &       & 276    & 570     & 577      & 273      & 284      & 273     & 274    & 1005       & 1091       &        &         \\
		ba10k   &       & 1393   & 2901    & 2974     & 1391     & 1465     & 1391    & 1390   & 5094       & 5593       &        &         \\
		ba20k   &       & 2828   & 5749    & 5896     & 2828     & 2942     & 2828    & 2817   &            & 11093      &        &         \\
		ba100k  &       & 13947  & 29378   & 29375    & 13947    & 14656    & 13947   & 13928  &            &            &        &         \\
		rnd10   & 2     & 2      & 2       & 2        & 2        & 2        & 2       & 2      & 7          &            & 2      & 2       \\
		rnd15   & 3     & 4      & 3       & 4        & 3        & 3        & 3       & 3      & 10         &            & 3      & 3       \\
		rnd20   & 4     & 4      & 5       & 5        & 4        & 5        & 4       & 4      & 15         &            & 4      & 4       \\
		rnd100  &       & 21     & 28      & 24       & 22       & 22       & 19      & 19     & 62         &            & 18     & 18      \\
		rnd200  &       & 41     & 52      & 53       & 40       & 39       & 40      & 39     & 126        &            &        &         \\
		rnd1000 &       & 201    & 284     & 281      & 202      & 210      & 200     & 199    & 633        &            &        &         \\
		rnd2000 &       & 411    & 541     & 552      & 414      & 411      & 404     & 399    & 1272       &            &        &         \\
		rnd10k  &       & 2083   & 2775    & 2760     & 2122     & 2098     & 2035    & 2024   & 6325       &            &        &         \\
		rnd20k  &       & 4175   & 5560    & 5542     & 4279     & 4193     & 4091    & 4053   &            &            &        &         \\
		zly10   &       & 10     & 35      & 35       & 10       & 10       & 10      & 10     & 10         & 10         & 10     & 10      \\
		zly20   &       & 20     & 120     & 120      & 20       & 20       & 20      & 20     & 20         & 20         & 20     & 20      \\
		zly100  &       & 100    & 2600    & 2600     & 100      & 100      & 100     & 100    & 100        & 100        &        &         \\ \hline
	\end{tabular}
}
	\smallskip \par
	{\footnotesize 
	V prvom stĺpci je označenie dát (štatistické vlastnosti dát sú v tabuľke 
	\ref{table:stats}) a ku príslušným dátam sú uvedené veľkosti nájdených 
	dominujúcich množín pre daný algoritmus (algoritmy sú popísané v kapitole 
	\ref{chap:algoritmy}). Za povšimnutie stojí, že 
	heuristiky určené na siete malého sveta (\alg{greedysw}, \alg{flower}) 
	nenašli typické črty pri grafoch tvorených Barabási-Albertovým modelom 
	(a teda nepriniesli zmenšenie výslednej množiny), 	avšak heuristiky sa 
	prejavili pri reálnych dátach, kde rôzne zmenšili výslednú množinu 
	(tabuľka \ref{table:real}).}
	\label{table:size}
\end{table}

\end{landscape}

\begin{landscape}
\begin{table}[h]
	\centering
	\caption{Výsledky behov algoritmov v jazyku \Java\ na novšom počítači.}
\scalebox{0.9}{
	\begin{tabular}{l|rrrrrrrrrrr}
		\hline
		& naive & greedy & greedyQ & ch7alg33 & greedysr & greedysw & floweru & flower & ch7alg34OT & fnaive  & fproper \\ \hline
		ba10    & 0,010 & 0      & 0       & 0        & 0        & 0        & 0       & 0,001  & 0,001      & 0,001   & 0,001   \\
		ba18    & 0,331 & 0      & 0       & 0        & 0        & 0        & 0       & 0,001  & 0,001      & 0,001   & 0,001   \\
		ba20    & 1,321 & 0      & 0       & 0        & 0        & 0        & 0       & 0,001  & 0,001      & 0,001   & 0,001   \\
		ba100   &       & 0,003  & 0       & 0,002    & 0,001    & 0,003    & 0,001   & 0,004  & 0,016      & 0,084   & 0,145   \\
		ba200   &       & 0,004  & 0       & 0,003    & 0,001    & 0,006    & 0,001   & 0,007  & 0,043      & 15,595  & 15,577  \\
		ba1000  &       & 0,021  & 0,003   & 0,013    & 0,011    & 0,040    & 0,011   & 0,046  & 0,358      &         &         \\
		ba2000  &       & 0,047  & 0,008   & 0,025    & 0,024    & 0,740    & 0,025   & 0,090  & 1,322      &         &         \\
		ba10k   &       & 0,308  & 0,079   & 0,115    & 0,251    & 0,933    & 0,249   & 0,952  & 21,37      &         &         \\
		ba20k   &       & 0,987  & 0,290   & 0,410    & 0,837    & 3,097    & 0,873   & 3,368  &            &         &         \\
		ba100k  &       & 19,602 & 6,831   & 19,201   & 21,315   & 65,587   & 20,743  & 60,146 &            &         &         \\
		rnd10   & 0,011 & 0      & 0       & 0        & 0        & 0        & 0       & 0      & 0          & 0       & 0       \\
		rnd15   & 0,058 & 0      & 0       & 0        & 0        & 0        & 0       & 0      & 0          & 0       & 0       \\
		rnd20   & 1,445 & 0      & 0       & 0        & 0        & 0        & 0       & 0      & 0          & 0,001   & 0,001   \\
		rnd100  &       & 0,003  & 0       & 0        & 0        & 0        & 0       & 0,001  & 0,002      & 634,429 & 631,217 \\
		rnd200  &       & 0,004  & 0       & 0,001    & 0        & 0,001    & 0       & 0,002  & 0,008      &         &         \\
		rnd1000 &       & 0,019  & 0,001   & 0,007    & 0,004    & 0,015    & 0,004   & 0,016  & 0,105      &         &         \\
		rnd2000 &       & 0,045  & 0,004   & 0,020    & 0,013    & 0,048    & 0,013   & 0,048  & 0,303      &         &         \\
		rnd10k  &       & 0,384  & 0,101   & 0,120    & 0,294    & 0,984    & 0,313   & 0,929  & 2,827      &         &         \\
		rnd20k  &       & 1,337  & 0,390   & 0,478    & 1,224    & 3,833    & 1,260   & 3,703  &            &         &         \\
		zly10   &       & 0      & 0       & 0        & 0        & 0        & 0       & 0      & 0          & 0       & 0       \\
		zly20   &       & 0      & 0       & 0        & 0        & 0        & 0       & 0,001  & 0,001      & 0,009   & 0,006   \\
		zly100  &       & 0,027  & 0,016   & 0,037    & 0,003    & 0,100    & 0,011   & 0,088  & 0,662      &         &         \\ \hline
	\end{tabular}
}
	\smallskip \par
	{\footnotesize
	V prvom stĺpci je označenie dát (štatistické vlastnosti dát sú v tabuľke 
	\ref{table:stats}) a ku príslušným dátam sú uvedené veľkosti nájdených 
	dominujúcich množín pre daný algoritmus (algoritmy sú popísané v kapitole 
	\ref{chap:algoritmy}). Oproti tabuľke \ref{table:java-stare} vidno 
	zrýchlenie vďaka výkonnejšiemu počítaču. V kombinácií s veľkosťou nájdenej 
	výslednej množiny (tabuľka \ref{table:size}) si môžeme všimnúť, ako čas 
	pažravých algoritmov závisí od veľkosti nájdenej množiny (algoritmy 
	\alg{greedy-} a \alg{flower-} a porovnanie \texttt{ba-} oproti \texttt{rnd-} 
	dátam rovnakej veľkosti).}
	\label{table:java}
\end{table}
\end{landscape}

\begin{landscape}
	\centering
\begin{table}[h]
	\centering
	\caption{Štatistické údaje dát.}
\scalebox{0.9}{
		\begin{tabular}{l|rrrrrrr}
		\hline
		& počet vrcholov & počet hrán & priemerný stupeň & hustota & priemerný klasterizačný koeficient & priemer & priemerná najkratšia cesta \\ \hline
		ba10    & 10             & 20         & 4                & 0,444   & 0,673                              & 3       & 1,667                      \\
		ba18    & 18             & 41         & 4,556            & 0,268   & 0,421                              & 4       & 1,948                      \\
		ba20    & 20             & 46         & 4,6              & 0,242   & 0,412                              & 4       & 2,016                      \\
		ba100   & 100            & 279        & 5,58             & 0,056   & 0,167                              & 5       & 2,622                      \\
		ba200   & 200            & 574        & 5,74             & 0,029   & 0,088                              & 5       & 2,909                      \\
		ba1000  & 1000           & 2964       & 5,928            & 0,005   & 0,028                              & 6       & 3,534                      \\
		ba2000  & 2000           & 5962       & 5,962            & 0,003   & 0,019                              & 6       & 3,766                      \\
		ba10k   & 10000          & 29950      & 5,99             & 0,006   & 0,006                              & 7       & 4,306                      \\
		ba20k   & 20000          & 59943      & 5,994            & 0,0003  & 0,003                              & 7       & 4,53                       \\
		ba100k  & 100000         & 299921     & 5,998            & 0,00006 & 0,0008                             & 8       & 5,039                      \\
		rnd10   & 10             & 30         & 6                & 0,667   & 0,742                              & 3       & 1,356                      \\
		rnd15   & 15             & 45         & 6                & 0,429   & 0,417                              & 3       & 1,59                       \\
		rnd20   & 20             & 60         & 6                & 0,316   & 0,262                              & 3       & 1,732                      \\
		rnd100  & 100            & 301        & 6,02             & 0,061   & 0,053                              & 5       & 2,746                      \\
		rnd200  & 200            & 601        & 6,01             & 0,03    & 0,033                              & 6       & 3,142                      \\
		rnd1000 & 1000           & 3000       & 6                & 0,006   & 0,006                              & 8       & 4,059                      \\
		rnd2000 & 2000           & 6042       & 6,042            & 0,003   & 0,003                              & 8       & 4,421                      \\
		rnd10k  & 10000          & 30000      & 6                & 0,006   & 0,0006                             & 11      & 5,351                      \\
		rnd20k  & 20000          & 60036      & 6,004            & 0,0003  & 0,0003                             & 11      & 5,73                       \\
		zly10   & 76             & 75         &                  &         &                                    &         &                            \\
		zly20   & 251            & 250        &                  &         &                                    &         &                            \\
		zly100  & 5251           & 5250       &                  &         &                                    &         &                            \\
		ca1     & 5242           & 28980      & 11,057           &         &                                    & 17      & 6,047                      \\
		ca2     & 12008          & 237010     & 39,475           &         & 0,696                              & 13      & 2,622                      \\
		ca22    & 9877           & 51971      & 10,524           &         &                                    & 18      & 5,945                      \\
		ca3     & 23133          & 186936     & 16,162           &         & 0,65                               & 15      &                            \\ \hline
	\end{tabular}
	}
	\smallskip\par
	{\footnotesize V tejto tabuľke sú uvedené základné štatistické údaje dát, 
		s ktorými sme pracovali. V prvom stĺpci je názov dát. Predpona 
		\texttt{ba-} patrí grafom podľa Barabási-Albertoveho modelu, predponou 
		\texttt{rnd-} sú označené grafy vytvorené náhodným modelom, predponou 
		\texttt{zly-} sú označené grafy vytvorené podľa obrázka \ref{img:zle1} 
		a predponu \texttt{ca-} majú reálne dáta (vo všetkých prípadoch je to 
		sieť vedeckých publikácií). Dáta sú popísané aj v časti \ref{sec:data}.}
	\label{table:stats}
\end{table}
\end{landscape}
