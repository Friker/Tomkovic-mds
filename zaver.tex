\cleardoublepage
\phantomsection
\addcontentsline{toc}{chapter}{Záver}
\chapter*{Záver}\label{chap:zaver}

V práci sme popísali niektoré algoritmy a porovnali ich implementácie. 
Fomin \citep{fomin05} napísal dobrú prácu... \ldots

\begin{comment}
dátové štruktúry union-find, písmenkový strom a sufixový 
strom (kapitoly~\ref{chap:uf}, \ref{chap:trie} a \ref{chap:sx}). Spísali sme 
návrh a implementáciu softvéru. Softvér sme implementovali 
podľa návrhu a splnili sme všetku funkcionalitu: vizualizovali sme všetky 
dátové štruktúry a algoritmy, doplnili funkčnosť, spravili komentáre v 
angličtine a slovenčine.

V ďalšom vývoji by sme chceli prerobiť komentáre pre sufixové stromy, keďže sa 
nám zdajú byť neprehľadné. Taktiež by sme chceli zvýrazniť niektoré časti 
Ukkonenovho algoritmu a pridať nové algoritmy vymenované v sekcií 
\ref{sec:sx:usage}.

Softvér by sme chceli v budúcnosti rozšíriť o nové dátové štruktúry, ktoré sa 
týkajú stringológie, napríklad \emph{radix tree}, \emph{PATRICIA}, 
\emph{orientovaný acyklický graf pre slová (DAWG)}, \emph{všeobecný sufixový 
strom}.

Chceli by sme projekt podstúpiť širšiemu použitiu a zozbierať viac spätnej 
väzby, aby sme softvér odladili podľa používateľských potrieb a opravili 
chyby, ktoré sme neobjavili. 
\end{comment}


\cleardoublepage
\phantomsection
\addcontentsline{toc}{chapter}{Literatúra}

\printbibliography

\backmatter

\newpage
\pagestyle{empty}
\hbox{}