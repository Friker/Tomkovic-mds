\cleardoublepage
\phantomsection
\addcontentsline{toc}{chapter}{Záver}
\chapter*{Záver}\label{chap:zaver}

V práci sme sa snažili naplniť ciele zadania. A to spraviť prehľad algoritmov 
hľdajúcich minimálne dominujúce množiny a skúsiť vylepšiť niektoré algoritmy, 
ktoré hľadajú minimálne dominujúce množiny na sieťach malého sveta. 

V kapitole \ref{chap:definicie} sme popísali základné pojmy a uviedli čitateľa 
do problematiky grafov, komplexných sietí a hľadania minimálnej dominujúcej 
množiny. Kapitolu \ref{chap:algoritmy} sme začali spísaním prehľadu súčasných 
základných existujúcich algoritmov na hľadanie minimálnej dominujúcej množiny. 
Ďalej sme v tejto kapitole načrtli výhody v prípade, že vstupným grafom je 
komplexná sieť. Nakoniec sme navrhli niekoľko heuristík, o ktorých sme 
predpokladali, že nejako zlepšia čas behu alebo výsledky algoritmov. Keďže 
sme sa v práci chceli zaoberať dátami z reálneho sveta, zvolili sme si 
heuristiky na zlepšenie tých algoritmov, ktoré majú šancu spracovať také veľké 
dáta.

V kapitolách \ref{chap:popis} a \ref{chap:implementacia} sme popísali návrh a 
implementáciu softvéru. Preskúmali sme existujúce riešenia a na základe analýzy 
navrhli, ako by mala implementačná časť práce vyzerať.
Softvér nakonie nenadväzoval na žiadne predošlé práce, aj keď 
používal pomocnú knižnicu (HPPC) na uchovanie základných dátových štruktúr. 
V konečnom dôsledku malo použitie externej knižnice pozitívny dopad. 
Implementačnú časť práce sme sa rozhodli spraviť samostatne a nenadväzovať 
kvôli úzkej špecifikácií algoritmov a pohodlnosti práce s vlastnými 
zložitejšími dátovými a programovými štruktúrami oproti existujúcim riešeniam.

V kapitole \ref{chap:vysledky} sme opísali výsledky z implementácie softvéru a 
porovnali sme algoritmy medzi sebou a na rôznych dátach, včetne reálnych sietí 
citácií. Z navrhovaných heuristík mala každá svoje slabé a silné miesta. 
Čo dáva priestor na ďalšie zlepšovania. 

Veľmi potešujúce však je, že sme našli heuristiku, ktorá na všetkých 
testovaných reálnych dátach prekonala veľkosťou dominujúcej množiny 
jednoduchý pažravý algoritmus (verzia \alg{greedysw}). Iba o trochu viac 
sofistikovaný spôsob vyberania (uprednostňuj vrcholy, ktoré svojím vybratím 
čo najviac zmenšia množinu vrcholov, ktoré majú nejakého bieleho suseda) 
dosiahol zlepšenie. Zlepšenie sa nepodarilo vždy pri algoritme so zložitejšou 
štruktúrou -- verzii \alg{flower}.
Tento výsledok odráža skutočnosť, že často algoritmy s 
jednoduchou štruktúrou bývajú tým najvhodnejším riešením.

V tejto práci sme nevyužili vedomosť, že siete malého sveta sú síce odolné voči 
náhodným útokom, no zraniteľné cielenými útokmi v tom zmysle, že ich vieme 
odobratím malého počtu hrán rozdeliť na viacero komponentov. Je možné, že keby 
sme vedeli, tieto hrany efektívne určiť, mohli by sme skúmať, ako dôležité sú 
vrcholy, ktoré sú incidentné s hranou a na základe toho ich prioritne ne/vybrať 
do výslednej množiny.

Vďaka tomu, že vieme, že siete malého sveta sú riedke grafy, môžeme tieto grafy 
rozdeliť na množinu stromov, na ktorých sa minimálna dominujúca množina hľadá 
ľahšie. Je však otázne, ako vyriešiť vrcholy, v ktorých tieto stromy v pôvodnom 
grafe susedia. Aj keď sa tento prístup zdá byť prijateľný, tak vieme, že siete 
malého sveta majú veľa lokálnych klastrov a pridelení týchto klastrov by 
vznikalo veľa stromov, čo by mohlo viesť ku vybratiu veľa "zbytočných" 
vrcholov. Vďaka tejto komplikácií a tomu, že sa tento spôsob riešenia výrazne 
odlišuje od zvyšných, sme takýto postup hľadania minimálnej dominujúcej množiny 
nakoniec vynechali.

V ďalšej činnosti je teda vhodné zamerať sa podrobnejšie na to, aké heuristiky 
dávajú lepšie výsledky pre rôzne typy dát ako napríklad hustejšie/redšie grafy, 
grafy s väčšou/menšou klasterizáciou alebo trebárs rôzne spôsoby vyberania 
výhonkov. To by umožňovalo vzniku možno lepšej heuristiky (či už z pohľadu času 
alebo veľkosti nájdenej množiny) alebo kombinovaného algoritmu s využitím 
viacerých heuristík (podobne ako naša verzia \alg{flower}). Taktiež je vhodné 
zamerať sa na dobré implementovanie distribuovaných algoritmov. Buď vylepšiť 
algoritmy \alg{ch7alg34} a \alg{ch7alg35} tým, že lepšie využijeme nejaké 
distribuované prostredie alebo skúsiť rozobrať distribuovanú verziu vynechaného 
algoritmu rozdelenia grafu na stromy. Taktiež by v ďalšej práci bolo vhodné 
výsledky vizualizovať. Vhodná vizualizácia by mohla pomôcť pri odstraňovaní 
slabých miest heuristík a priniesť veľa podnetov na tvorbu ďalších heuristík.


\cleardoublepage
\phantomsection
\addcontentsline{toc}{chapter}{Literatúra}

\printbibliography

\backmatter

\newpage
\pagestyle{empty}
\hbox{}