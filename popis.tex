\chapter{Popis softvéru}\label{chap:popis}

V predchádzajúcich kapitolách sme spravili prehľad problematiky a algoritmov, 
ktoré slúžia na hľadanie minimálnej dominujúcej množiny. V tejto kapitole sa 
venujeme popisu a návrhu softvéru, ktorý slúži na porovnanie týchto algoritmov 
a otestovanie týchto algoritmov v praxi.


\section{Analýza existujúcich riešení}

Keďže problém nájdenia minimálnej dominujúcej množiny je podobný problému 
vrcholového pokrytia je často súčasťou aplikácií venujúcim sa zhromažďovaniu 
rôznych grafových algoritmov. Veľa nám známych z nich obsahuje iba základy 
výpočet všetkých možností, alebo jednoduchý aproximačný algoritmus, preto 
tu uvedieme iba jedného zástupcu tohto druhu. Popri vyvíjaní nášho 
softvéru sme pracovali aj s nástrojmi pre analýzu grafov. Tieto sa taktiež 
zväčša vyskytujú v balíkoch s inými nástrojmi pracujúcimi s grafmi (napr. 
vizualizácia). Uvedieme tiež iba jedného zástupcu.

% \todo{
% Cieľom tejto etapy je oboznámenie sa s prostredím, v ktorom bude aplikácia používaná. 
% Dôraz sa kladie na hlbšie pochopenie interakcie aplikácie s prostredím. 
% Ak existujú, treba vykonať aj analýzu podobných systémov.
% }

\subsection{Network Benchmark}

Reprezentantom softvérov na analýzu grafov je Network Benchmark. Ide o softvér, 
ktorý získal podporu aj Alberta Barabásiho. Bol vyvíjaný v rokoch 2005 až 2011. 
Obsahuje príklady, dokumentáciu a podklady k práci s analýzou grafu. Je 
naprogramovaný v jazyku \Java\ a k jeho silným stránkam patrí relatívna 
prehľadnosť pri veľkom množstve nastaveniach a možnostiach analýzy. 

Je dostupný na adrese: \url{http://nwb.cns.iu.edu/index.html}

\begin{figure}
	\centering
	\greybox{%
		\includegraphics[scale=0.3]{obrazky/nwb1}%
	}
	\caption{\emph{Softvér Network Benchmark.} V ľavej hornej časti 
		obrazovky sa nachádzajú výpisy. V dolnej prebiehajúce a ukončené 
		akcie a výpočty. Vpravo sú zhromaždené výsledky výpočtov.}
	\label{img:vis:nwb1}
\end{figure}

\begin{figure}
	\centering
	\greybox{%
		\includegraphics[scale=0.3]{obrazky/nwb2}%
	}
	\caption{\emph{Softvér Network Benchmark.} Tu je zobrazená vizualizácia 
		siete malého sveta. Bublinky znázorňujú jednotlivé klastre.}
	\label{img:vis:nwb2}
\end{figure}

Na obrázku \ref{img:vis:nwb1} je vidno hlavnú obrazovku aplikácie. Práca s 
aplikáciou prebieha tradične tak, že používateľ načíta nejakú grafovú štruktúru, 
následne sa mu zobrazí v stĺpci napravo, označí ju a z bohatej ponuky zvolí 
nejakú analýzu. Vľavo dole, potom vidí priebeh jeho akcií a vľavo hore bližšie 
popisy toho, čo sa práve vykonáva.


\subsection{NetworkX}

Tento softvér je reprezentantom zbierky rôznych algoritmov na grafoch. Je 
vyvíjaný od roku 2005 až doteraz. Posledná stabilná verzia v čase písania 
tejto práce je zo septembra roku 2014. Ide o rozšírenie jazyka Python o novú 
knižnicu.

Sídlo projektu je na adrese: \url{https://networkx.github.io/}

\begin{figure}
	\centering
	\greybox{%
		\includegraphics[scale=0.6]{obrazky/nx}%
	}
	\caption{\emph{Softvér NetworkX.} Tu je zobrazená vizualizácia 
		náhodného grafu s rôznymi pravdepodobnosťami hrán.}
	\label{img:vis:nx}
\end{figure}

Na obrázku \ref{img:vis:nx} je znázornená vizualizácia so softvérom NetworkX. 
Ide o tvorbu náhodného grafu pomocou určenia pravdepodobnosti vzniku hrany. 
Vidno, že aj keď je NetworkX hlavne zbierkou algoritmov, poskytuje peknú 
vizualizáciu. Jeho nedostatkom pre nás je, že na riešenie problému hľadania 
minimálnej dominujúcej množiny poskytuje iba pažravý algoritmus.

\subsection{Gephi}

Gephi je platforma na vizualizáciu grafov. Zameriava sa na vizualizovanie 
grafov a počítanie rôznych hodnôt pre analýzu komplexných sietí. Podporuje veľa 
grafových formátov. Je vo vývoji od roku 2009 \citep{gephi}. Zatiaľ posledná 
verzia vyšla v roku 2013. Platforma je dostupná na internetovej stránke 
\url{http://gephi.github.io/}. Na obrázku \ref{img:vis:gephi} je príklad 
vizualizácie grafu v programe Gephi.

Aj keď platforma je veľmi silným nástrojom na analýzu komplexných sietí, 
neposkytuje žiaden algoritmus na nájdenie minimálnej dominujúcej množiny.

\begin{landscape}
\begin{figure}
	\centering
	\greybox{%
		\includegraphics[scale=0.61]{obrazky/gephi}%
	}
	\caption{\emph{Platforma Gephi.} Vľavo je vizualizácia grafu. Vpravo sú 
		uvedené rôzne vlastnosti grafu.}
	\label{img:vis:gephi}
\end{figure}
\end{landscape}

\subsection{JUNG}

Projekt JUNG (Java Universal Network/Graph Framework) je grafová knižnica 
naprogramovaná v jazyku \Java. Ponúka vlastný jazyk, ktorý poskytuje 
modelovanie, analýzu a vizualizáciu grafov a sietí.

\section{Špecifikácia požiadaviek}

Ako vidno, súčasné softvéry majú dobrú ponuku rôznych algoritmov a veľa 
odlišných vizualizácií. Avšak daň za to, že ponúkajú riešenie na veľa problémov 
je tá, že zväčša je dostupný iba jeden algoritmus na riešenie.

Keďže sa my v práci zaoberáme porovnávaním algoritmov, veľmi nám táto 
skutočnosť nesedí. Preto by sme mali navrhnúť vlastný softvér, ktorý zrejme 
nebude poskytovať riešenia na veľa problémov ale veľa riešení na jeden problém.

%\todo{Čo očakávame od softvéru a prečo je viac fajn ako konkurencia.}

\subsection{Požiadavky}

Naša aplikácia slúži najmä na porovnávanie relatívnej rýchlosti behu algoritmov 
a bude používaná zrejme iba úzkym okruhom ľudí.

Preto sa nám zdalo byť vhodné aplikáciu nechať iba v príkazovom riadku. Znížia 
sa tým odchýlky spôsobené grafickým prostredím. Softvér by mal vedieť spracovať 
graf zadaný v špecifikovanom tvare. Taktiež by mal správne 
implementovať všetky algoritmy popísané v kapitole \ref{chap:algoritmy} a 
poskytnúť informácie o behu daného algoritmu.

Implementované algoritmy by mali zahŕňať:

\begin{itemize}
	\item exaktný algoritmus skúšaním všetkých možností,
	\item exaktný algoritmus skúšaním všetkých možností s heuristikou,
	\item exaktný algoritmus riešený pomocou prevedenia na problém množinového 
		pokrytia,
	\item distribuovaný algoritmus,
	\item upravený distribuovaný algoritmus,
	\item jednoduchý pažravý algoritmus,
	\item pažravý algoritmus s rôznymi heuristikami;
\end{itemize}

\subsection{Prevádzkové požiadavky}

Keďže ide o softvér, ktorý medzi sebou porovnáva jednotlivé algoritmy 
relatívne, tak nepotrebuje špecifický operačný systém. Výkon algoritmov je 
priamo úmerný od hardvéru. To al neznamená, že by hardvér mal byť nejako 
špecifický. Keďže sa súčasné architektúry z nášho pohľadu veľmi nelíšia, 
nekladieme po tejto stránke žiadne požiadavky.

% \todo{
% Východiskom tejto etapy je etapa analýzy. 
% Zaoberá sa úlohami, ktoré má aplikácia zabezpečiť. 
% Oboznámime sa s prostredím, v ktorom bude aplikácia používaná. 
% Zaujímame sa o to, čo chce zadávateľ riešiť, čo požaduje, aby aplikácia vykonávala a pre koho je určená. 
% Zatiaľ nás nezaujíma spôsob realizácie.  
% Cieľom špecifikácie požiadaviek je stanovenie služieb, ktoré zákazník požaduje od systému a ohraničenia na jeho vývoj a prevádzku. 
% Teda zaujímame sa o funkcionálnu stránku systému, potom o to, aké by mali byť vstupy a výstupy systému 
% a s akými údajmi systém bude pracovať, potom o prevádzkové požiadavky 
% (počet a charakteristika používateľov, časová odozva systému, potrebný hardvér a softvér, bezpečnosť a ochrana systému a iné potrebné požiadavky). 
% Prevádzkové požiadavky nazývame aj nefunkcionálne požiadavky. 
% Detailný opis špecifikácie požiadaviek na softvér  najdeme v (1) alebo v štandarde IEEE 830.
% }

\section{Návrh}

Našou hlavnou úlohou je naprogramovať program, ktorý vie spracovať nejaký 
formát vstupu, spraviť z neho graf, vykonať na ňom zvolený algoritmus. Takže 
bude obsahovať moduly na spracovanie týchto požiadaviek. V následujúcich 
častiach popisujeme požiadavky kladené na náš softvér.

% vzťahy medzi časťami softvéru, 
%štruktúru dát a rozhranie systému.

\subsection{Technické požiadavky}

Softvér nevyžaduje žiaden špecifické technické veci. Vstupné a výstupné údaje 
môžu prúdiť cez štandardný systémový vstup a výstup. V prípade lepšej práce 
môžu byť použité knižnice na prácu s nejakými dátovými štruktúrami.

\subsection{Používateľské požiadavky}

Používateľmi tejto aplikácie sú najmä ľudia, ktorí chcú porovnať jednotlivé 
algoritmy. Teda stačí, aby bolo poskytnuté dobré vstupno-výstupné prostredie.

Ak chceme porovnávať medzi sebou rôzne algoritmy na nejakom grafe, alebo ak 
chceme porovnávať jeden algoritmus na rôznych grafoch je dobré mať aj 
prostredie pre dávkové úlohy.

\subsection{Použité technológie}

Keďže algoritmy porovnávame relatívne na sebe, tak môžeme zvoliť ľubovoľný 
programovací jazyk za hlavný. Keďže chceme, aby bola aplikácia ľahko 
spustiteľná na každom operačnom systéme, vhodným kandidátom je programovací 
jazyk \Java .

\Java\ poskytuje aj implementáciu základných dátových štruktúr. Avšak túto možno 
zameniť s nejakými inými externými knižnicami. Na implementáciu rôznych 
algoritmov na tom istom základe je vhodné použiť návrhový vzor \emph{strategy}.

\subsection{Rozhranie aplikácie}

Ako sme už spomenuli, rozhranie aplikácie tvorí príkazový štandardný vstup a 
výstup. Na vstupe pre aplikáciu sú dva argumenty. Algoritmus použitý pri behu a 
cesta k súboru s grafom. Súbor s grafom je textový súbor obsahujúci riadky, 
v ktorých sú dvojice čísel, označujúcich neorientované hrany medzi vrcholmi. 
Na vytvorenie vnútornej reprezentácie grafu teda treba meno súboru, ktorý graf 
obsahuje. výstupom je objekt, s ktorým vie algoritmus pracovať. Algoritmus 
dostane na vstup tento objekt, vypočíta potrebné pomocné dátové štruktúry, 
spustí výpočet a vyráta výsledok. Ten potom vypíše na štandardný výstup. Čo je 
zároveň aj výstupom aplikácie.

Pre algoritmus výpočtu pomocou prevedenia sú potrebné pomocné dátové štruktúry, 
ale v zásade ide o jednoduchý princíp posielania si rôznych reprezentácií grafu 
objektami.

Pre dávkový proces je vstupom textový súbor, ktorý obsahuje riadky. V každom 
riadku je argument pre jeden beh aplikácie. Aplikácia sa spúšťajú za sebou, 
následujúca až vtedy, keď sa predošlá dokončí.


% \todo{
% V tejto etape sa ujasňuje koncepcia systému. 
% Navrhne sa jeho dekompozícia, určia sa vzťahy medzi časťami systému a ohraničenia funkcionality. 
% Túto časť návrhu zvyčajne modelujeme technikami softvérového inžinierstva napr. procesným modelom DFD (Data Flow Diagram, str.24 v (1)) 
% Dekompozícia sa môže urobiť aj na základe iných princípov ako je funkcionálny (str. 53 v (1)). 
% V ďalšom sa identifikuje štruktúra údajov, ktoré do systému vstupujú, ktoré systém spracováva a produkuje. 
% Vzťah medzi údajmi sa modeluje entitno-relačným diagramom, v ktorom sa pomenovávajú vzťahy medzi údajovými entitami. 
% Ak použijeme model DMD (Data Model Diagram), potom v ňom znázorňujeme kvantifikované vzťahy dohodnutými značkami na označenie kardinality. 
% Ak rozložíme vzťah typu M:N pomocou väzobnej entity, potom dostávame fyzický model údajov.  (str.31 v (1))
% V etape návrhu sa navrhne rozhranie systému ( čo do systému vstupuje a čo z neho vystupuje), 
% navrhne sa typ používateľského rozhrania (príkazovo orientované rozhranie, menu, priame riadenie, komunikácia v prirodzenom jazyku ), 
% plán realizácie systému a stanovia sa podmienky, za akých bude používateľ akceptovať produkt. 
% Odhadnú sa potrebné ľudské a materiálne zdroje a navrhne sa postup zaškolovania používateľov.
% }

