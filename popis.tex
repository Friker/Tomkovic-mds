\chapter{Popis softvéru}\label{chap:popis}

V predchádzajúcich kapitolách sme si popísali niektoré dátové štruktúry a 
vybrané algoritmy. Hlavnou náplňou je ich testovanie. 

\section{Analýza existujúcich riešení}

V marci roku 2015 o žiadnych podobných riešeniach nevieme.


% \todo{
% Cieľom tejto etapy je oboznámenie sa s prostredím, v ktorom bude aplikácia používaná. 
% Dôraz sa kladie na hlbšie pochopenie interakcie aplikácie s prostredím. 
% Ak existujú, treba vykonať aj analýzu podobných systémov.
% }

\clearpage
\section{Špecifikácia požiadaviek}

\todo{Čo očakávame od softvéru a prečo je viac fajn ako konkurencia.}

\subsection{Požiadavky}

\todo{Čo má softvér robiť z hladiska funkčnosti.}

\subsection{Prevádzkové požiadavky}

\todo{Čo má softvér robiť z hladiska hardvéru.}

% \todo{
% Východiskom tejto etapy je etapa analýzy. 
% Zaoberá sa úlohami, ktoré má aplikácia zabezpečiť. 
% Oboznámime sa s prostredím, v ktorom bude aplikácia používaná. 
% Zaujímame sa o to, čo chce zadávateľ riešiť, čo požaduje, aby aplikácia vykonávala a pre koho je určená. 
% Zatiaľ nás nezaujíma spôsob realizácie.  
% Cieľom špecifikácie požiadaviek je stanovenie služieb, ktoré zákazník požaduje od systému a ohraničenia na jeho vývoj a prevádzku. 
% Teda zaujímame sa o funkcionálnu stránku systému, potom o to, aké by mali byť vstupy a výstupy systému 
% a s akými údajmi systém bude pracovať, potom o prevádzkové požiadavky 
% (počet a charakteristika používateľov, časová odozva systému, potrebný hardvér a softvér, bezpečnosť a ochrana systému a iné potrebné požiadavky). 
% Prevádzkové požiadavky nazývame aj nefunkcionálne požiadavky. 
% Detailný opis špecifikácie požiadaviek na softvér  najdeme v (1) alebo v štandarde IEEE 830.
% }

\section{Návrh}

\todo{Stručný opis, čo chceme dosiahnuť.}

% vzťahy medzi časťami softvéru, 
%štruktúru dát a rozhranie systému.

\subsection{Technické požiadavky}

\todo{Self-descripting}

\subsection{Používateľské požiadavky}

\todo{Self-descripting}

\subsection{Použité technológie a návrhové vzory atď.}

\todo{Self-descripting}

%Keďže projekt bol pokračovaním predošlej práce za hlavný programovací jazyk 
%bola zvolená \Java. Na grafické znázornenie sú to jednotlivé triedy balíkov 
%AWT a Swing. Na generovanie náhodných reťazcov sme použili špeciálne upravenú 
%triedu s návrhovým vzorom \emph{singleton}. Vďaka celkovej nenáročnosti na 
%technológie a zameraniu projektu nebolo potrebné použiť viac technológií.

\subsection{Rozhranie aplikácie}

\todo{Vstup/výstup -- aj pre jednotlivé časti softvéru}

%Rozhranie vizualizácií bolo podobné ako všetkých iných štruktúr 
%implementovaných v~predošlej verzií softvéru. Vstupné dáta boli zadávané do 
%vstupného textového poľa, alebo klikaním myšou. Podľa stlačeného tlačidla sa 
%spustil daný proces. Výstupne údaje boli prezentované vo vykresľovacom okne 
%pre dátovú štruktúru a vo výpise pre komentáre. Vykreslená štruktúra sa dala 
%zväčšiť, zmenšiť a posunúť pomocou myšky.

%Pre každú dátovu štruktúru bolo potrebné zabezpečiť vstupné textové pole, 
%prípadne dodatočné vstupné metódy a tlačidlá pre všetky operácie.

% \todo{
% V tejto etape sa ujasňuje koncepcia systému. 
% Navrhne sa jeho dekompozícia, určia sa vzťahy medzi časťami systému a ohraničenia funkcionality. 
% Túto časť návrhu zvyčajne modelujeme technikami softvérového inžinierstva napr. procesným modelom DFD (Data Flow Diagram, str.24 v (1)) 
% Dekompozícia sa môže urobiť aj na základe iných princípov ako je funkcionálny (str. 53 v (1)). 
% V ďalšom sa identifikuje štruktúra údajov, ktoré do systému vstupujú, ktoré systém spracováva a produkuje. 
% Vzťah medzi údajmi sa modeluje entitno-relačným diagramom, v ktorom sa pomenovávajú vzťahy medzi údajovými entitami. 
% Ak použijeme model DMD (Data Model Diagram), potom v ňom znázorňujeme kvantifikované vzťahy dohodnutými značkami na označenie kardinality. 
% Ak rozložíme vzťah typu M:N pomocou väzobnej entity, potom dostávame fyzický model údajov.  (str.31 v (1))
% V etape návrhu sa navrhne rozhranie systému ( čo do systému vstupuje a čo z neho vystupuje), 
% navrhne sa typ používateľského rozhrania (príkazovo orientované rozhranie, menu, priame riadenie, komunikácia v prirodzenom jazyku ), 
% plán realizácie systému a stanovia sa podmienky, za akých bude používateľ akceptovať produkt. 
% Odhadnú sa potrebné ľudské a materiálne zdroje a navrhne sa postup zaškolovania používateľov.
% }

